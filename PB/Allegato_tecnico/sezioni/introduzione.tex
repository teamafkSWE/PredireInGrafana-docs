\section{Introduzione}

\subsection{Scopo del documento}
Lo scopo del documento è una descrizione esaustiva delle capacità del software \textit{predire in Grafana} sviluppato dal team AFK.
Il documento si concluderà con un resoconto su quanto sia stato soddisfatto dei vari requisiti.

\subsection{Scopo del prodotto}
Predire in Grafana\glo soddisfa le necessità di monitorare costantemente applicazioni e informazioni contenute in esse. 
Con questo scopo il team AFK si propone per la realizzazione per l’azienda Zucchetti S.p.A. di un tool\glo di addestramento e di un plug-in\glo di monitoraggio per Grafana che utilizzi algoritmi di SVM\glo e Regressione Lineare\glo sul dati in ingresso.


\subsection{Glossario}
Per evitare ambiguità nei documenti formali, viene fornito il documento \textit{Glossario}, contenente tutti i termini considerati di difficile comprensione. Perciò nella documentazione fornita ogni vocabolo contenuto nel Glossario è contrassegnato dalla lettera G a pedice.

\subsection{Riferimenti}

\subsubsection{Riferimenti normativi}
\begin{itemize}
	\item
	\textbf{Capitolato Appalto C4 :} https://www.math.unipd.it/~tullio/IS-1/2019/Progetto/C4.pdf
	\item 
	Norme di progetto v3.0.0
	\item
	Aggiungere verbali
\end{itemize}

\subsubsection{Riferimenti informativi}
	\item 
	Analisi di requisiti v3.0.0
	