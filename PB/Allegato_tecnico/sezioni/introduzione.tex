\section{Introduzione}

\subsection{Scopo del documento}
Lo scopo del documento è una descrizione esaustiva delle capacità del software \textit{Predire in Grafana} sviluppato dal team AFK.
Il documento si concluderà con un resoconto su quanto sia stato soddisfatto dei vari requisiti.

\subsection{Scopo del prodotto}
Predire in Grafana\glo soddisfa le necessità di monitorare costantemente applicazioni e informazioni contenute in esse. 
Con questo scopo il team AFK si propone per la realizzazione per l’azienda \textit{Zucchetti SPA} di un tool\glo di addestramento e di un plug-in\glo di monitoraggio per Grafana che utilizzi algoritmi di SVM\glo e Regressione Lineare\glo sui dati in ingresso.


\subsection{Glossario}
Per evitare ambiguità nei documenti formali, viene fornito il documento \textit{Glossario}, contenente tutti i termini considerati di difficile comprensione. Perciò nella documentazione fornita ogni vocabolo contenuto nel Glossario è contrassegnato dalla lettera G a pedice.

\subsection{Riferimenti}

\subsubsection{Riferimenti normativi}
\begin{itemize}
	\item \textbf{Capitolato d'appalto C4}: \\
	\url{https://www.math.unipd.it/~tullio/IS-1/2019/Progetto/C4.pdf};
	\item \textit{norme\_di\_progetto\_v3.0.0};
	\item \textit{VI\_2020-06-01}.
\end{itemize}

\subsubsection{Riferimenti informativi}
\begin{itemize}
	\item \textit{analisi\_dei\_requisiti\_v3.0.0};
	\item \textbf{Model-View Patterns - Materiale didattico del corso di Ingegneria del Software}: \\
	\url{https://www.math.unipd.it/~rcardin/sweb/2020/L02.pdf};
	\item \textbf{Diagrammi delle classi - Materiale didattico del corso di Ingegneria del Software}: \\
	\url{https://www.math.unipd.it/~tullio/IS-1/2019/Dispense/E01b.pdf};
	\begin{itemize}
		\item proprietà e operazioni, slide 11 - 34;
		\item caratteristiche, slide 35 - 38.
	\end{itemize}
	\item \textbf{Diagrammi dei package - Materiale didattico del corso di Ingegneria del Software}: \\
	\url{https://www.math.unipd.it/~tullio/IS-1/2019/Dispense/E01c.pdf};
	\begin{itemize}
		\item package e dipendenze, slide 12 - 15.
	\end{itemize}
	\item \textbf{Diagrammi di sequenza - Materiale didattico del corso di Ingegneria del Software}: \\
	\url{https://www.math.unipd.it/~tullio/IS-1/2019/Dispense/E02a.pdf};
	\begin{itemize}
		\item partecipanti e segnali, slide 8 - 16;
		\item modellazione, slide 18 - 21.
	\end{itemize}
	\item \textbf{Design Pattern Comportamentali - Materiale didattico del corso di Ingegneria del Software}: \\
	\url{https://www.math.unipd.it/~tullio/IS-1/2019/Dispense/E08.pdf}
	\begin{itemize}
		\item strategy, slide 6 - 8.
	\end{itemize}
\end{itemize}