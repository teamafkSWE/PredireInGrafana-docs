\section{Riscontro dei rischi}
Di seguito vengono riportati i rischi in cui il gruppo si è imbattuto durante lo svolgimento del progetto, suddivisi per periodi. \\
Nella colonna Efficacia viene rappresentato quanto la contromisura adottata è stata efficace a risolvere il rischio incontrato, essa può essere:
\begin{itemize}
	\item ottima;
	\item sufficiente;
	\item insufficiente.
\end{itemize}

\subsection{Rischi nella fase di Analisi}
I seguenti rischi sono stati riscontrati durante il periodo di analisi. \\
\textit{Periodo: da 2020-03-16 a 2020-04-13}

\begin{longtable}{C{3cm} L{4.5cm} L{5.5cm} C{2cm}}
\rowcolor{white}\caption{Attualizzazione dei rischi - Analisi} \\
		\rowcolor{redafk}
\textcolor{white}{\textbf{Rischio}} &
\textcolor{white}{\textbf{Descrizione}} &
\textcolor{white}{\textbf{Contromisura}} &
\textcolor{white}{\textbf{Efficacia}}\\
		\endfirsthead
		\rowcolor{white}\caption[]{(continua)} \\
		\rowcolor{redafk}
\textcolor{white}{\textbf{Rischio}} &
\textcolor{white}{\textbf{Descrizione}} &
\textcolor{white}{\textbf{Contromisura}} &
\textcolor{white}{\textbf{Efficacia}}\\
		\endhead

RiO01 - Emergenza sanitaria	& L'epidemia ha costretto gli stakeholders ad attuare lo smart working. & Sono stati usati vari mezzi di comunicazione, in particolare si ha optato per applicazioni che permettessero comunicazioni rapide e già conosciute così da ridurre il disagio al minimo. & Ottima
\\
RiO06 - Divisione errata del lavoro & Durante la suddivisione dei compiti alcuni sono stati sottovalutati. & Il responsabile, una volta informato sugli errori di valutazione,ha proceduto ad individuare una migliore suddivisione. Per ridurre l'occorrenza di questo rischio il gruppo cercherà di fare spesso incontri di pochi minuti in cui discutere l'avanzamento del proprio compito e la necessità o possibilità di ricevere o dare aiuti. & Ottima
\\
RiO07 - Errata analisi dei requisiti & Durante l'analisi sono sorti alcuni dubbi sui requisiti esposti dal proponente. & Il gruppo ha proceduto ad effettuare degli incontri con il proponente per poter chiarire tutti i dubbi rilevati. & Ottima
\\

\end{longtable}


\subsection{Rischi nella fase di Progettazione e codifica per la Technology Baseline}
I seguenti rischi sono stati riscontrati durante il periodo di progettazione e codifica per la Technology Baseline. \\
\textit{Periodo: da 2020-04-21 a 2020-05-11}


\begin{longtable}{C{3cm} L{4.5cm} L{5.5cm} C{2cm}}
\rowcolor{white}\caption{Attualizzazione dei rischi - Progettazione e codifica per la Technology Baseline} \\
		\rowcolor{redafk}
\textcolor{white}{\textbf{Rischio}} &
\textcolor{white}{\textbf{Descrizione}} &
\textcolor{white}{\textbf{Contromisura}} &
\textcolor{white}{\textbf{Efficacia}}\\
		\endfirsthead
		\rowcolor{white}\caption[]{(continua)} \\
		\rowcolor{redafk}
\textcolor{white}{\textbf{Rischio}} &
\textcolor{white}{\textbf{Descrizione}} &
\textcolor{white}{\textbf{Contromisura}} &
\textcolor{white}{\textbf{Efficacia}}\\
		\endhead

RiT01 - Inesperienza tecnologica & I programmatori non conoscevano a pieno i linguaggi e le librerie che sono state utilizzate & \'E stato suddiviso il lavoro in modo da rispettare le conoscenze dei membri. In caso di nessuna conoscenza precedente, si è suddiviso il compito di studiare le documentazioni, per poi spiegarle agli altri membri. & Ottima
\\
RiT04 - Configurazione dell'ambiente di lavoro & Alcuni membri con SO\glo Unix/Linux hanno riscontrato problemi nel far individuare a Grafana il plugin di test che era stato creato. & Si è consultata a fondo la documentazione individuando così le impostazioni da cambiare. & Sufficiente
\\
RiT02 - Errori nelle dipendenze & Un cambiamento all'interno degli strumenti forniti da Grafana per sviluppare il plugin ha causato l'impossibilità di effettuare la build del prodotto. & Si è proceduto al passaggio ad una versione precedente di tali strumenti. & Ottima
\\
RiO01 - Emergenza sanitaria	& L'epidemia ha costretto gli stakeholders ad attuare lo smart working. & Sono stati usati vari mezzi di comunicazione, in particolare si ha optato per applicazioni che permettessero comunicazioni rapide e già conosciute così da ridurre il disagio al minimo. & Ottima
\\
RiO03 - Impegni accademici & Un membro del gruppo ha dovuto svolgere un esame & Durante la breve mancanza di un membro il resto del gruppo si è dedicato all'approfondimento e allo studio delle tecnologie utilizzate. & Ottima
\\
RiO08 - Suddivisione delle ore di lavoro & La suddivisione delle ore per questa fase non è stata rispettata totalmente & Sono stati riscontrati problemi principalmente nelle ore di programmazione, ogni membro è stato informato dei problemi riscontrati e della loro soluzione in modo tale da evitare il loro ripresentarsi nelle fasi successive, evitando così di rallentare ulteriormente il lavoro. & Ottima
\\

\end{longtable}


\subsection{Rischi nella fase di Progettazione di dettaglio e codifica}
I seguenti rischi sono stati riscontrati durante il periodo di progettazione di dettaglio e codifica. \\
\textit{Periodo: da 2020-05-11 a 2020-06-11}


\begin{longtable}{C{3cm} L{4.5cm} L{5.5cm} C{2cm}}
\rowcolor{white}\caption{Attualizzazione dei rischi - Progettazione di dettaglio e codifica} \\
		\rowcolor{redafk}
\textcolor{white}{\textbf{Rischio}} &
\textcolor{white}{\textbf{Descrizione}} &
\textcolor{white}{\textbf{Contromisura}} &
\textcolor{white}{\textbf{Efficacia}}\\
		\endfirsthead
		\rowcolor{white}\caption[]{(continua)} \\
		\rowcolor{redafk}
\textcolor{white}{\textbf{Rischio}} &
\textcolor{white}{\textbf{Descrizione}} &
\textcolor{white}{\textbf{Contromisura}} &
\textcolor{white}{\textbf{Efficacia}}\\
		\endhead

RiO01 - Emergenza sanitaria	& L'epidemia ha costretto gli stakeholders ad attuare lo smart working. & Sono stati usati vari mezzi di comunicazione, in particolare si ha optato per applicazioni che permettessero comunicazioni rapide e già conosciute così da ridurre il disagio al minimo. & Ottima
\\


\end{longtable}


\begin{comment}
\subsection{Rischi nella Fase di Validazione e collaudo}
I seguenti rischi sono stati riscontrati durante il periodo di Validazione e collaudo. \\
\textit{Periodo: da 2020-06-19 a 2020-07-06}

\begin{longtable}{C{3cm} L{4.5cm} L{5.5cm} C{2cm}}
\rowcolor{white}\caption{Attualizzazione dei rischi - Validazione e collaudo} \\
		\rowcolor{redafk}
\textcolor{white}{\textbf{Rischio}} &
\textcolor{white}{\textbf{Descrizione}} &
\textcolor{white}{\textbf{Contromisura}} &
\textcolor{white}{\textbf{Efficacia}}\\
		\endfirsthead
		\rowcolor{white}\caption[]{(continua)} \\
		\rowcolor{redafk}
\textcolor{white}{\textbf{Rischio}} &
\textcolor{white}{\textbf{Descrizione}} &
\textcolor{white}{\textbf{Contromisura}} &
\textcolor{white}{\textbf{Efficacia}}\\
		\endhead

RiO01 - Emergenza sanitaria	& L'epidemia ha costretto gli stakeholders ad attuare lo smart working. & Sono stati usati vari mezzi di comunicazione, in particolare si ha optato per applicazioni che permettessero comunicazioni rapide e già conosciute così da ridurre il disagio al minimo. & Ottima
\\



\end{longtable}


\end{comment}