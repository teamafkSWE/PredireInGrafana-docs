\section{Introduzione}

\subsection{Premessa}
Per stabilire le varie attività il gruppo si è basato sui processi, sui bisogni e sui vincoli di dipendenza che intervengono nel progetto. In questo modo è stato possibile stabilire per ciascuna attività il tempo e le persone da impiegare visto che si tratta di risorse fondamentali per la vita di qualunque progetto.

\subsection{Scopo del documento}
Il seguente documento definirà le attività che saranno svolte nel progetto e la loro collocazione nel tempo.\\
Nello specifico il documento è così strutturato:
\begin{itemize}
\item analisi dei rischi;
\item descrizione modello di sviluppo;
\item collocazione membri nelle attività;
\item stima delle risorse per lo sviluppo del progetto.
\end{itemize}
\subsection{Scopo del prodotto}
Lo sviluppo del prodotto vedrà la realizzazione di un plug-in per la piattaforma Grafana\glo e  di un tool di addestramento standalone\glo che permetteranno rispettivamente di monitorare e predire lo stato di un sistema in analisi e di creare un file JSON contenente i parametri necessari per il funzionamento della predizione. Grazie alle predizioni sarà possibile attivare l'invio di messaggi di allarme così da poter gestire preventivamente eventuali situazioni di rischio. \\
Il plug-in\glo utilizzerà la Support Vector Machine\glo (SVM) per poter effettuare categorizzazioni sui dati forniti e la regressione lineare\glo per la predizione.
\begin{comment}
I due plug-in\glo utilizzeranno la Support Vector Machine\glo (SVM) o la Regressione Lineare per classificazione o regressione sui dati forniti.
\end{comment}

\subsection{Glossario}
Per evitare ambiguità nei documenti formali viene fornito il documento \textit{Glossario}, contenente tutti i termini considerati di difficile comprensione. Perciò nella documentazione fornita ogni vocabolo contenuto nel Glossario è contrassegnato dalla lettera G a pedice.

\subsection{Riferimenti}
\subsubsection{Riferimenti normativi}
\begin{itemize}
	\item Norme di Progetto: \textit{Norme\_di\_Progetto\_v2.0.0}.
	\item Capitolato d'appalto C4: \url{https://www.math.unipd.it/~tullio/IS-1/2019/Progetto/C4.pdf}.
\end{itemize}
\subsubsection{Riferimenti informativi}
\begin{itemize}
	\item \textbf{Slide L06 del corso Ingegneria del Software - Gestione di Progetto}: \\
	\url{https://www.math.unipd.it/~tullio/IS-1/2019/Dispense/L06.pdf};
	\item \textbf{Ingegneria del Software - Ian Sommerville - 10\textsuperscript{a} Edizione.}\\
	Capitoli di riferimento: \begin{itemize}
	\item \S 19 - Gestione della progettazione;
	\item \S 20 - Pianificazione della progettazione.
	\end{itemize}
\end{itemize}
\subsection{Scadenze}
Il gruppo \textit{TeamAFK} si impegna a presentare il proprio materiale nei seguenti appuntamenti:\\
\begin{itemize}
\item \textbf{Revisione dei Requisiti}: 2020-04-20;
\item \textbf{Revisione di Progettazione}: 2020-05-18;
\item \textbf{Revisione di Qualifica}: 2020-06-18;
\item \textbf{Revisione di Accettazione}: 2020-07-13. 
\end{itemize}
