\section{Specifica dei test}
Per verificare la qualità del prodotto software, il gruppo fornitore ha deciso di adottare il \textbf{Modello di Sviluppo a V}\glo, sviluppando così una serie di test. Questi hanno lo scopo di controllare che tutte le unità di cui è composto il sistema siano state implementate correttamente, rispettando tutti gli aspetti del progetto.
Per semplificare la loro consultazione i test saranno suddivisi in categorie, per mezzo di tabelle, mostrando l'output prodotto e sottolineando se è un risultato atteso o non atteso.
\subsection{Stato dei test}
Per definire lo stato dei test, si usano le seguenti sigle:
\begin{itemize}
\item \textbf{I}: test implementato;
\item \textbf{NI}: test non implementato.
\end{itemize}

%non mi convince scrivere questa riga, alla fine loro non sanno come sono stati implementati...
%La maggior parte dei test di seguito descritti sono stati verificati attraverso la funzione js \texttt{console.log()}, che mostra in console il risultato voluto.

\subsection{Test di accettazione}

\begin{longtable}{C{2.5cm} L{8cm} C{2cm}}
\rowcolor{white}\caption{Tabella dei test di accettazione} \\
		\rowcolor{redafk}
\textcolor{white}{\textbf{Codice}} &
\textcolor{white}{\textbf{Descrizione}} &
\textcolor{white}{\textbf{Esito}} \\
		\endfirsthead
		\rowcolor{white}\caption[]{(continua)} \\
		\rowcolor{redafk}
\textcolor{white}{\textbf{Codice}} &
\textcolor{white}{\textbf{Descrizione}} &
\textcolor{white}{\textbf{Esito}} \\
		\endhead
TAOF1 & Verificare che l’utente possa addestrare gli algoritmi di previsione sull’applicazione. & I \\
TAOF1.1 & Verificare che l’utente possa selezionare e caricare, dal suo dispositivo, un file CSV contenente i dati su cui effettuare l’addestramento. & I\\
TAOF1.1.1 & Verificare che l’inserimento di un file CSV non valido venga visualizzato un messaggio d’errore. & I \\
TAOF1.2 & Verificare che l’utente possa selezionare e caricare, dal suo dispositivo, un file JSON contenente la configurazione di un addestramento precedentemente eseguito. & I\\
TAOF1.2.1 & Verificare che l’inserimento di un file JSON non valido venga visualizzato un messaggio d’errore. & NI \\
TADF2 & Verificare che l’utente possa visualizzare un grafico a dispersione che rappresenti i dati utilizzati per l’addestramento nel tool. & I \\
TAOF3 & Verificare che l’utente possa scegliere quale algoritmo utilizzare per effettuare l’addestramento dei dati. & I \\
TAOF4 & Verificare che l’utente posso avviare l’addestramento dell’algoritmo di predizione scelto utilizzando i dati inseriti. & I \\
TAOF5 & Verificare che, alla fine del processo di addestramento, venga visualizzato un messaggio di operazione completata con successo. & I \\
TAOF6 & Verificare che l’utente, alla fine del processo di addestramento, riceva un file JSON contenente il risultato dell’addestramento. & I\\
TAOF7 & Verificare che l’utente possa avviare il plug-in di Grafana. & I \\
TAOF8 & Verificare che l’utente possa caricare il file JSON ottenuto dall’addestramento effettuato dal tool. & I\\
TAOF9 & Verificare che l’utente possa collegare i predittori letti dal file JSON al flusso dati. & I \\
TAOF9.1 & Verificare che l’utente possa selezionare un flusso di dati su cui eseguire delle previsioni. & I\\
TAOF9.2 & Verificare che l’utente possa visualizzare un messaggio che conferma il successo nel collegamento dei nodi al flusso dati. & I\\
TAOF9.3 & Verificare che, se il collegamento dei nodi al flusso dati non va a buon file, l’utente visualizzi un messaggio di errore. & I \\
TAOF10 & Verificare che l’utente possa visualizzare il grafico dei risultati della previsione all’interno di una dashboard precedentemente configurata. & I\\
TAOF11 & Verificare che l’utente possa fermare l’esecuzione del plug-in premendo il relativo bottone "Interrompi monitoraggio". & I \\
TAOF12 & Verificare che l'utente possa inserire i valori di soglia ad uno specifico collegamento. & NI \\
TAOF13 & Verificare che l'utente possa modificare un collegamento precedentemente creato. & I \\
TAOF14 & Verificare che l'utente possa eliminare un collegamento precedentemente creato. & I \\
TAOF15 & Verificare che l'utente possa salvare la previsione. & NI \\

\end{longtable}

\subsection{Test di sistema}

\begin{longtable}{C{2.5cm} C{2.5cm} L{8cm} C{2cm}}
\rowcolor{white}\caption{Tabella dei test di sistema} \\
		\rowcolor{redafk}
\textcolor{white}{\textbf{Codice}} &
\textcolor{white}{\textbf{Caso d'uso}} &
\textcolor{white}{\textbf{Descrizione}} &
\textcolor{white}{\textbf{Esito}} \\
		\endfirsthead
		\rowcolor{white}\caption[]{(continua)} \\
		\rowcolor{redafk}
\textcolor{white}{\textbf{Codice}} &
\textcolor{white}{\textbf{Caso d'uso}} &
\textcolor{white}{\textbf{Descrizione}} &
\textcolor{white}{\textbf{Esito}} \\
		\endhead

%-------------------------------------------- Simo		
TSOF1 & UC1 &
L'utente  deve poter creare il file JSON\glo contenente il/i predittore/i\glo. \newline
All'utente viene chiesto di:
\begin{itemize}
	\item scegliere i dati di addestramento\glo da caricare;
	\item selezionare l’algoritmo di previsione\glo;
	\item conferma delle operazioni;
	\item salvataggio file JSON contenente i predittori.
\end{itemize} & I \\

TSOF1.1 & UC1.1 &
L'utente  deve poter scegliere i dati di addestramento. \newline All'utente viene chiesto di:
\begin{itemize}
	\item cliccare il pulsante "Carica dati di addestramento";
 	\item verificare che si apra la finestra che visualizza il file system\glo.
 	\item verificare che dalla finestra di dialogo siano visibili solo file CSV\glo;
	\item selezionare i dati di addestramento.
\end{itemize} 
& I \\
 
TSOF1.2 & UC1.2 & 
L'utente deve poter scegliere l'algoritmo di predizione. \newline All'utente viene chiesto di:
\begin{itemize}
	\item cliccare sulla Combo Box\glo con etichetta "Seleziona algoritmo";
	\item scegliere uno degli algoritmi proposti (RL o SVM).
\end{itemize} & I \\
 
TSOF1.3 & UC1.3 & 
L'utente deve poter confermare la scelta dell'algoritmo. \newline All'utente viene chiesto di:
\begin{itemize}
	\item cliccare sul pulsante "Conferma".
\end{itemize} & I \\

TSOF1.3.1 & UC7 & 
L'utente deve poter visualizzare un messaggio di errore se non era stato inserito nessun csv. \newline All'utente viene chiesto di:
\begin{itemize}
	\item verificare la visualizzazione del messaggio;
	\item verificare di essere rimandati al TSOF1.1.
\end{itemize} & I \\

TSOF1.3.2 & UC8 & 
L'utente deve poter visualizzare un messaggio di errore se non era stato scelto nessun algoritmo. \newline All'utente viene chiesto di:
\begin{itemize}
	\item verificare la visualizzazione del messaggio;
	\item verificare di essere rimandati al TSOF1.2.
\end{itemize} & I \\

TSOF1.3.3 & UC9 & 
L'utente deve poter visualizzare un messaggio d'errore se la scelta dell'algoritmo non è compatibile con i dati di addestramento. \newline All'utente viene chiesto di:
\begin{itemize}
	\item verificare la visualizzazione dell'errore;
	\item verificare di essere rimandati al TSOF1.1.
\end{itemize} & I \\

TSOF1.4 & UC1.4 & 
L'utente deve poter visualizzare il messaggio di notifica di avvenuto addestramento. \newline All'utente viene chiesto di:
\begin{itemize}
	\item verificare la visualizzazione del messaggio "Addestramento avvenuto con successo";
	\item verificare che si possa procedere con TSOF1.6.
\end{itemize} & I \\

TSOF1.5 & UC1.5 & 
L'utente deve poter visualizzare il messaggio di alert dell'addestramento non avvenuto. \newline All'utente viene chiesto di:
\begin{itemize}
	\item verificare la visualizzazione del messaggio "Addestramento non riuscito";
	\item verificare di essere rimandati a TSOF1.3.1 oppure TSOF1.3.2 oppure TSOF1.3.3.
\end{itemize} & I \\

TSOF1.6 & UC1.6 & 
L'utente deve poter salvare il file JSON in locale contenente predittori. \newline All'utente viene chiesto di:
\begin{itemize}
	\item cliccare sul pulsante "Download".
	\item verificare che il file venga salvato in locale.
\end{itemize} & I \\

TSOF2 & UC2 &
L'utente deve poter caricare il file JSON nel plug-in. \newline All'utente viene chiesto di:
\begin{itemize}
	\item cliccare il pulsante per caricare il file JSON;
	\item selezionare il file JSON;
	\item confermare il caricamento del file.
\end{itemize} & I	\\

TSOF2.1 & UC2.1 &
L'utente deve poter selezionare il file JSON. \newline All'utente viene chiesto di:
\begin{itemize}
	\item cliccare su "Carica JSON";
	\item verificare che siano visibili solo file JSON;
	\item selezionare il file dalla finestra di dialogo.
\end{itemize} & I	\\

TSOF2.1.1 & UC10 &
L'utente deve poter visualizzare il messaggio di alert\glo del caricamento già avvenuto e caricare nuovamente il file. \newline All'utente viene chiesto di:
\begin{itemize}
	\item visualizzare il messaggio di alert "File JSON già caricato";
	\item cliccare su "Conferma" per sovrascrivere il file.
\end{itemize} & NI	\\

TSOF2.1.2 & UC10 &
L'utente deve poter visualizzare il messaggio di alert del caricamento già avvenuto e annullare il caricamento. \newline All'utente viene chiesto di:
\begin{itemize}
	\item visualizzare il messaggio di alert "File JSON già caricato";
	\item cliccare su "Annulla" per tornare alla sezione di caricamento.
\end{itemize} & NI	\\

TSOF2.2 & UC2.2 &
L'utente deve poter confermare il caricamento del file. \newline All'utente viene chiesto di:
\begin{itemize}
	\item cliccare sul pulsante "Conferma".
\end{itemize} & I	\\


TSOF2.2.1 & UC11 &
L'utente deve poter visualizzare un messaggio d'errore in caso di problemi con il caricamento. \newline All'utente viene chiesto di:
\begin{itemize}
	\item visualizzare il messaggio d'errore "Struttura del file JSON non supportata";
	\item cliccare il pulsante "Conferma";
	\item verificare di essere ritornato alla selezione del file.
\end{itemize} & NI	\\

TSOF2.3 & UC2.3 &
L'utente deve poter visualizzare un messaggio di notifica di caricamento avvenuto con successo. \newline All'utente viene chiesto di:
\begin{itemize}
	\item visualizzare il messaggio di notifica "Avvenuto successo caricamento file JSON";
	\item cliccare il pulsante "Continua".
\end{itemize} & NI	\\

TSOF2.4 & UC2.4 &
L'utente deve poter visualizzare il contenuto del file JSON appena caricato. \newline All'utente viene chiesto di:
\begin{itemize}
	\item verificare che sia visibile il contenuto del file caricato;
	\item verificare che quello visualizzato corrisponda all'effettivo contenuto del file.
\end{itemize} & I	\\
%-------------------------------------------- Olly
TSOF3 & 
UC3 &
L'utente  deve poter collegare un predittore ad un flusso. In particolare l'utente deve:
\begin{itemize}
	\item selezionare uno o più predittori scegliendoli tra quelli disponibili in una lista che verrà visualizzata una volta caricato il file JSON;
	\item selezionare il nodo del flusso dati da associare al predittore;
	\item poter impostare delle soglie sui predittori;
	\item confermare le impostazioni di collegamento selezionate.
\end{itemize} &
I \\ 

TSOF3.1 &
UC3.1 &
L'utente  deve poter selezionare il predittore da associare al flusso. All'utente viene chiesto di:
\begin{itemize}
	\item visualizzare l'elenco dei predittori;
	\item verificare di poter selezionare il/i predittore/i desiderato/i;
\end{itemize}&
I \\

TSOF3.2 &
UC3.2 &
L'utente deve poter selezionare un nodo\glo del flusso. All'utente viene chiesto di:
\begin{itemize}
	\item verificare di poter selezionare il nodo desiderato;
	\item verificare di aver a disposizione il nodo desiderato;
	\item selezionare il nodo da associare.
\end{itemize}&
I \\

TSOF3.3 &
UC3.3 &
L'utente  deve poter stabilire una o più soglie\glo al collegamento. All'utente viene chiesto di:
\begin{itemize}
	\item verificare se la funzionalità è disponibile;
	\item impostare la soglia desiderata.
\end{itemize}&
I \\

TSOF3.4 &
UC3.4 &
L'utente  deve poter confermare il collegamento e vedere la lista dei collegamenti. All'utente viene chiesto di:
\begin{itemize}
	\item poter visualizzare e cliccare il pulsante etichettato "Conferma collegamento";
	\item verificare l'effettiva conferma del collegamento;
	\item verificare la possibilità di effettuare un altro collegamento.
\end{itemize}&
I \\

TSOF3.4.1 &
UC12 &
L'utente  deve poter visualizzare il messaggio d'errore sulla soglia stabilita. All'utente viene chiesto di:
\begin{itemize}
	\item poter visualizzare il messaggio "Errore impostazione soglia non valida";
	\item poter cliccare il pulsante "Conferma";
	\item verificare che dopo il click sul pulsante "Conferma", sia possibile impostare la soglia.
\end{itemize} &
I \\

TSOF3.4.2 &
UC13 &
L'utente  deve poter visualizzare il messaggio d'errore sulle impostazioni di collegamento. All'utente viene chiesto di:
\begin{itemize}
	\item poter visualizzare il messaggio "Errore impostazione di collegamento";
	\item poter cliccare il pulsante "Conferma";
	\item verificare che dopo il click sul pulsante "Conferma", sia possibile impostare il/i campi dato/i errato/i.
\end{itemize}&
I \\

TSOF3.5 &
UC3.5 &
L’utente deve poter visualizzare il messaggio di notifica per la buona riuscita del collegamento. All’utente viene chiesto di:
\begin{itemize}
	\item visualizzare il messaggio "Collegamento avvenuto con successo";
	\item poter visualizzare e cliccare il pulsante "Conferma".
\end{itemize}&
I \\

TSOF3.6 &
UC3.6 &
L’utente deve poter visualizzare la lista di collegamenti effettuati. All’utente viene chiesto di:
\begin{itemize}
	\item visualizzare la lista di collegamenti correttamente inseriti fino a quel momento;
	\item visualizzare che siano presenti i pulsanti di scollegamento e modifica accanto ad ogni collegamento.
\end{itemize}&
I \\

% ------------- FINE TEST 3 ------------------
% ------------- INIZIO TEST 4 ----------------

TSOF4 &
UC4 &
L'utente deve poter effettuare delle operazioni sui collegamenti. All'utente viene chiesto di:
\begin{itemize}
	\item cliccare sul bottone di scollegamento;
	\item confermare o annullare l'operazione;
	\item verificare l'effettiva esecuzione dell'operazione.
\end{itemize}&
I \\


TSOF4.1 &
UC4.1 &
L'utente deve poter scollegare un collegamento. All'utente viene chiesto di:
\begin{itemize}
	\item cliccare sul pulsante di scollegamento;
	\item verificare di poter procedere con TSOF4.2.
\end{itemize}&
I \\

TSOF4.2 &
UC4.2 &
L'utente deve poter visualizzare il messaggio di alert prima di continuare con lo scollegamento. All'utente viene chiesto di:
\begin{itemize}
	\item verificare che sia visibile il messaggio d'alert;
	\item verificare di poter procedere con TSOF4.3.
\end{itemize}&
I \\

TSOF4.3 &
UC4.3 &
L'utente deve poter confermare o annullare lo scollegamento. All'utente viene chiesto di:
\begin{itemize}
	\item cliccare sul pulsante di conferma per procedere;
	\item cliccare sul pulsante di annullamento per interrompere l'operazione;
	\item verificare l'avvenuta operazione selezionata.
\end{itemize}&
I \\

TSOF4.4 &
UC4.4 &
L'utente deve poter visualizzare il messaggio di notifica dell'avvenuto scollegamento. All'utente viene chiesto di:
\begin{itemize}
	\item verificare che sia visibile il messaggio "Scollegamento avvenuto con successo";
	\item verificare di poter procedere con TSOF4.5.
\end{itemize}&
I \\

TSOF4.5 &
UC4.5 &
L'utente deve poter modificare un collegamento già inserito. All'utente viene chiesto di:
\begin{itemize}
	\item cliccare il pulsate "Modifica collegamento";
	\item verificare di poter modificare il collegamento in tutte le sue parti;
	\item modificare il collegamento e confermare le nuove impostazioni.
\end{itemize}&
I \\

% ------------- FINE TEST 4 ------------------

%-------------------------------------------- Davide		


TSOF5 &
UC5 &
L'utente deve poter avviare il calcolo delle previsioni. All'utente viene chiesto di:
\begin{itemize}
	\item cliccare sul pulsante di avvio monitoraggio;
	\item verificare che sia possibile interromperlo;
	\item cliccare sul pulsante di salvataggio per salvare le previsioni nel database.
\end{itemize}&
NI \\

TSOF5.1 &
UC5.1 &
L'utente deve poter avviare il monitoraggio del flusso dati. All'utente viene chiesto di:
\begin{itemize}
	\item cliccare il pulsante "Avvia monitoraggio".
\end{itemize}&
I \\

TSOF5.1.1 &
UC14 &
L'utente deve poter visualizzare il messaggio d'errore in caso nessun predittore fosse stato collegato. All'utente viene chiesto di:
\begin{itemize}
	\item visualizzare il messaggio d'errore;
	\item verificare di essere rimandato al TSOF3.
\end{itemize}&
NI \\

TSOF5.2 &
UC5.2 &
L'utente deve poter visualizzare il messaggio di notifica di monitoraggio avviato. All'utente viene chiesto di:
\begin{itemize}
	\item visualizzare il messaggio di notifica "Monitoraggio avviato con successo";
	\item verificare di poter proseguire con TSOF5.3.
\end{itemize}&
NI \\

TSOF5.3 &
UC5.3 &
L'utente deve poter interrompere il monitoraggio. All'utente viene chiesto di:
\begin{itemize}
	\item verificare che sia presente il pulsante di interruzione del monitoraggio;
	\item cliccare sul pulsante "Interrompi monitoraggio".
\end{itemize}&
I \\

TSOF5.4 &
UC5.4 &
L'utente deve poter visualizzare il messaggio di notifica dell'interruzione del monitoraggio. All'utente viene chiesto di:
\begin{itemize}
	\item visualizzare il messaggio di notifica "Monitoraggio interrotto";
	\item verificare di poter proseguire con TSOF5.5.
\end{itemize}&
NI \\

TSOF5.5 &
UC5.5 &
L'utente deve poter salvare le previsioni. All'utente viene chiesto di:
\begin{itemize}
	\item cliccare sul pulsante di salvataggio delle previsioni.
\end{itemize}&
NI \\

TSOF5.6 &
UC5.6 &
L'utente deve poter visualizzare il messaggio di notifica di successo salvataggio. All'utente viene chiesto di:
\begin{itemize}
	\item visualizzare il messaggio di notifica.
\end{itemize}&
NI \\

TSOF6 &
UC6 &
L'utente deve poter visualizzare le previsioni nella dashboard. All'utente viene chiesto di:
\begin{itemize}
	\item verificare che nella dashboard siano presenti le previsioni calcolate.
\end{itemize}&
I \\

TSOF6.1 &
UC6.1 &
L'utente deve poter visualizzare le previsioni. All'utente viene chiesto di:
\begin{itemize}
	\item verificare che siano presenti le previsioni calcolate;
	\item verificare che le previsioni si aggiornino in base alla politica temporale selezionata.
\end{itemize}&
I \\

TSOF6.2 &
UC6.2 &
L'utente deve poter visualizzare il messaggio di alert in caso una soglia critica sia raggiunta. All'utente viene chiesto di:
\begin{itemize}
	\item visualizzare il messaggio d'alert "Soglia critica raggiunta".
\end{itemize}&
I \\

\end{longtable}

\subsection{Test di integrazione}

\begin{longtable}{C{2cm} L{8cm} C{2cm}}
\rowcolor{white}\caption{Tabella dei test di integrazione} \\
		\rowcolor{redafk}
\textcolor{white}{\textbf{Codice}} &
\textcolor{white}{\textbf{Descrizione}} &
\textcolor{white}{\textbf{Esito}} \\
		\endfirsthead
		\rowcolor{white}\caption[]{(continua)} \\
		\rowcolor{redafk}
\textcolor{white}{\textbf{Codice}} &
\textcolor{white}{\textbf{Descrizione}} &
\textcolor{white}{\textbf{Esito}} \\
		\endhead
TIOF1 & Verificare che vengano renderizzati i nomi dei file selezionati mediante il componente CSVReader. & I \\
TIOF2 & Verificare che venga aggiornato lo stato del componente App e che venga fatto partire l’addestramento dopo aver cliccato il button "Avvia addestramento" per verificare la correttezza del metodo handleTraining() del componente App e l’integrazione con il componente TrainButton. & I \\
TIOF3 & Verificare che venga aperta la finestra per salvare il file json dopo che l’utente ha schiacciato il pulsante "Download" per dimostrare la correttezza del metodo downloadJsonData() del componente App e del metodo downloadJsonFile() del componente DownloadJson. & I \\
TIOF4 & Verificare che venga modificato lo stato del componente App dopo che l’utente ha cliccato il button "Download" nel componente DownloadJson. & I \\
TIOF5 & Verificare che venga modificato lo stato del componente App quando l’utente modifica l’elemento input nel componente DownloadJson. & I \\
TIOF6 & Verificare che la funzione setDataFromFile di App gestisca in modo corretto i file in formato CSV scelti mediante l’elemento input del componente CSVReader. & I \\
TIOF7 & Verificare che la funzione setDataFromFile di App gestisca in modo corretto i file con un formato non accettato scelti mediante l’elemento input del componenente CSVReader. & I \\
TIOF8 & Verificare che la funzione changeAlgorithm di App gestisca in modo corretto la scelta, o cambio, dell'algoritmo di addestramento, mediante l'elemento input ComboBoxAlgorithm. & I \\
TIDF8.1 & Verificare che il pulsante "Seleziona parametri" cambi lo stato del componente App e permetta di selezionare i parametri desiderati da visualizzare nel grafico. & NI \\
TIOF9 & Verificare che la funzione JSONData() della classe SupportRl ritorni le informazioni corrette riguardanti l'algoritmo utilizzato, la data dell'addestramento, la versione del file, l'autore, la lista dei predittori, i risultati voluti e un esempio di retta. & I \\
TIOF10 & Verificare che la funzione JSONData() della classe SupportSvm ritorni le informazioni corrette riguardanti l'algoritmo utilizzato, la data dell'addestramento, la versione del file, l'autore, la lista dei predittori, i risultati voluti e un esempio di retta. & I \\
TIOF11 & Verificare che la classe Influx venga istanziata correttamente tramite costruttore, creando un’istanza InfluxDB collegata ad un database contenuto nella dashboard di Grafana. & NI \\
TIOF12 & Verificare che il metodo writeArrayToInflux e writePointToInflux scrivano correttamente i dati all'interno del database Influx. & NI \\
TIOF13 & Verificare che il metodo \_viewGraph() prenda correttamente i dati dal dal file json attraverso il metodo getJson() e stampi il grafico attraverso il metodo \_setupGraphSeries() dopo aver settato correttamente le predizioni con il metodo updatePredictions(data). & I \\
\end{longtable}

\subsection{Test di unità}
\begin{longtable}{C{2cm} L{8cm} C{2cm}}
\rowcolor{white}\caption{Tabella dei test di unità} \\
		\rowcolor{redafk}
\textcolor{white}{\textbf{Codice}} &
\textcolor{white}{\textbf{Descrizione}} &
\textcolor{white}{\textbf{Esito}} \\
		\endfirsthead
		\rowcolor{white}\caption[]{(continua)} \\
		\rowcolor{redafk}
\textcolor{white}{\textbf{Codice}} &
\textcolor{white}{\textbf{Descrizione}} &
\textcolor{white}{\textbf{Esito}} \\
		\endhead
TUOF1 & Verificare che il componente venga renderizzato correttamente controllando che venga effettuata la renderizzazione degli elementi contenuti al suo interno, per dimostrare la correttezza del metodo Render() di TrainButton. & I \\
TUOF1.1 & Verificare che le funzioni passate come proprietà al componente TrainButton vengano correttamente chiamate dopo un evento onClick. & I \\
TUOF1.2 & Verificare che il componente Button renderizzi un testo differente rispetto a quello normale se viene effettuata una operazione asincrona mediante il Button per dimostrare la correttezza del metodo Render() di TrainButton. & I\\
TUDF1.3 & Verificare che il componente venga renderizzato correttamente controllando che venga effettuata la renderizzazione degli elementi contenuti al suo interno, per dimostrare la correttezza del metodo Render() di ChangeParam. & NI \\
TUDF1.4 & Verificare che le funzioni passate come proprietà al componente ChangeParam vengano correttamente chiamate dopo un evento onClick. & NI \\
TUOF2 & Verificare che il componente venga renderizzato correttamente controllando che venga effettuata la renderizzazione degli elementi contenuti al suo interno, per dimostrare la correttezza del metodo Render() di Chart. & I \\
TUOF3 & Verificare che il componente venga renderizzato correttamente controllando che venga effettuata la renderizzazione degli elementi contenuti al suo interno, per dimostrare la correttezza del metodo Render() di ComboBoxAlgorithm. & I \\
TUOF4 & Verificare che il componente venga renderizzato correttamente controllando che venga effettuata la renderizzazione degli elementi contenuti al suo interno, per dimostrare la correttezza del metodo Render() di DownloadJson. & I \\
TUOF5 & Verificare che il componente venga renderizzato correttamente controllando che venga effettuata la renderizzazione degli elementi contenuti al suo interno, per dimostrare la correttezza del metodo Render() di Header. & I \\
TUOF6 & Verificare che il componente venga renderizzato correttamente controllando che venga effettuata la renderizzazione degli elementi contenuti al suo interno, per dimostrare la correttezza del metodo Render() di InsertCsvButton. & I \\
TUOF7 & Verificare che il componente venga renderizzato correttamente controllando che venga effettuata la renderizzazione degli elementi contenuti al suo interno, per dimostrare la correttezza del metodo Render() di InsertCsvButton. & I \\
TUOF8 & Verificare che se l’utente seleziona l’opzione "Seleziona l'algoritmo" venga aggiunto, nella posizione corretta, un elemento non nullo nello state. & I \\
TUOF9 & Verificare che le funzioni passate come proprietà al componente CSVReader vengano correttamente chiamate dopo un evento onFileLoaded. & I \\
TUOF10 & Verificare che se l’utente seleziona un file valido, l’elemento dell’elemento input del componente CSVReader mostri il nome del file per mostrare la correttezza del metodo render(). & I\\
TUOF11 & Verificare che il componente venga renderizzato correttamente controllando che venga effettuata la renderizzazione degli elementi contenuti al suo interno, per dimostrare la correttezza del metodo Render() di Chart. & I \\
TUOF12 & Verificare che le funzioni passate come proprietà al componente InsertCsvButton vengano correttamente chiamate dopo un evento onFileLoaded. & I \\
TUOF13 & Verificare che la funzione corretta, passata come proprietà, venga chiamata dopo aver cliccato il pulsante "Download". & I \\
TUOF14 & Verificare che le funzioni passate come proprietà al componente ComboBoxAlgorithm vengano correttamente chiamate dopo un evento onChange. & I \\
TUOF15 & Verificare che il componente venga renderizzato correttamente controllando che venga effettuata la renderizzazione degli elementi contenuti al suo interno, per dimostrare la correttezza del metodo Render() di App. & I \\
TUOF16 & Verificare che venga creato l’oggetto concreto CSVReader. & I \\
TUOF17 & Verificare che venga chiamata la funzione setDataFromFile della classe concreta App. & I \\
TUOF18 & Verificare che venga chiamata la funzione changeAlgorithm della classe concreta App. & I \\
TUOF19 & Verificare che venga chiamata la funzione resetAlgorithm della classe concreta App. & I \\
TUOF20 & Verificare che venga chiamata la funzione handleTraining della classe concreta App. & I \\
TUOF21 & Verificare che venga chiamata la funzione downloadJsonData della classe concreta App. & I \\
TUOF22 & Verificare che venga creato l’oggetto concreto RLTrain. & I \\
TUOF23 & Verificare che le funzioni della classe concreta RLTrain vengano chiamate con i parametri corretti. & I \\
TUOF24 & Verificare che venga lanciata un’eccezione TypeError in caso venga chiamato il costruttore se non è stato implementato in una classe che estende la classe Train.& I \\
TUOF25 & Verificare che venga lanciata un’eccezione TypeError in caso venga chiamata la funzione train se non è stata implementata in una classe che estende la classe Train.& I \\
TUOF26 & Verificare che venga lanciata un’eccezione TypeError in caso venga chiamata la funzione getCoefficients se non è stata implementata in una classe che estende la classe Train. & I \\
TUOF27 & Verificare che venga lanciata un’eccezione TypeError in caso venga chiamata la funzione getJSON se non è stata implementata in una classe che estende la classe Train.& I \\
TUOF28 & Verificare che venga lanciata un’eccezione TypeError in caso venga chiamata la funzione getDataChart se non è stata implementata in una classe che estende la classe Train.& I \\
TUOF29 & Verificare che venga creato l’oggetto concreto SVMTrain. & I \\
TUOF30 & Verificare che le funzioni della classe concreta SVMTrain vengano chiamate con i parametri corretti. & I \\
TUOF31 & Verificare che venga creato l’oggetto concreto corretto a seconda dell’algoritmo che si vuole addestrare. & I \\
TUOF32 & Verificare che la funzione getCoefficients() ritorni tutti i parametri utilizzati per l’addestramento. & I \\
TUOF33 & Verificare che venga chiamata la funzione isSVM() della strategia concreta nel metodo setStrategy. & I \\
TUOF34 & Verificare che venga chiamata la funzione isRL() della strategia concreta nel metodo setStrategy. & I \\
TUOF35 & Verificare che venga creato correttamente l'oggetto RLTrain della strategia concereta nel metodo setStrategy(), dopo aver effettuato il controllo col metodo isRL(). & I \\
TUOF36 & Verificare che venga creato correttamente l'oggetto SVMTrain della strategia concereta nel metodo setStrategy(), dopo aver effettuato il controllo col metodo isSVM(). & I \\
TUOF37 & Verificare che il metodo performTraining esegua l'addestramento correttamente. & I \\
TUOF38 & Verificare che il metodo getJsonContent prenda correttamente i dati dal file JSON. & I \\
TUOF39 & Verificare che il metodo getChartData prenda correttamente i dati corretti per mostrarli nel grafico. & I \\
TUOF40 & Verificare che il metodo formatData della classe Chart passi correttamente alle props i dati da visualizzare nel grafico. & I \\
TUOF41 & Verificare che il metodo getColumnsName ottenga correttamente i nomi delle colonne presenti nel file CSV. & I \\
TUOF42 & Verificare che il metodo getDate ottenga correttamente la data odierna, nel formato YYYY-MM-DD. & I \\
TUOF43 & Verificare che il metodo print\_retta stampi correttamente la retta nel file JSON. & I \\
TUOF44 & Verificare che il metodo trainRl calcoli correttamente i coefficienti di regressione attraverso il metodo calculateCoefficients presente nel file \texttt{regression.js}. & I \\
TUOF45 & Verificare che il metodo getCoefficientsRl ottenga correttamente i coefficienti di regressione attraverso il parametro coefficients presente all'interno della classe. & I \\
TUOF46 & Verificare che il metodo trainSvm calcoli correttamente i coefficienti di SVM attraverso i metodi train e getWeights presenti nel file \texttt{svm.js}. & I \\
TUOF47 & Verificare che il metodo Weights ritorni correttamente i weights della SVM. & I \\
TUOF48 & Verificare che il metodo confermaPredizioneSvm confermi correttamente la predizione qualora i weights non fossero nulli. & I \\

%--------------plugin-------------------------------

TUOFyy & Verificare che vengano renderizzate le tabs per dimostrare la correttezza del metodo render() di Editor. & I \\
TUOFyy & Verificare che venga renderizzato il contenuto della tab caricamento json per dimostrare la correttezza del metodo render() di CaricamentoJsonView. & I \\
TUOFyy & Verificare che la funzione setJson(file) del controller sia invocata dopo un evento onChange del componente Files. & I \\
TUOFyy & Verificare che venga mostrato il nome del file e il suo contenuto una volta caricato un json per dimostrare la correttezza del metodo update() di CaricamentoJsonView. & I \\
TUOFyy & Verificare che venga renderizzato il contenuto della tab collegamento per dimostrare la correttezza del metodo render() di CollegamentoView. & I \\
TUOFyy & Verificare che non siano presenti query nell'impostazione del collegamento in caso nessuna query sia stata inserita per dimostrare la correttezza del metodo getPredictors() di CollegamentoView. & I \\
TUOFyy & Verificare che non siano presenti predittori in caso nessun file sia stato inserito per dimostrare la correttezza del metodo getPredictors() di CollegamentoView. & I \\
TUOFyy & Verificare che siano presenti tutti i predittori contenuti nel file json e che siano selezzionabili tutte le query impostate in caso sia stato caricato un file json e sia stata impostata almeno una query per dimostrare la correttezza del metodo getPredictors() di CollegamentoView. & I \\
TUOFyy & Verificare che nessun collegamento sia aggiunto alla lista di collegamenti, se non impostato correttamente, a seguito di un evento onClick del componente Button. & I \\
TUOFyy & Verificare che il collegamento impostato correttamente sia aggiunto alla lista di collegamenti a seguito di un evento onClick del componente Button. & I \\
TUOFyy & Verificare che venga renderizzato il contenuto della tab lista collegamenti per dimostrare la correttezza del metodo render() di ListaCollegamentiView. & I \\
TUOFyy & Verificare che vengano renderizzati tutti i collegamenti effettuati per dimostrare la correttezza del metodo showConnection() di ListaCollegamentiView. & I \\
TUOFyy & Verificare che venga rimosso il collegamento a seguito di un evento onClick del bottone elimina collegamento per dimostrare la correttezza del metodo handleDelete() di ListaCollegamentoView. & I \\
TUOFyy & Verificare che venga renderizzata la form di modifica del collegamento a seguito di un evento onClick del bottone modifica collegamento per dimostrare la correttezza del metodo render() di FormEdit. & I \\
TUOFyy & Verificare che renderizzi il contenuto della form di modifica pre dimostrare la correttezza del metodo getPredictors() di FormEdit. & I \\
TUOFyy & Verificare che venga non venga modificato il collegamento in caso di errori durante la modifica per dimostrare la correttezza del metodo sendConnectionToController() di FormEdit. & I \\
TUOFyy & Verificare che venga aggiornato il collegamento a seguito di una corretta modifica per dimostrare la correttezza del metodo sendConnectionToController() di FormEdit. & I \\
TUOFyy & Verificare che venga renderizzato il contenuto della tab previsione per dimostrare la correttezza del metodo render() di PrevisioneView. & I \\
TUOFyy & Verificare che il bottone di avvio monitoraggio venga cambiato, a seguito di un evento onClick, con il bottone di stop monitoraggio. & I \\
TUOFyy & Verificare che il bottone di stop monitoraggio venga cambiato, a seguito di un evento onClick, con il bottone di avvio monitoraggio. & I \\
TUOFyy & Verificare che venga renderizzato il grafico per dimostrare la correttezza del metodo render() di Panel. & I \\
TUOFyy & Verificare che venga aggiunta una nuova serie visualizzabile nel grafico in caso non fosse già presente per dimostrare la corretezza del metodo setupGraphSeries() di Panel. & I \\
TUOFyy & Verificare che venga aggiornata la serie del grafico e che non venga creata una nuova in caso fosse già presente per dimostrare la correttezza del metodo setupGraphSeries() di Panel. & I \\
TUOFyy & Verificare che non venga sollevata un'eccezione in caso il file json non sia compatibile per dimostrare la correttezza del metodo setJson(file) di Controller. & NI \\
TUOFyy & Verificare che venga letto il contenuto del file json, che siano definiti gli stessi predittori contenuti nel file e che sia stato impostato l'algoritmo corretto per dimostrare la correttezza del metodo setJson(file) di Controller. & I \\
TUOFyy & Verificare che venga aggiunta una nuova connessione alla lista di connessioni per dimostrare la correttezza del metodo addConnection(connection) di Controller. & I \\
TUOFyy & Verificare che venga aggiornata la connessione corretta tra le connessioni presenti per dimostrare la correttezza del metodo updateConnection(id, connection) di Controller. & I \\
TUOFyy & Verificare che venga rimossa la connessione corretta tra le connessioni presenti per dimostrare la correttezza del metodo removeConnection(id) di Controller. & I \\
TUOFyy & Verificare che venga mostrato l'alert di corretto inserimento delle soglie per dimostrare la correttezza del metodo handleSoglie(sMin, sMax) di Controller. & I \\
TUOFyy & Verificare che venga mostrato l'alert di inserimento errato delle soglie per dimostrare la correttezza del metodo handleSoglie(sMin, sMax) di Controller. & I \\
TUOFyy & Verificare che venga restituito l'array di predizioni corretto rispetto al parametro passato alla funzione per dimostrare la correttezza del metodo getPredictedData(connectionName) di Controller. & I \\
TUOFyy & Verificare che venga aggiornata la predizione della connessione se essa era già presente nella lista di previsioni per dimostrare la correttezza del metodo updatePredictions(series) di Controller. & I \\
TUOFyy & Verificare che venga aggiunta una nuova predizione se la connessione non era già presente nella lista di previsioni per dimostrare la correttezza del metodo updatePredictions(series) di Controller. & I \\
TUOFyy & Verificare che non venga eseguito l'aggiornamento delle predizioni se gli input sono stati definiti in modo errato per dimostrare la correttezza del metodo updatePredictions(series) di Controller. & I \\
TUOFyy & Verificare che non venga eseguito l'aggiornamento delle predizioni se non è stato possibile calcolare la predizione per dimostrare la correttezza del metodo updatePredictions(series) di Controller. & I \\
TUOFyy & Verificare che venga calcolata correttamente la predizione per dimostrare la correttezza del metodo predict(inputs) di Regression. & I \\
TUOFyy & Verificare che venga lanciata un'eccezione in caso il numero di input sia maggiore o minore di quanto aspettato per dimostrare la correttezza del metodo predict(inputs) di Regression. & I \\
TUOFyy & Verificare che venga calcolato correttamente il gruppo di appartenenza degli input per dimostrare la correttezza del metodo predict(inputs) di Svm. & I \\
TUOFyy & Verificare che venga lanciata un'eccezione in caso il numero di input sia maggiore o minore di quanto aspettato per dimostrare la correttezza del metodo predict(inputs) di Svm. & I \\

\end{longtable}

\subsection{Riepilogo dei test}

\begin{longtable}{c c c c}
\rowcolor{white}\caption{Tabella di riepilogo dei test} \\
		\rowcolor{redafk}
\textcolor{white}{\textbf{Tipologia}} &
\textcolor{white}{\textbf{Implementati}} &
\textcolor{white}{\textbf{Non implementati}} &
\textcolor{white}{\textbf{Totale}} \\
		\endfirsthead
		\rowcolor{white}\caption[]{(continua)} \\
		\rowcolor{redafk}
\textcolor{white}{\textbf{Tipologia}} &
\textcolor{white}{\textbf{Implementati}} &
\textcolor{white}{\textbf{Non implementati}} &
\textcolor{white}{\textbf{Totale}}\\
		\endhead
Accettazione & 19 & 3 & 22\\
Sistema &  &  & \\
Integrazione & 11 & 3 & 14\\
Unità &  &  & \\		
\textbf{Totale} & \textbf{30} & \textbf{6} & \textbf{36} \\
\end{longtable}