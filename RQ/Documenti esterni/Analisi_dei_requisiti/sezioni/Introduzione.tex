\section{Introduzione}
	\subsection{Scopo del documento}
		Il presente documento ha lo scopo di descrivere in modo dettagliato i requisiti stabiliti per il prodotto. Tali requisiti sono stati individuati a seguito dell'analisi del capitolato\glo C4, ed i successivi incontri con la proponente\glo \emph{Zucchetti SPA.}

	
\subsection{Scopo del prodotto}
   Lo scopo del progetto è la realizzazione di un plug-in\glo per Grafana\glo per la previsione sul flusso dati\glo. Tale plug-in ha la capacità di utilizzare una Support Vector Machine\glo o Regressione Lineare\glo addestrate dall'utente la cui definizione è contenuta in un file in formato JSON\glo. Il sistema ,per aiutare l’utente nel monitoraggio, sarà in grado di lanciare allarmi.
Nello specifico il plug-in deve poter monitorare i dati in ingresso da un certo flusso, come per esempio percentuali di utilizzo della memoria o temperatura del processore, i quali verranno successivamente mostrati attraverso l'interfaccia grafica\glo di Grafana.
La necessità di lanciare un allarme verrà valutata dalla SVM\glo o RL\glo in base ai predittori su cui è stata tarata dopo la fase di addestramento.
Il plug-in rimane in esecuzione su Grafana e riceve continuamente informazioni in ingresso da un flusso di dati: viene quindi continuamente ricalcolata la necessità di lanciare un alert\glo.
La SVM, sopra menzionata, potrà essere sviluppata attraverso la libreria\glo SVMJS\glo (\url{ https://github.com/karpathy/svmjs}), mentre la RL utilizzerà una libreria fornita dal proponente\glo e sarà disponibile un’ulteriore guida al link \url{https://github.com/Tom-Alexander/regression-js}.

	
	\subsection{Glossario}
		Affinché sia possibile evitare ambiguità relative al linguaggio utilizzato nei documenti formali viene fornito il \emph{glossario\_v.3.0.0}. In tale documento vengono definiti e descritti tutti i termini con un significato particolare. Per facilitare la lettura i termini sono contraddistinti mediante una lettera \emph{"G"} a pedice.
		
\subsection{Riferimenti}
		\subsubsection{Normativi}
			\begin{itemize}
				\item \textbf{Norme di progetto}: \emph{norme\_di\_progetto\_v3.0.0};
				\item \textbf{Capitolato d'appalto \emph{C4 - Predire in Grafana}}: \url{https://www.math.unipd.it/~tullio/IS-1/2019/Progetto/C4.pdf} ;
				\item \textbf{Approfondimento capitolato d'appalto \emph{C4 - Predire in Grafana}}: \url{https://www.math.unipd.it/~tullio/IS-1/2019/Dispense/C4a.pdf} ;
				\item \textbf{Verbale esterno}: 2020-03-31;
				\item \textbf{Verbale esterno}: 2020-06-05.
			\end{itemize}
		
		\subsubsection{Informativi}
			\begin{itemize}
			\item \textbf{Grafana}: \url{https://grafana.com/docs/grafana/latest/} ;
			\item \textbf{Grafana plugins}: \url{https://grafana.com/docs/grafana/latest/plugins/developing/development/} ;
			\item \textbf{InfluxDB}: \url{https://docs.influxdata.com/influxdb/v1.8/} ;

			\item \textbf{Materiale didattico del corso di Ingegneria del Software:}
			\begin{itemize}
				\item \textbf{Software Engineering - Ian Sommerville - 10th Edition 2014}: \begin{itemize}
				\item \S 7 Design and implementation.
				\end{itemize}
				\item \textbf{Analisi dei requisiti}: \url{https://www.math.unipd.it/~tullio/IS-1/2019/Dispense/L08.pdf} ;				\item \textbf{Diagrammi dei casi d'uso}: \url{https://www.math.unipd.it/~tullio/IS-1/2019/Dispense/E03.pdf}.
					
				\end{itemize}
			\end{itemize}				