\section{Glossary}

{\Large\textbf{G}\par}
\textbf{Grafana} \\
Piattaforma open-source che consente di monitorare i dati provenienti da diverse sorgenti, attraverso una loro rappresentazione grafica all'interno di una
dashboard.

{\Large\textbf{M}\par}
\textbf{Machine Learning} \\
Sinonimo di \hyperref[par:appr_auto]{apprendimento automatico}. L’apprendimento automatico è lo studio di algoritmi che si migliorano automaticamente attraverso l’esperienza. Gli algoritmi di apprendimento automatico costruiscono modelli matematici basati su dati campione noti come “dati di addestramento”.

{\Large\textbf{P}\par}
\textbf{Predittore} \\
E' una statistica, cioè una funzione dei dati, definita allo scopo di effettuare previsioni su una o più variabili.

{\Large\textbf{R}\par}
\textbf{RL} \\
Rappresenta un metodo di stima del valore atteso condizionato di una variabile dipendente, dati i valori di altre variabili indipendenti. Si definisce retta di regressione la retta: $y = \alpha x + \beta $.

{\Large\textbf{S}\par}
\textbf{SVM} \\
Inventata da V. Vapnik nel 1990, è un modello di apprendimento automatico supervisionato che utilizza algoritmi di classificazione per valutare specifici problemi. Questi algoritmi trovano impiego nell'ambito del machine learning.