\section{The Plug-in}
Here a step by step explanation will guide the user to obtaning a prediction.

\subsection{Loading the Plug-in}
\begin{enumerate}
	\item The user will have to select the plus icon from the sidebar, from which a drop-down menu containing four options will appear, from this menu the “dashboard” option has to be selected.
	\item The user will now have to select the “Chose Visualization” button.
	\item Finally, by pressing on the “Predire in Grafana” button, the user can use the plug-in.
\end{enumerate}
	
\subsection{Loading a JSON File}
The user can select the “Insert File” button contained in the “Inserimento File JSON” section.
This will open a window from which the JSON file can be selected.
The content of the JSON file will be displayed in a section to the right of the previously mentioned section.


\subsection{Connecting the Nodes}
\begin{enumerate}
	\item The user can choose from the “lista predittori” section which queries are to be associated with which nodes, by opening the drop down menu under “select predictors”.
	\item The user can now select which part of the data stream to couple with a predictor in the “selezione del flusso dati” section, by opening the drop down menu to the right of “select query”.
	\item Maximum and minimum thresholds can also be set in the “impostazione soglie” section, by inserting numbers in the dedicated boxes.
	\item Finally, the user can confirm the various operations in the “Conferma Collegamento” section by pressing “conferma collegamento”. In this section the user can also view all the predictor-data stream connections that have been made.
\end{enumerate}

\subsection{Modifying the connections}
	\begin{enumerate}
		\item The user can select the connection he has intention of modifying in the “lista collegamenti” section by pressing on the “selezionare collegamento” button and choosing a connection from the ones displayed.
		\item In the “opzioni collegamenti” section the user can modify, disconnect or create new connections between nodes by selecting the appropriate buttons.
	\end{enumerate}
\subsection{Prediction Operations}
