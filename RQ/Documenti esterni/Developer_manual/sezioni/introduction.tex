\section{Introduction}
	\subsection{Document's purpose}
	Il manuale dello sviluppatore permette ad ogni sviluppatore che si approcci al prodotto software \emph{"Predire in Grafana"} di assimilare le informazioni principali per poter manutenere e estendere tale prodotto. All'interno del documento verrà descritto il prodotto in modo dettagliato, consentendo allo sviluppatore di ottenere una spiegazione esaustiva necessaria per l'attività che andrà a svolgere.

The developer's manual allows each developer reader to absorb \emph{"Predire in Grafana"}'s product key information to maintain and extend the product itself.
This document describes the product in its totality, giving the developer an exaustive explanation required for his tasks.
	
\subsection{Predire in Grafana’s Purpose}

\emph{"Predire in Grafana"} è un plugin realizzato per la piattaforma open source\glo Grafana che permette di calcolare delle previsioni su un flusso dati. Viene utilizzato un algoritmo addestrato dall'utente, la cui definizione è contenuta in un file in formato JSON\glo, per poter effettuare le previsioni.  Viene fornito un applicativo esterno per l'addestramento degli algoritmi di previsione, attualmente sono implementati gli algoritmi di Support Vector Machine\glo e Regressione Lineare\glo . Nello specifico il plugin monitora i dati in ingresso da un certo flusso, come
per esempio percentuali di utilizzo della memoria o temperatura del processore, e produce delle previsioni che vengono successivamente mostrate attraverso l'interfaccia graficaG di Grafana. Il plugin rimane in esecuzione e riceve continuamente informazioni in ingresso da un flusso di dati. In questo modo gli operatori potranno monitorare eventuali cambiamenti sul flusso dati grazie alla previsione in real time\glo ed intervenire, se necessario, sull'origine del problema. 
 
\emph{"Predire in Grafana"} is a plugin made for Grafana which is an open source\glo platform commonly used to analyze data series. The plugin allows users to predict datas on a stream data. \emph{"Predire in Grafana"} plugin uses a JSON file which contains a trained algorithm definition to get predictions. Users can use an external training tool, which use Machine Learning\glo , to get these JSON' files. At the moment only Support Vector Machine and Linear Regression algorithm are implemented. In more detail input datas, like cpu's usage and cpu's temperature, are constantly monitored to get predictions on the aspect you want to examine. Predictions are shown throght Grafana GUI and continue to be updated after being calculated from datas coming from a database. Thanks to this operators can monitor each process and intervene at the root of the problem whenever neccessary.


\subsection{Glossary}
A fine documento viene riportato nell'appendice un glossario che racchiude al suo interno ciascun termine che necessita di una spiegazione dettagliata per risultare più comprensibile al lettore. Tali termini vengono contrassegnati nel documento con una G a pedice.
 
At the end of the document in the appendix is available a glossary where explanations for new or ambiguous terms can be found. These are marked with a subscript G.

\subsection{References}
\subsubsection{Installation}
\begin{itemize}
	\item \url{https://nodejs.org/en/} (\url{https://nodejs.org/it/});
	\item \url{https://git-scm.com/};
	\item \url{https://www.gitkraken.com/};
	\item \url{https://grafana.com/get};
	\item \url{https://www.jetbrains.com/idea/}.
\end{itemize}
\subsubsection{Technologies}
\begin{itemize}
	\item \url{https://reactjs.org/docs/getting-started.html};
	\item \url{https://grafana.com/docs/grafana/latest/developers/plugins/};
	\item \url{https://www.jetbrains.com/help/idea/eslint.html}.
\end{itemize}
\subsubsection{Legal}
\begin{itemize}
	\item \url{https://www.apache.org/licenses/LICENSE-2.0}.
\end{itemize}

\subsubsection{Informative}
\begin{itemize}
	\item \url{https://en.wikipedia.org/wiki/Linear_regression};
	\item \url{https://en.wikipedia.org/wiki/Support_vector_machine}.
\end{itemize}
	

