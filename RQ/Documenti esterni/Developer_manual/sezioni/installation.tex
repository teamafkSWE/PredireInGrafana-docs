\section{Installation}

\subsection{Instruments installation}

\subsubsection{Node.js installation}
The installation of the runtime JavaScript Node.js can be done by visiting  Node.js page at \url{https://nodejs.org/en/download/}. In this site the developer can download the most suitable version of Node.js for his operative system. For Linux user is also possible to use the package manager provided by the OS\glo and exclusively for \textbf{Ubuntu/Debian} developers can run in terinal this code:\\
\texttt{apt-get install nodejs} \\

\subsubsection{Git installation}
For the installation of the version control system, the developer needs to reach Git site's at \url{https://git-scm.com/downloads}. Inside the 'Download' section are available the links to download the compatible version with his operative system. Also Linux base system  can install Git from their package manager or running from \textbf{Ubuntu/Debian} terminal this line:\\
\texttt{apt-get install git} \\

\subsection{Grafana installation}
Developers can install Grafana by reaching its download page at \url{https://grafana.com/grafana/download}. There they can download compatible version for MacOs, Windows and Linux base system. Furthermore, the most common Unix base systems can install Grafana running terminal lines showed in the same page.

\subsubsection{WEB Grafana execution}

Depending from which OS the developer is working on,the execution of WEB Grafana can be done by:
\begin{itemize}
\item \textbf{Windows}: opening "bin" folder in Grafana installation folder
(it should be xxxxxxxxx), and doubl-clicking the "grafana-server" file;
\item \textbf{Linux and MacOS}: running on terminal the following line:\\
\texttt{systemclt start grafana-server}.
\end{itemize} 
After that, the developer needs to reach the address \url{http://localhost:3000/} through a browser. For the first access, the developer needs to fill the fields username with "admin" and password with "admin", and once he/she is in, he/she will need to register himself/herself into the system for next accesses.

\begin{figure}[H]
\centering
\includegraphics[scale=0.25]{./img/web_grafana_login.png}
\caption{WEB Grafana page at \url{http://localhost:3000}}
\end{figure}
\subsection{Plugin and Tool installation}
The following sections will guide developers to install correctely both Training Tool and Prediction Tool.
\subsubsection{GitHub repository clone}
The developer to clone the GitHub repository needs to open the terminal and choose a folder inside the filesystem with command: \\
\texttt{cd /path/folder}
After in the same location, the developer needs to run on terminal this line:\\
\texttt{git clone https://github.com/teamafkSWE/PredireInGrafana-SW.git}
This code creates the folder that contains the source code of Training Tool and Prediction Plugin.
\paragraph*{Dependency}
The following table contains all the dependency adopted for making the tool and the plugin.
%table

\paragraph*{Developer dependency}






