\section{Installation}

\subsection{Instruments installation}

\subsubsection{Node.js installation}
The installation of the runtime JavaScript Node.js can be done by visiting  Node.js page at \url{https://nodejs.org/en/download/}. In this site the developer can download the most suitable version of Node.js for his operative system. For Linux user is also possible to use the package manager provided by the OS\glo and exclusively for Ubuntu/Debian developers can run in terinal this code:\\
\texttt{apt-get install nodejs} \\

\subsubsection{Git installation}
For the installation of the version control system, the developer needs to reach Git site's at \url{https://git-scm.com/downloads}. Inside the 'Download' section are available the links to download the compatible version with his operative system. Also Linux base system  can install Git from their package manager or running from Ubuntu/Debian terminal this line:\\
\texttt{apt-get install git} \\

\subsection{Grafana installation}
Developers can install Grafana by reaching its download page at \url{https://grafana.com/grafana/download}. There they can download compatible version for MacOs, Windows and Linux base system. Furthermore, the most common Unix base systems can install Grafana running terminal lines showed in the same page.

\subsubsection{WEB Grafana execution}

Depending from which OS the developer is working on,the execution of WEB Grafana can be done by:
\begin{itemize}
\item \textbf{Windows}: opening "bin" folder in Grafana installation folder
(the path should be like \texttt{C:/Program Files/grafana-6.7.3/bin/}), and double-clicking on "grafana-server.exe";
\item \textbf{Linux and MacOS}: running on terminal the following line:\\
\texttt{systemclt start grafana-server}\\
\end{itemize} 
After that, the developer needs to reach the address \url{http://localhost:3000/} through a browser. For the first access, the developer needs to fill the fields username with "admin" and password with "admin", and once he/she is in, he/she will need to register himself/herself into the system for next accesses.

\begin{figure}[H]
\centering
\includegraphics[scale=0.25]{./img/web_grafana_login.png}
\caption{WEB Grafana page at \url{http://localhost:3000}}
\end{figure}
\subsection{Plugin and Tool installation}
The following sections will guide developers to install correctely both Training Tool and Prediction Tool.
\subsubsection{GitHub repository clone}
The developer to clone the GitHub repository needs to open the terminal and choose a folder inside the filesystem with command: \\
\texttt{cd /path/to/folder}\\
After that, in the same location, the developer needs to run  this line:\\
\texttt{git clone https://github.com/teamafkSWE/PredireInGrafana-SW.git}\\
This line creates the folder that contains the source code of Training Tool and Prediction Plugin.\mbox \\

\subsection{Dependencies}
Dependencies are a list of packages needed to develop and run the project. For this reason, other developers will need to install all them to run correctely the tool and plugin. This operation can be done by moving to tool or plugin's folder and there run this line:\\
\texttt{npm install}\\
The following tables contain all the dependencies adopted to make both tool and plugin.

\begin{longtable}{C{5cm} C{4cm}}
\rowcolor{white}
\caption{Table of Training Tool dependency}\\
\rowcolor{redafk}
	\textcolor{white}{\textbf{Packacge}} &
	\textcolor{white}{\textbf{Version}} \\
		\endfirsthead
		\rowcolor{white}\caption[]{(continua)} \\
\rowcolor{redafk}
	\textcolor{white}{\textbf{Package}} &
	\textcolor{white}{\textbf{Version}} \\
		\endhead
@testing-library/react & 9.5.0 \\
@testing-library/user-event & 7.2.1 \\
bootstrap & 4.4.1\\
chart.js & 2.9.3\\
is-promise & 2.2.2\\
libsvm-js & 0.2.1\\
ml-svm & 2.1.2\\
react & 16.13.1\\
react-bootstrap & 1.0.1\\
react-chartjs-2 & 2.9.0\\
react-csv-reader & 3.5.0\\
react-dom & 16.13.1\\
react-script & 3.4.1\\
svm & 0.1.1\\
\end{longtable}

\begin{longtable}{C{5cm} C{4cm}}
\rowcolor{white}
\caption{Table of Prediction Plugin dependency}\\
\rowcolor{redafk}
	\textcolor{white}{\textbf{Package}} &
	\textcolor{white}{\textbf{Version}} \\
		\endfirsthead
		\rowcolor{white}\caption[]{(continua)} \\
\rowcolor{redafk}
	\textcolor{white}{\textbf{Package}} &
	\textcolor{white}{\textbf{Version}} \\
		\endhead
@influxdata/influxdb-client & 1.3.0\\
axios & 0.19.2\\
react-datepicker & 2.16.0\\
react-files & 2.4.8\\
\end{longtable}

\paragraph{Developer dependency}\mbox{} \\ \mbox{} \\
Developer will have to install dependencies to run correctely the project.
All them are reported down below.

\begin{longtable}{C{5cm} C{4cm}}
\rowcolor{white}
\caption{Table of Prediction Tool developer dependency}\\
\rowcolor{redafk}
	\textcolor{white}{\textbf{Package}} &
	\textcolor{white}{\textbf{Version}} \\
		\endfirsthead
		\rowcolor{white}\caption[]{(continua)} \\
\rowcolor{redafk}
	\textcolor{white}{\textbf{Package}} &
	\textcolor{white}{\textbf{Version}} \\
		\endhead
@testing-library/jest-dom & 4.2.4\\
csv-parser & 2.3.2\\
eslint-plugin-react-hooks & 4.0.4\\
\end{longtable}

\begin{longtable}{C{5cm} C{4cm}}
\rowcolor{white}
\caption{Table of Prediction Plugin developer dependency}\\
\rowcolor{redafk}
	\textcolor{white}{\textbf{Package}} &
	\textcolor{white}{\textbf{Version}} \\
		\endfirsthead
		\rowcolor{white}\caption[]{(continua)} \\
\rowcolor{redafk}
	\textcolor{white}{\textbf{Package}} &
	\textcolor{white}{\textbf{Version}} \\
		\endhead
@grafana/data & next\\
@grafana/toolkit & 6.7.2\\
@grafana/ui & next\\
webpack & 4.43.0
\end{longtable}

\subsection{IDE Intellij IDEA installation}
To install the integrated developing environment (IDE), it's necessary to reach the page \url{https://www.jetbrains.com/idea/download/}. This will address directely inside the section where Intellij IDEA can be downloaded for Windows, MacOS and Linux version. Ubuntu users can run one of the following lines to install the snap package:\\
\begin{itemize}
\item \textbf{Community version}:\\
\texttt{sudo snap install intellij-idea-community --classic} 
\item \textbf{Ultimate version}:\\
\texttt{sudo snap install intellij-idea-ultimate --classic}
\item \textbf{Educatoinal version}:\\
\texttt{sudo snap install intellij-idea-educational --classic}
\end{itemize}




