\section{Test}
A unity test is the operation that aims to check the correct execution of functions of a component.
The run tests in this projetct, the developer needs to open the terminal and to move inside the folder of the product that needs to be tested (in this case \texttt{PredireInGrafanaSW/my-plugin} or \texttt{PredireInGrafanaSW/prediction-tool}). Inside the folder, the developer has to run the command \texttt{npm run test} inside the terminal. 
\subsection{Edit tests}
Inside the project every components need to be tested. For this reason the developer has to name a test with \textbf{<component name>.test.ts}. Doing so a single component can be checked by a single file.
\subsection{Code Coverage}
Code coverage is the percentuage of the project's code that is analised by tests. This operation is made by Coverall which is an external tool linked to the code through the project's repository. So if the developer wants to chek coverage's value, he has to reach the READ.me file inside the GitHub repository at \url{https://github.com/teamafkSWE/PredireInGrafana-SW}. Inside that file there are two badges and the one with the label "coverage" shows the value of code coverage until the last build.