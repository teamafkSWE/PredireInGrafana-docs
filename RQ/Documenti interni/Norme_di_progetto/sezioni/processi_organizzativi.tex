\section{Processi organizzativi}

\subsection{Gestione organizzativa del progetto}

\subsubsection{Scopo}
Lo scopo del processo di gestione organizzativa è definito dai seguenti punti:
\begin{itemize}
\item creazione di un modello organizzativo che specifica i rischi che possono
verificarsi;
\item definizione un modello di sviluppo da seguire;
\item pianificazione del lavoro in base alle scadenze fissate;
\item creazione e calcolo di un piano economico suddiviso tra i ruoli;
\item definizione finale del bilancio sul totale delle spese.
\end{itemize}
Queste attività sono definite a cura del \textit{Responsabile di Progetto} e redatte all’interno del documento \textit{piano\_di\_progetto\_v3.0.0}.

\subsubsection{Aspettative}
Gli obiettivi del processo di gestione organizzativa del progetto sono: \begin{itemize}
\item coordinare i componenti del gruppo attraverso l’assegnazione di ruoli e
compiti;
\item coordinare la comunicazione tra i membri del gruppo con l’ausilio di strumenti che la rendano facile ed efficace;
\item pianificare l’esecuzione delle attività in modo ragionevole;
\item gestire le attività anche dal punto di vista economico;
\item monitorare costantemente il team, i processi e i prodotti in modo efficace.
\end{itemize}

\subsubsection{Descrizione}
Il processo di gestione organizzativa del progetto è composto dalle seguenti attività: \begin{itemize}
\item definizione dello scopo;
\item stima e pianificazione di risorse, costi e tempo per lo svolgimento del progetto;
\item assegnazione dei ruoli e, per ognuno di essi, delle attività;
\item gestione del controllo e dell’esecuzione di tutte le attività;
\item valutazione periodica dello stato e dell’esecuzione delle attività rispetto a quanto pianificato.
\end{itemize}

\subsubsection{Attività}
\paragraph{Pianificazione delle risorse}\mbox{} \\ \mbox{} \\
Per gestire le risorse disponibili per il nostro progetto dobbiamo monitorare il rispetto dei costi e dei tempi preventivati. A tal proposito definiamo i ruoli di progetto e le relative funzioni.

\paragraph{Ruoli di progetto}\mbox{} \\ \mbox{} \\
Ciascun membro del gruppo, a rotazione, deve ricoprire il ruolo che gli viene assegnato e che corrisponde all'omonima figura aziendale. Nel \textit{Piano di Progetto} vengono organizzate e pianificate le attività assegnate agli specifici ruoli. I ruoli che ogni componente del gruppo è tenuto a rappresentare sono descritti di seguito.

\subparagraph*{Assegnazione}
I ruoli scelti corrispondono alla rispettiva figura aziendale e valgono per la
durata di una milestone (consegna). Al termine di ognuna di esse, viene eseguita una rotazione dei ruoli così da permettere a ciascuno dei membri del gruppo di
ricoprirli tutti almeno una volta.

\subparagraph*{Responsabile di progetto}\mbox{} \\ \mbox{} \\
Il \textit{Responsabile di Progetto} (o semplicemente \textit{Responsabile}) è una figura chiave in quanto ricadono su di lui le responsabilità di pianificazione, gestione, controllo e coordinamento delle risorse e attività del gruppo. Il \textit{Responsabile} si occupa anche di interfacciare il gruppo con le persone esterne facendo da intermediario: sono quindi di sua competenza le comunicazioni con committente e proponente.
Questa figura è incaricata anche di analizzare e gestire le criticità, che si incontrano durante il progetto, e di approvare i documenti.

\subparagraph*{Amministratore}\mbox{} \\ \mbox{} \\
L'\textit{Amministratore} ha il compito di supporto e controllo dell'ambiente di lavoro.
Egli deve quindi:
\begin{itemize}
	\item dirigere le infrastrutture di supporto;
	\item risolvere problemi legati alla gestione dei processi;
	\item gestire la documentazione;
	\item controllare versioni e configurazioni.
\end{itemize}

\subparagraph*{Analista}\mbox{} \\ \mbox{} \\
L'\textit{Analista} si occupa di analisi dei problemi e del dominio applicativo. Questa figura ha anche il compito di redigere i documenti, in questo caso può essere definito come \textit{Redattore}.
Le sue responsabilità sono:
\begin{itemize}
	\item studio del dominio del problema;
	\item definizione della complessità e dei requisiti dello stesso;
	\item redazione del documenti:\textit{ Analisi dei Requisiti} e \textit{Studio di Fattibilità}.
\end{itemize}

\subparagraph*{Progettista}\mbox{} \\ \mbox{} \\
Il \textit{Progettista} gestisce gli aspetti tecnologici e tecnici del progetto.
Il \textit{Progettista} deve:
\begin{itemize}
	\item effettuare scelte efficienti ed ottimizzate su aspetti tecnici del progetto;
	\item sviluppare un'architettura che sfrutti tecnologie note ed ottimizzate, su cui basare un prodotto stabile e manutenibile.
\end{itemize}

\subparagraph*{Programmatore}\mbox{} \\ \mbox{} \\
Il \textit{Programmatore} è responsabile della codifica del progetto e delle componenti di supporto che serviranno per effettuare le prove di verifica e validazione sul prodotto.
Il \textit{Programmatore} si occupa di:
\begin{itemize}
	\item implementare le decisioni del progettista;
	\item creare o gestire componenti di supporto per la verifica e validazione del codice.
\end{itemize}

\subparagraph*{Verificatore}\mbox{} \\ \mbox{} \\
Il \textit{Verificatore} si occupa di controllare il prodotto del lavoro svolto dagli altri membri del team, sia esso codice o documentazione. Per le correzioni si affida agli standard definiti nelle \textit{Norme di Progetto}, nonché alla propria esperienza e capacità di giudizio.
Il \textit{Verificatore} deve:
\begin{itemize}
	\item ispezionare i prodotti in fase di revisione, avvalendosi delle tecniche e degli strumenti definiti nelle \textit{Norme di Progetto};
	\item evidenziare difetti ed errori del prodotto in esame;
	\item segnalare eventuali errori trovati all'autore dell'oggetto preso in esame o alla persona che ha responsabilità su di esso.
\end{itemize}

\paragraph{Gestione dei rischi}\mbox{} \\ \mbox{} \\
È compito del \textit{Responsabile} rilevare i rischi e renderli noti, tramite un continuo monitoraggio e una continua identificazione di quest'ultimi. \\
Qualora dovesse essere \textbf{identificato} un nuovo rischio è necessario procedere con i seguenti passaggi:
\begin{enumerate}
	\item \textbf{Classificare} il rischio seguendo la codifica;
	\item \textbf{Descrivere} una strategia da applicare per gestire il rischio;
	\item \textbf{Riportare} il rischio nel \textit{Piano di Progetto}.
\end{enumerate}

\subparagraph{Codifica}\mbox{} \\ \mbox{} \\
La codifica dei rischi è utilizzata per la classificazione. Il codice di un rischio si presenta nella forma: \\
\centerline{\textbf{Ri[Categoria][Numero]}}
dove:
\begin{itemize}
	\item \textbf{Categoria}: indica la categoria del rischio, essa può assumere i valori:
	\begin{itemize}
		\item \textbf{O} per i rischi organizzativi;
		\item \textbf{T} per i rischi tecnologici;
		\item \textbf{P} per i rischi interpersonali.
	\end{itemize}
	\item \textbf{Numero}: insieme a categoria identifica in maniera univoca il rischio, può assumere un valore intero progressivo a due cifre (01-99).
\end{itemize}

\paragraph{Gestione delle comunicazioni}
\subparagraph{Comunicazioni interne}\mbox{} \\ \mbox{} \\
Le comunicazioni interne ai membri del gruppo vengono gestite  tramite 2 applicazioni:
\begin{itemize}
	\item Telegram\glo;
	\item Discord\glo.
\end{itemize}
È stato disposto un gruppo Telegram sul quale si discute di tematiche generali o collettive, garantendo risposte rapide e ordinate in caso di decisioni per votazione, grazie ad apposite funzioni dette bot di Telegram\glo. \\
Discord viene usato principalmente per le riunioni tra i membri del gruppo, ma l'applicazione mette anche a disposizione dei canali testuali. Tali canali sono stati suddivisi per tema e vengono usati per le comunicazioni specifiche per agevolare la stesura dei documenti. Vengono suddivisi in:
\begin{itemize}
	\item \textbf{General}: per discussioni riguardanti rotazioni dei ruoli e decisione degli argomenti da discutere nelle riunioni;
	\item \textbf{Links}: per tenere traccia di tutti i link utili al progetto;
	\item \textbf{Analisi-requisiti}: per discutere gli Use Case\glo e i requisiti necessari alla stesura dell'\textit{Analisi dei Requisiti};
	\item \textbf{Norme}: per discutere riguardo le regole del \textit{Way of Working} del gruppo, le norme da seguire e, di conseguenza, la stesura del documento \textit{Norme di Progetto}\glo;
	\item \textbf{Piano-progetto}: per confrontarsi riguardo il monte ore dei vari ruoli e per facilitare la stesura del documento \textit{Piano di Progetto};
	\item \textbf{Piano-qualifica}: per discutere di strategie da attuare per garantire qualità attraverso verifica\glo e validazione\glo.
\end{itemize}
\subparagraph{Comunicazioni esterne}\mbox{} \\ \mbox{} \\
Le comunicazioni con soggetti esterni al gruppo sono di competenza del responsabile. Gli strumenti predefiniti sono la posta elettronica, dove viene utilizzato l'indirizzo \href{mailto:gruppoafk15@gmail.com}{gruppoafk15@gmail.com}.
Per comunicare con \textit{Zucchetti SPA} viene usato il servizio Skype\glo per le chiamate. Il responsabile ha il compito di tenere informati gli altri componenti del gruppo in caso di assenza.

\subparagraph{Gestione riunioni}\mbox{} \\ \mbox{} \\
Le riunioni possono essere interne o esterne. All'inizio di ogni riunione il \textit{Responsabile} nomina, tra i componenti del gruppo, un \textit{Segretario} che si occuperà di far rispettare l'ordine del giorno. Inoltre ha l'onere della stesura del \textit{Verbale di Riunione}\glo.

\subparagraph*{Riunioni interne} \mbox{} \\ \mbox{} \\
È compito del \textit{Responsabile} organizzare riunioni interne al gruppo. Ciò prevede, più nello specifico, la stesura dell'ordine del giorno e
stabilire data, orario e luogo di incontro, mediando se necessario con i membri per permettere la presenza di tutti. Le riunioni sono tenute principalmente usando Discord, così da essere il più facilmente raggiungibili.
Il \textit{Responsabile} deve inoltre assicurarsi, attraverso la comunicazione
mediante i mezzi propri del gruppo, che ogni componente sia pienamente a conoscenza della riunione in tutti i suoi dettagli. \\
D'altro canto ogni membro del gruppo deve presentarsi puntuale agli appuntamenti,
e comunicare in anticipo eventuali ritardi o assenze adeguatamente giustificate.

\subparagraph*{Riunioni esterne} \mbox{} \\ \mbox{} \\
È nuovamente compito del \textit{Responsabile} organizzare riunioni esterne.
Nello specifico egli deve preoccuparsi di contattare l'azienda proponente per fissare gli
incontri e qualora sia necessario, tenendo conto anche delle preferenze di date e orario
espresse dagli altri membri del gruppo. La partecipazione a tali riunioni deve essere,
a meno di casi eccezionali, unanime.
Ogni membro del gruppo può, inoltre, esprimere al Responsabile una richiesta, adeguatamente motivata, di fissare una riunione esterna. A questo punto sarà compito
dello stesso Responsabile giudicare come valida o meno la richiesta presentatagli ed
agire di conseguenza.

\subparagraph*{Verbale di riunione} \mbox{} \\ \mbox{} \\
Ad ogni riunione, interna o esterna, è compito del \textit{Segretario} designato redigere il \textit{Verbale di riunione} corrispondente, che deve essere poi approvato dal \textit{Responsabile}. La struttura del \textit{Verbale} è definita in \hyperref[par:verbali]{§ 3.1.4.3.5}.
\subsubsection{Metriche di qualità}
L’obiettivo è pianificare le risorse per garantire un corretto avanzamento del
progetto, monitorando e rispettando costi e tempistiche.
\begin{longtable}{ C{4.5cm} c L{9.5cm} }
	\rowcolor{white}\caption{Metriche di qualità per la pianificazione efficiente delle risorse}\\
		\rowcolor{redafk}
		\textcolor{white}{\textbf{Nome}} & \textcolor{white}{\textbf{Codice}} & \centerline{\textcolor{white}{\textbf{Descrizione}}} \\
		\endfirsthead
		\rowcolor{white}\caption[]{(continua)} \\
		\rowcolor{redafk}
		\textcolor{white}{\textbf{Nome}} & \textcolor{white}{\textbf{Codice}} & \centerline{\textcolor{white}{\textbf{Descrizione}}} \\
		\endhead
		Schedule Variance (SV)  & MP01 & La Schedule Variance indica se una certa attività o processo è in anticipo, in pari, o in ritardo rispetto alla data di scadenza prevista. È calcolata utilizzando la seguente formula: \newline
\[ SV = DCE-DCP\]
dove: \begin{itemize}
\item \textbf{DCE}: data conclusione effettiva;
\item \textbf{DCP}: data conclusione pianificata.
\end{itemize}
Se $SV \leq 0$ significa che l'attività o il processo è in pari o in anticipo, invece, se $SV > 0$ significa che l'attività è in ritardo. \\	
		Budget Variance (BV) & MP02 & Permette di controllare i costi sostenuti alla data corrente rispetto al budget preventivato. Viene calcolata in fase di consuntivo di periodo utilizzando la seguente formula: \newline
		\[ BV[\%] = \frac{CP-CE}{CE}\cdot 100 \]		
dove: \begin{itemize}
\item \textbf{CP}: costo preventivato;
\item \textbf{CE}: costo effettivo.
\end{itemize}
Se $BV[\%] \geq 0$ indica che il budget sta venendo speso più lentamente di quanto pianificato, se negativo invece indica che il budget sta venendo speso più velocemente di quanto pianificato. \\
Produttività (P) & MP03 & Rappresenta la produttività media delle risorse impiegate, cioè delle persone coinvolte, nelle diverse fasi del progetto. È misurata in termini di numero di linee di codice (\hyperref[par:MS01]{LOC}) sviluppate da una persona nell’unità di tempo stabilita (settimana). È utilizzata per valutare lo sforzo richiesto per lo sviluppo  del progetto a fronte delle sue dimensioni. 
\[ Pmedia = \frac{LOC}{settimana}\]
	\end{longtable}

\subsubsection{Strumenti di supporto}
Il gruppo, nel corso del progetto, ha utilizzato o utilizzerà i seguenti strumenti:
\begin{itemize}
	\item \textbf{Telegram}: strumento di messaggistica usato per comunicazioni veloci tra i membri; 
	\item \textbf{Discord}: per comunicazioni specifiche o per riunioni interne;
	\item \textbf{Git}: sistema di controllo di versionamento;
	\item \textbf{Gitflow}: sistema per agevolare varie operazioni su Git;
	\item \textbf{GitHub}: per il versionamento e il salvataggio in remoto di tutti i file riguardanti il progetto;
	\item \textbf{GanttProject}: software OpenSource\glo usato per la realizzazione dei diagrammi di Gantt;
	\item \textbf{Google Doc}: editor di testo cloud, usato per tenere degli appunti modificabili da tutti;
	\item \textbf{Google Drive}: utilizzato per il salvataggio in remoto dei file non sottoposti a versionamento, in modo da essere reperibili a tutti i membri;
	\item \textbf{Google Calendar}: per tenere traccia delle varie scadenze o riunioni fissate;
	\item \textbf{Skype}: servizio che offre possibilità di fare videoconferenze e chiamate VoIP, utilizzato per comunicare con il proponente;
	\item \textbf{Sistema Operativo}: i requisiti non indicano la necessità di usare un sistema operativo specifico, verranno quindi utilizzati Windows, Linux e Mac OS dai diversi membri del team.
\end{itemize}

\subsection{Formazione dei membri del gruppo}
\subsubsection{Scopo}
Lo scopo del processo di formazione è garantire che ogni membro del gruppo abbia conoscenze e capacità sufficienti per svolgere le attività assegnategli. Per una migliore organizzazione, il lavoro sarà suddiviso in compiti (task). 

\subsubsection{Aspettative}
L’aspettativa del gruppo, nel caso di questo processo, è riuscire a ottimizzare i tempi non impiegando tutti i membri nell’apprendimento di uno stesso strumento o tecnologia, ma lavorare in modo parallelo, al fine di assimilare più concetti possibili per poi condividerli con gli altri componenti.

\subsubsection{Descrizione}
Il processo di formazione è composto da un insieme di attività (task) che definiscono come vengono formati i componenti del gruppo. Lo studio necessario per svolgere i compiti e utilizzare gli strumenti viene principalmente eseguito in modo individuale e autonomo da un componente del gruppo definito periodicamente dal \textit{Responsabile di Progetto}. Al termine dello studio, egli deve comunicare ciò che ha imparato agli altri membri per garantire che tutti apprendano le nozioni. Qualora un componente del gruppo ritenga di non essere in grado di svolgere un task, dovrà segnalarlo immediatamente al \textit{Responsabile di Progetto} che dovrà organizzare le attività necessarie all'apprendimento.

\subsubsection{Attività}
\paragraph{Guide e documentazione}\mbox{} \\ \mbox{} \\
I membri del gruppo \textit{TeamAFK} provvedono in modo autonomo allo studio individuale delle tecnologie che verranno utilizzate nel corso del progetto. Si vuole operare secondo il principio del miglioramento continuo. Il versionamento dei prodotti servirà anche per apprendere dall'operato altrui, in modo da integrare le conoscenze personali migliorando la qualità e l'efficienza delle
attività.

\paragraph{Condivisione del materiale}\mbox{} \\ \mbox{} \\
Ogni componente del gruppo è libero di utilizzare per l'apprendimento personale altro materiale oltre a quello indicato in questo documento. Nel caso in cui lo ritenesse utile, potrà inoltre condividere tale materiale, utilizzando il canale comunicativo principale del gruppo (Telegram).

\subsubsection{Metriche di qualità}
Per questo processo non sono state definite metriche di qualità specifiche.

\subsubsection{Strumenti di supporto}
Per questo processo non vengono utilizzati degli strumenti di supporto.