\section{Installation}
\subsection{The plug-in}
For the installation of the the plug-in, which is a component of our product,  it is necessary to download the repository at the following link:\\ \url{https://github.com/teamafkSWE/PredireInGrafana-SW}\\
All the downloaded files must be inserted into the correct folder, based on the OS used:
\begin{itemize} 
\item Windows: \texttt{bin\textbackslash data\textbackslash plugins};
\item GNU\textbackslash Linux, MacOs : \texttt{var\textbackslash lib\textbackslash grafana\textbackslash plugins}.
\end{itemize}

\subsection{External training tool}
The external training tool is available online at the following link : \url{https://trainingtool.000webhostapp.com/}.\\

\subsection{Grafana}
All documetation about Grafana's installation is available on  \url{grafana.com/grafana/download}.\\For a correct using of the plugin you must select Grafana 6.7.3 version.
Here the correct download for the operating system of your choice can be found, all major OSs are available : MacOS, Windows and GNU\slash Linux and a step-by-step installation guide.
\subsubsection{Grafana WEB service}

To launch the Grafana WEB service, the following steps must be followed depending on which operating system is being used:

\begin{itemize}
\item\textbf{Linux}: give the following command to a in instance of a terminal:\texttt{sudo service grafana-server start} ;
\item\textbf{Windows}: the extracted Grafana folder contains the folder "bin" with the WEB services setup file, double click on \texttt{grafana-server.exe} ;
\item\textbf{Mac}: in the "bin" folder open an instance of the  terminal and give the following command: \texttt{./grafana-server web}.
\end{itemize}

Sould the user then open the preferred browser and connect to the default Grafana local host : \texttt{http://localhost:3000/}.
The credentials required for a first-time run are "admin" for the username field and "admin" for the password field
\begin{figure}[H]
\centering
\includegraphics[scale=0.65]{img/install/login.jpg}
\caption{Grafana login page}
\end{figure}