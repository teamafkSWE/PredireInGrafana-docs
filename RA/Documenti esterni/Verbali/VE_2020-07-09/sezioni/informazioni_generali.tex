\section{Informazioni generali}
\subsection{Informazioni incontro}
\begin{itemize}
	\item \textbf{Luogo}: Skype;
	\item \textbf{Data}: 2020-07-09;
	\item \textbf{Ora di inizio}: 11:00;
	\item \textbf{Ora di fine}: 12:15;
	\item \textbf{Partecipanti}:
	\begin{itemize}
		\item tutti i membri;
		\item Gregorio Piccoli (proponente).
	\end{itemize}
\end{itemize}

\subsection{Topic}
Durante questo incontro il gruppo ha mostrato al proponente la versione finale del prodotto. Qualora vi fossero delle migliorie da apportare, il \textit{TeamAFK} si impegnerà per implementarle in breve tempo e fornire al proponente il prodotto finale desiderato.

\subsubsection{Punto 1}
Il gruppo ha esposto quanto prodotto, a seguito delle richieste del proponente pervenute durante l'ultimo incontro del 2020-06-05. 
Le funzionalità aggiunte sono le seguenti:
\begin{enumerate}
\item visualizzazione contenuto file caricato;
\item mostrare la retta appresa dall’algoritmo a seguito dell’addestramento. Se sono presenti più valori per la x va fatto scegliere all’utente una delle possibili, mantenendo fissa la y;
\item mostrare di colore o forma diversa i dati sbagliati della SVM.
\end{enumerate}
Pertanto, risultano soddisfatte tutte le richieste del proponente.

\subsubsection{Punto 2}
Il proponente ci ha suggerito di aggiungere una textArea al tool di addestramento per l'inserimento di note specifiche per il file json prodotto. Inoltre, dev'essere possibile dare un nome specifico al file json prodotto. 

\pagebreak
\subsection{Tracciamento delle decisioni}
\centering
\begin{longtable}{ C{4cm}  C{12cm} }
\rowcolor{redafk}
\textcolor{white}{\textbf{Codice}} & \textcolor{white}{\textbf{Decisione}}\\	
		VE\_2020-07-09\_1 & Demo versione finale del prodotto, a seguito delle migliorie richieste in data 2020-06-05.\\
		VE\_2020-07-09\_2 & Inserire textArea per note e possibilità di dare un nome al file prodotto sul tool di addestramento.\\



\end{longtable}




