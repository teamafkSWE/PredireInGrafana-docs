\section{Valutazioni per il miglioramento}
In questa sezione viene riportata la valutazione fatta dal gruppo riguardo il lavoro svolto finora.
Lo scopo di questa scelta è trattare i problemi sorti e procedere alla loro più efficiente risoluzione
in modo tale che non si verifichino in futuro. \\
Verrano dunque tracciati problemi riguardanti i seguenti ambiti: \begin{itemize}
\item \textbf{Organizzazione}: vengono analizzati i problemi riguardanti l'organizzazione e la comunicazione all'interno del gruppo;
\item \textbf{Ruoli}: vengono analizzati i problemi riguardanti il corretto svolgimento di un ruolo;
\item \textbf{Strumenti di lavoro}: vengono analizzati i problemi riguardanti l'uso degli strumenti scelti.
\end{itemize}
Poichè non vi è una persona esterna che possa dare una valutazione oggettiva, ogni problema viene sollevato sulla base dell'autovalutazione dei soli membri del gruppo. Nonostante sia un sistema poco efficace, il gruppo ha beneficiato di questa scelta dal punto di vista comunicativo e produttivo, migliorando progressivamente la qualità del lavoro.\\
Questa sezione verrà aggiornata con l'avanzamento del prodotto riportando nuove problematiche, qualora queste dovessero verificarsi.

\subsection{Valutazioni sull'organizzazione}
\subsubsection{RR}
\begin{table}[H]
\caption{Problematiche relative all'organizzazione durante il periodo di RR}
\begin{center}
\begin{tabular}{ C{2.5cm} L{5cm} c L{5.5cm} }
\rowcolor{redafk}
\textcolor{white}{\textbf{Problema}} & \centerline{\textcolor{white}{\textbf{Descrizione}}} & \textcolor{white}{\textbf{Gravità}} & \centerline{\textcolor{white}{\textbf{Soluzione}}}\\
Incontro tra stakeholders\glo & A causa del Covid19, gli stakeholders hanno dovuto adattarsi alle restrizioni imposte, e tuttora in corso, impiegando tecnologie di comunicazione adatte allo smart working. & Bassa & Gli stakeholders hanno quindi utilizzato le tecnologie di comunicazione riportate nelle \textit{Norme di Progetto} per proseguire il progetto senza ulteriori intoppi. \\
\end{tabular}
\end{center}
\end{table}

\paragraph*{Considerazioni}\mbox{} \\ \mbox{} \\
In relazione al ciclo di Deming\glo si possono fare alcune considerazioni riguardo l'organizzazione del team.
Infatti l'obbiettivo (plan) è quello di individuare gli strumenti necessari a raggiungere una comunicazione fluida con gli stackeholders. Per raggiungere tale obiettivo sono state necessarie alcune prove (do), per via dei diversi mezzi di comunicazione a nostra disposizione. Infine sono stati scelti (act) gli opportuni mezzi sulla base dei riscontri (check) dei membri.

\subsubsection{RQ}
\begin{table}[H]
\caption{Problematiche relative all'organizzazione durante il periodo di RQ}
\begin{center}
\begin{tabular}{ C{2.5cm} L{5cm} c L{5.5cm} }
\rowcolor{redafk}
\textcolor{white}{\textbf{Problema}} & \centerline{\textcolor{white}{\textbf{Descrizione}}} & \textcolor{white}{\textbf{Gravità}} & \centerline{\textcolor{white}{\textbf{Soluzione}}}\\
Gestione del tempo a disposizione & A causa dei vari esami arretrati da parte di alcuni componenti del team, è risultato difficile gestire il tempo a dispozione in modo ottimale per proseguire il progetto senza intoppi. & Alta & Il \textit{Responsabile di Progetto} ha suddiviso nuovamente i vari task da svolgere in base alle possibilità di ogni componente. È stato quindi deciso di affidare ai membri più impegnati i compiti meno complicati, per permettere lo svolgimento del progetto in modo collaborativo e parallelo.
\end{tabular}
\end{center}
\end{table}

\paragraph*{Considerazioni}\mbox{} \\ \mbox{} \\
Il \textit{Responsabile di Progetto} ha avuto ruolo chiave nella risoluzione di questo problema. La soluzione adottata infatti ha permesso di proseguire lo sviluppo del progetto (documenti e software) senza rallentamenti. Il \textit{TeamAFK} ha già preso in considerazione la possibilità che questo problema possa ripresentarsi durante la Revisione di Accettazione.
\pagebreak
\subsection{Valutazioni sui ruoli}
\subsubsection{RR}
\begin{table}[H]
\caption{Problematiche relative ai ruoli riscontrati durante la RR}
\begin{center}
\begin{tabular}{ C{2.5cm} L{5cm} c L{5.5cm} }
\rowcolor{redafk}
\textcolor{white}{\textbf{Problema}} & \centerline{\textcolor{white}{\textbf{Descrizione}}} & \textcolor{white}{\textbf{Gravità}} & \centerline{\textcolor{white}{\textbf{Soluzione}}}\\
Ruolo di \textit{Responsabile} & A causa dell'inesperienza, chi ha lavorato come \textit{Responsabile} ha avuto difficoltà nella suddivisione bilanciata delle ore tra i membri provocando diverse ridistribuzioni delle ore. & Alta & Per evitare eventuali ritardi nelle consegne, il gruppo ha deciso di dedicare del tempo per analizzare meglio la mole di lavoro e compiere così una più accurata distribuzione delle ore. \\
\end{tabular}
\end{center}
\end{table}

\paragraph*{Considerazioni}\mbox{} \\ \mbox{} \\
Per via dell'inesperienza del team non è stato possibile stabilire  il miglior approccio nella gestione del tempo e per questo sono emerse alcune difficolte per alcuni ruoli. Nonostante ciò tramite le segnalazioni dei membri (check), è possibile comprendere e risolvere le problematiche adottando anche differenti approcci. Il tutto per ottimizzare il tempo messo a disposizione per ciascun componente e quindi rispettare le scadenze (plan).

\subsubsection{RP}
\begin{table}[H]
\caption{Problematiche relative ai ruoli riscontrati durante la RP}
\begin{center}
\begin{tabular}{ C{2.5cm} L{5cm} c L{5.5cm} }
\rowcolor{redafk}
\textcolor{white}{\textbf{Problema}} & \centerline{\textcolor{white}{\textbf{Descrizione}}} & \textcolor{white}{\textbf{Gravità}} & \centerline{\textcolor{white}{\textbf{Soluzione}}}\\
Ruolo di \textit{Progettista} & L’attività di progettazione è stata molto complessa e abbiamo riscontrato più difficoltà di quanto preventivato. & Alta & Si è deciso di assegnare più ore ai progettisti a scapito di altri ruoli per riuscire a comprendere e realizzare una buona architettura del nostro prodotto. \\
\end{tabular}
\end{center}
\end{table}
\paragraph*{Considerazioni}\mbox{} \\ \mbox{} \\
La soluzione adottata ha migliorato la situazione e ha permesso di svolgere il lavoro pianificato rispettando tempi e budget.
\pagebreak
\subsection{Valutazioni sugli strumenti di lavoro}
\subsubsection{RR}
\begin{longtable}{ C{2.3cm} L{5.5cm} c L{5.5cm} }
\rowcolor{white}\caption{Problematiche relative agli strumenti di lavoro durante la RR}\\
		\rowcolor{redafk}
\textcolor{white}{\textbf{Problema}} & \centerline{\textcolor{white}{\textbf{Descrizione}}} & \textcolor{white}{\textbf{Gravità}} & \centerline{\textcolor{white}{\textbf{Soluzione}}}\\
		\endfirsthead
		\rowcolor{white}\caption[]{(continua)} \\
		\rowcolor{redafk}
\textcolor{white}{\textbf{Problema}} & \centerline{\textcolor{white}{\textbf{Descrizione}}} & \textcolor{white}{\textbf{Gravità}} & \centerline{\textcolor{white}{\textbf{Soluzione}}}\\
		\endhead
GitHub & Si sono riscontrati in più occasioni
conflitti sui file in cui si stava lavorando e il tempo utilizzato per risolverli è stato sottratto dal tempo di lavoro. & Media & Il gruppo è stato istruito sull’uso di specifici branch\glo in modo tale che la modifiche di tutti i componenti si potessero integrare con il proprio lavoro senza che quest’ultimo potesse avere dei conflitti. \\
\LaTeX{} & A causa dell’inesperienza di
alcuni membri del gruppo nell’utilizzo
di questo strumento, si sono riscontrate diverse
difficoltà sopratutto nella costruzione di tabelle e nell'inserimento di formule matematiche. & Bassa & Per risolvere in breve tempo questa problematica, si è deciso di affiancare ai membri meno esperti chi sapeva già utilizzare i comandi di \LaTeX{} dando così la possibilità ai primi di imparare e permettendo ai secondi di non subire grossi rallentamenti nel lavoro. \\
\end{longtable}

\subsubsection{RP}
\begin{longtable}{ C{2.3cm} L{5.5cm} c L{5.5cm} }
\rowcolor{white}\caption{Problematiche relative agli strumenti di lavoro durante la RP}\\
		\rowcolor{redafk}
\textcolor{white}{\textbf{Problema}} & \centerline{\textcolor{white}{\textbf{Descrizione}}} & \textcolor{white}{\textbf{Gravità}} & \centerline{\textcolor{white}{\textbf{Soluzione}}}\\
		\endfirsthead
		\rowcolor{white}\caption[]{(continua)} \\
		\rowcolor{redafk}
\textcolor{white}{\textbf{Problema}} & \centerline{\textcolor{white}{\textbf{Descrizione}}} & \textcolor{white}{\textbf{Gravità}} & \centerline{\textcolor{white}{\textbf{Soluzione}}}\\
		\endhead
IntelliJ & A causa dell’inesperienza di alcuni membri del gruppo nell’utilizzo
di questo IDE, si sono riscontrate alcune difficoltà nell'apprendimento delle funzionalità necessarie per lo sviluppo del software. & Bassa & Per risolvere in breve tempo questa problematica, si è deciso di affiancare ai membri meno esperti chi sapeva già utilizzare questo strumento. \\
NPM & Si sono riscontrate delle problematiche durante la fase di configurazione di questo strumento, dovute soprattutto all'installazione di quest'ultimo in diversi sistemi operativi (Windows, Linux). & Media & Il \textit{TeamAFK} ha deciso di installare una versione stabile comune di questo strumento nei sistemi operativi in uso, evitando problemi di integrazione.
\end{longtable}

\pagebreak
\subsection{Valutazione globale}

\subsubsection{Analisi}
La prima critica costruttiva che ci poniamo riguarda il periodo di analisi.
Durante la prima parte del progetto, che dovrebbe essere dedicata maggiormente
all’analisi e alla definizione dei requisiti del nostro prodotto, non
ci siamo soffermati a sufficienza per comprenderli fino in fondo. Perciò, nei
periodi successivi, abbiamo dovuto spendere più ore nel ruolo di analista, riducendo
di conseguenza il numero di ore dedicate ad altri ruoli che avrebbero
potuto portare allo sviluppo di funzionalità aggiuntive.

\subsubsection{Sviluppo ed implementazione dei test}
La seconda critica costruttiva che ci poniamo riguarda lo sviluppo ed implementazione dei test, in modo particolare i test di unità. Quest'ultimi sono risultati non banali da sviluppare ed inizialmente ci siamo posti l’obiettivo di implementare più test possibili, indistintamente dalla loro priorità. In questo modo abbiamo impiegato un numero importante di ore nel ruolo di verificatore mentre se avessimo implementato i test con più raziocinio, avremmo avuto più ore da impiegare in altri ruoli, quali progettista e programmatore, per sviluppare funzionalità aggiuntive o raffinare alcuni dettagli.

\subsubsection{Valutazione per il miglioramento}
L'alternativa più efficace al nostro processo di miglioramento continuo sarebbe stata l’implementazione del ciclo di Deming: plan - do - check - act. Consiste nella
pianificazione dell’azione migliorativa (plan), nel monitoraggio della sua esecuzione
(do), nella verifica del suo esito (check) e nella valutazione critica
della sua efficacia (act) che può portare ad un’approvazione o, in caso di
insuccesso, ad un’ulteriore ripetizione del ciclo. Seguendo questo approccio,
avremmo potuto identificare le possibili migliorie da apportare in modo più
efficace e più dettagliato.

\subsubsection{Conclusioni}
Le metriche di qualità che ci siamo posti inizialmente sono state soddisfatte
con valori di accettabilità e spesso anche di ottimalità quindi siamo soddisfatti del nostro lavoro. Abbiamo avuto dei riscontri sempre più positivi da parte del committente e del proponente, e proprio quest’ultimo, durante uno degli ultimi incontri, si è detto soddisfatto del prodotto che abbiamo sviluppato.