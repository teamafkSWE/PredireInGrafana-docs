\section{Qualifica di prodotto}
\subsection{Scopo}
Per stabilire la Qualità di prodotto, il team di Quality Management\glo usa come riferimento informativo \textbf{ISO/IEC 9126} per stabilire il modello della qualità del software. Per decretare il raggiungimento di un determinato obiettivo di qualità, ogni voce trattata è accompagnata da un apposito parametro.
\subsection{Obiettivi}
Gli obiettivi di qualità che il team di Quality Management vuole raggiungere sono:
\begin{itemize}
\item affidabilità;
\item usabilità.
\end{itemize}
\subsection{Metriche generali}
Per una visione generale della qualità del prodotto vengo le adottate le misure di qualità riportate di seguito.
\subsubsection{MG01 - Percentuale di Metriche Soddisfatte}
Misura la percentuale di metriche soddisfatte rispetto alla totalità di metriche da utilizzare nel progetto.\\ \\ 
\textbf{Parametri adottati:} 
\begin{itemize}
\item range accettabile: [$90\%$, $95\%$);
\item range ottimale: [$95\%$, $100\%$].
\end{itemize}
 
\subsubsection{MG02 - Percentuale di Requisiti Obbligatori Soddisfatti}
Misura la percentuale di requisiti oblligatori soddisfatti rispetto alla totalità dei requisiti obbligatori da sviluppare e rispettare.\\ \\ 
\textbf{Parametri adottati:} 
\begin{itemize}
\item range accettabile: [$60\%$, $99\%$);
\item range ottimale: 100\%.
\end{itemize}

\subsubsection{MG03 - Percentuale di Requisiti Desiderabili Soddisfatti}
Misura la percentuale di requisiti desiderabili e soddisfatti rispetto alla totalità dei requisiti desiderabili da sviluppare e rispettare.\\ \\ 
\textbf{Parametri adottati:} 
\begin{itemize}
\item range accettabile: [$70\%$, $90\%$);
\item range ottimale: [$90\%$, $100\%$].
\end{itemize}

\subsection{Metriche della documentazione}
In relazione agli obiettivi prestabiliti, il team adotta i diversi strumenti per misurare la qualità del prodotto, riportati di seguito.
\subsubsection{MD01 - Indice di Gulpease}
L'Indice di Gulpease registra la leggibilità di un documento. \\ \\ 
\textbf{Parametri adottati:} 
\begin{itemize}
\item range accettabile: [40, 60);
\item range ottimale: [60, 100].
\end{itemize}

\subsubsection{MD02 - Indice Fog}
Misura la lungezza media delle parole e delle frasi presenti in un documento, così da comprendere la loro complessità.\\
\textbf{Parametri adottati:} 
\begin{itemize}
\item range accettabile: [5, 13);
\item range ottimale: [13, 20].
\end{itemize}

\subsection{Metriche del codice}
\subsubsection{MS01 - Linee di Codice}
È la metrica che registra la dimensione media di tutto il codice sorgente di un metodo.\\ \\
\textbf{Parametri adottati:}
\begin{itemize}
\item range accettabile: (10, 20];
\item range ottimale: [1, 10].
\end{itemize}
\subsubsection{MS02 - Numero dei Metodi}
Questa metrica conteggia il numero di metodi presenti nella classe di un oggetto (file).\\ \\ 
\textbf{Parametri adottati:} 
\begin{itemize}
\item range accettabile: (8, 15];
\item range ottimale: [0, 8].
\end{itemize}
\subsubsection{MS03 - Numero di Parametri}
Questo strumento tiene conto del numero medio di parametri formali di un metodo.\\ \\ 
\textbf{Parametri adottati:} 
\begin{itemize}
\item range accettabile: (3, 6];
\item range ottimale: [0, 3].
\end{itemize}
\subsubsection{MS04 - Commenti per Linee di Codice}
È il rapporto tra le righe di commento e il codice effettivo.\\ \\ 
\textbf{Parametri adottati:} 
\begin{itemize}
\item range accettabile: [0.05, 0.10);
\item range ottimale: [0.10, 0.20].
\end{itemize}
\subsubsection{MS05 - Fan-In}
Misura il numero di funzioni o metodi che invocano altre funzioni o metodi.\\
Per questa metrica non  è stato fissato un range ottimale, ma ci limiteremo soltanto ad indicare il numero effettivo.

\subsubsection{MS06 - Fan-Out}
Misura il numero di funzioni o metodi che vengono invocate da altre funzioni o metodi.\\Per questa metrica non è stato fissato un range ottimale, ma ci limiteremo soltanto ad indicare il numero effettivo.

\subsubsection{MS07 - Code Coverage}
È la metrica con il compito di misurare l'indice di copertura del codice da parte dei test in termini percentuali.\\ \\ 
\textbf{Parametri adottati:} 
\begin{itemize}
\item range accettabile: [70, 80)\%;
\item range ottimale: [80, 100]\%.
\end{itemize}
Seppur l'obiettivo del team di sviluppo sia quello di avere una Code Coverage del 100\%, tale traguardo potrebbe non essere raggiunto in quanto comporterebbe un aumento dei costi di progetto che risulterebbero troppo elevati.
\subsubsection{MS08 - Passed Test Cases Percentuage}
Misura la percentuale dei test superati sul totale dei test eseguiti.\\
\textbf{Parametri adottati:} 
\begin{itemize}
\item range accettabile: [90, 95)\%;
\item range ottimale: [95, 100]\%.
\end{itemize}

\subsubsection{MS09 - Failed Test Cases Percentuage}
Misura la percentuale dei test falliti sul totale dei test eseguiti.\\
\textbf{Parametri adottati:} 
\begin{itemize}
\item range accettabile: (5, 10]\%;
\item range ottimale: [0, 5]\%.
\end{itemize}

\subsubsection{MS10 - Requisiti obbligatori implementati}
Misura la percentuale dei test implementati sul totale dei test previsti.\\
\textbf{Parametri adottati:} 
\begin{itemize}
\item range accettabile: [75, 99]\%;
\item range ottimale: 100\%.
\end{itemize}

\subsubsection{MS11 - Requisiti desiderabili implementati}
Misura la percentuale dei test di requisiti desiderabili di implementati sul totale di quelli previsti.\\
\textbf{Parametri adottati:} 
\begin{itemize}
\item range accettabile: (60, 90]\%;
\item range ottimale: [90, 100]\%.
\end{itemize}

\subsection{Riepilogo metriche}
\begin{longtable}{C{2cm} C{4cm} C{5cm}}
\rowcolor{white}
\caption{Tabella riepilogativa delle metriche per la qualità generale del prodotto}\\
\rowcolor{redafk}
	\textcolor{white}{\textbf{Codice}} &
	\textcolor{white}{\textbf{Nome}} &
	\textcolor{white}{\textbf{Range}} \\
		\endfirsthead
		\rowcolor{white}\caption[]{(continua)} \\
		\rowcolor{redafk}
\textcolor{white}{\textbf{Codice}} &
\textcolor{white}{\textbf{Nome}} &
\textcolor{white}{\textbf{Range}} \\
		\endhead
MG01 & 
Percentuale di Metriche Soddisfatte &
\textbf{Accettabile}: [$90\%$, $95\%$)
\textbf{Ottimale}: [$95\%$, $100\%$] \\
MG02 &
Percentuale di Requisiti Obbligatori Soddisfatti &
\textbf{Accettabile}: [$75\%$, $99\%$];
\textbf{Ottimale}: 100\% \\
MG03 &
Percentuale di Requisiti Desiderabili Soddisfatti&
\textbf{Accettabile}: [$60\%$, $90\%$);
\textbf{Ottimale}: [$90\%$, $100\%$].\\
\end{longtable}

\begin{longtable}{C{2cm} C{4cm} C{5cm}}
\rowcolor{white}
\caption{Tabella riepilogativa delle metriche per la qualità dei documenti}\\
\rowcolor{redafk}
	\textcolor{white}{\textbf{Codice}} &
	\textcolor{white}{\textbf{Nome}} &
	\textcolor{white}{\textbf{Range}} \\
		\endfirsthead
		\rowcolor{white}\caption[]{(continua)} \\
		\rowcolor{redafk}
\textcolor{white}{\textbf{Codice}} &
\textcolor{white}{\textbf{Nome}} &
\textcolor{white}{\textbf{Range}} \\
		\endhead
MD01 &
Indice di Gulpease &
\textbf{Accettabile}: [40, 60)
\textbf{Ottimale}: [60, 100] \\
MD02 & Indice Fog &
\textbf{Accettabile}: [5, 13)
\textbf{Ottimale}: [13, 20] \\
\end{longtable}

\begin{longtable}{C{2cm} C{4cm} C{5cm}}
\rowcolor{white}
\caption{Tabella riepilogativa delle metriche per la qualità del codice}\\
\rowcolor{redafk}
	\textcolor{white}{\textbf{Codice}} &
	\textcolor{white}{\textbf{Nome}} &
	\textcolor{white}{\textbf{Range}} \\
		\endfirsthead
		\rowcolor{white}\caption[]{(continua)} \\
		\rowcolor{redafk}
\textcolor{white}{\textbf{Codice}} &
\textcolor{white}{\textbf{Nome}} &
\textcolor{white}{\textbf{Range}} \\
		\endhead
MS01 &
Linee di Codice &
\textbf{Accettabile}: (10, 20]
\textbf{Ottimale}: [1, 10] \\
MS02 &
Numero dei Metodi &
\textbf{Accettabile}: (8, 15]
\textbf{Ottimale}: [0, 8] \\

MS03 &
Numero di Parametri &
\textbf{Accettabile}: (3, 6]
\textbf{Ottimale}: [0, 3] \\

MS04 &
Commenti per Linee di Codice &
\textbf{Accettabile}: [0.05, 0.10)
\textbf{Ottimale}: [0.10, 0.20] \\

MS05 & 
Fan-In & 
\#effettivo\_funzioni
 \\

MS06 & 
Fan-Out & 
\#effettivo\_funzioni
 \\

MS07 & 
Code Coverage &
\textbf{Accettabile}: [70, 80)\%
\textbf{Ottimale}: [80, 100]\% \\


MS08 &
Passed Test Cases  Percentuage &
\textbf{Accettabile}: [90, 95)\%
\textbf{Ottimale}: [95, 100]\%.
\\

MS09 &
Failed Test Cases  Percentuage &
\textbf{Accettabile}: (5, 10]\%
\textbf{Ottimale}: [0, 5]\%.
\\
MS10 &
Requisiti obbligatori implementati &
\textbf{Accettabile}: [75, 99]\%;
\textbf{Ottimale}: 100\% \\

MS11 &
Requisiti desiderabili implementati &
\textbf{accettabile}: (60, 90]\%
\textbf{Ottimale}: [90, 100]\%
\end{longtable}

