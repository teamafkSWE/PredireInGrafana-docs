\documentclass[a4paper, oneside, openany, dvipsnames, table, 12pt]{article}
\usepackage{../../../../Template/AFKstyle}
\usepackage{hyperref}
\usepackage{verbatim} %per commenti di più righe \begin{comment} \end{comment}
\usepackage{amsmath}
\newcommand{\Titolo}{Verbale interno 2020-06-05}

\newcommand{\Gruppo}{TeamAFK}

\newcommand{\Redattori}{}

\newcommand{\Verificatori}{}

\newcommand{\pathimg}{../../../../Template/img/logoAFK.png}

\newcommand{\Approvatore}{}

\newcommand{\Distribuzione}{Prof. Vardanega Tullio \newline Prof. Cardin Riccardo \newline TeamAFK}

\newcommand{\Uso}{Interno}

\newcommand{\NomeProgetto}{"Predire in Grafana"}

\newcommand{\Mail}{gruppoafk15@gmail.com}

\newcommand{\Versionedoc}{1.0.0}

\newcommand{\DescrizioneDoc}{Riassunto dell'incontro del gruppo \textit{TeamAFK} tenutosi il 2020-06-05.}


\makeindex

\begin{document}
\copertina{}

%------------------ COLORI TABELLE 
\definecolor{pari}{RGB}{255, 207, 158} %{HTML}{E1F5FE} %azzurrino
\definecolor{dispari}{HTML}{FAFAFA} %bianco/grigetto 

%definizione colori per tabelle (tranne copertina)
\definecolor{redafk}{RGB}{255, 133, 51}
\definecolor{grey2}{RGB}{204, 204, 204}
\definecolor{greyRowafk}{RGB}{234, 234, 234}
\definecolor{lastrowcolor}{RGB}{176, 196, 222} %steel blue %{255,165,0} orange %{RGB}{255, 207, 158}
\rowcolors{2}{pari}{dispari}
\renewcommand{\arraystretch}{1.5}

%------------------

\newpage
\section*{Registro delle modifiche}
{
	\centering
	\begin{longtable}{ c c  C{4cm}  C{4cm}  C{3cm} }
		\rowcolor{redafk}
		\textcolor{white}{\textbf{Versione}} & \textcolor{white}{\textbf{Data}} & \textcolor{white}{\textbf{Descrizione}} & \textcolor{white}{\textbf{Nominativo}} & \textcolor{white}{\textbf{Ruolo}}\\		
		2.1.1 & 2020-05-27 & Aggiunta incrementi \S 6.2. Verificato il documento. & &\adm{} \newline \ver{} \\
		2.1.0 & 2020-05-26 & Refactoring \S 4.1 e \S 4.3. Apportate modifiche normative al \textit{Registro delle Modifiche}. Verificato il documento. & Davide Zilio \newline Victor Dutca & \adm{} \newline \ver{} \\
		2.0.0 & 2020-05-10 & Approvazione del documento per la RP. & Davide Zilio &\RdP{} \\
		1.1.0 & 2020-05-08 & Stesura \S 6.2. Verificato il documento. & Simone Meneghin \newline Alessandro Canesso &\adm{} \newline \ver{}\\
		1.0.2 & 2020-05-06 & Ampliamento \S 2. Verificato il documento. & Victor Dutca \newline Alessandro Canesso &\Res{} \newline \ver{}\\
		1.0.1 & 2020-04-30 & Correzioni e stesura \S 2.1. Verificato il documento. & Simone Meneghin \newline Alessandro Canesso &\Res{} \newline \ver{}\\
		1.0.0 & 2020-04-12 & Approvazione del documento per la RR. & Victor Dutca &\RdP{} \\
		0.7.0 & 2020-03-10 & Stesura \S B. Verificato il documento. & Alessandro Canesso \newline Olivier Utshudi &\Res{} \newline \ver{}\\
		0.6.0 & 2020-03-10 & Stesura \S 6. Verificato il documento. & Alessandro Canesso \newline Olivier Utshudi &\Res{} \newline \ver{}\\
		0.5.0 & 2020-03-06 & Stesura \S 5. Verificato il documento.  & Fouad Farid \newline Olivier Utshudi &\Res{} \newline \ver{}\\
		0.4.0 & 2020-04-04 & Stesura \S 4. Verificato il documento. & Simone Federico Bergamin \newline Simone Meneghin &\adm{} \newline \ver{}\\	
		0.3.0 & 2020-04-03 & Stesura \S 3. Verificato il documento.  & Simone Federico Bergamin \newline Olivier Utshudi &\adm{} \newline \ver{}\\	
		0.2.0 & 2020-03-30 & Stesura \S 2. Verificato il documento.  & Alessandro Canesso \newline Simone Meneghin &\Res{} \newline \ver{}\\	
		0.1.0 & 2020-03-30 & Stesura \S 1. Verificato il documento. & Alessandro Canesso \newline Simone Meneghin &\Res{} \newline \ver{}\\		
	\end{longtable}
} 



%Didascalia tabelle/immagini (prendono come riferimento la subsection)
\counterwithin{table}{subsection}
\counterwithin{figure}{subsection}
\newpage

%indice, indice figure e indice tabelle
\tableofcontents
\newpage
\begin{comment}
\listoffigures
\newpage
\listoftables
\newpage
\end{comment}

\section{Informazioni generali}
\subsection{Informazioni incontro}
\begin{itemize}
\item \textbf{Luogo}: Skype;
\item \textbf{Data}: 2020-04-28;
\item \textbf{Ora di inizio}: 14:30;
\item \textbf{Ora di fine}: 14:45;
\item \textbf{Partecipanti}:
	\begin{itemize}
		\item tutti i membri;
		\item Gregorio Piccoli (proponente).
	\end{itemize}
\end{itemize}

\subsection{Topic}
Esposizione di alcune domande al proponente e conseguente richiesta di chiarimenti, relativi lo sviluppo della Proof of Concept da presentare in Revisione di Progettazione, prevista per il 2020-05-18.\\
In particolare le domande, e relative risposte ed attuazione, sono le seguenti:
\begin{itemize}
	\item Il tool esterno deve essere accessibile da Grafana\glo?\\
	\textbf{Risposta}: non necessariamente, deve essere sviluppato nel modo più facile per il team;\\
	\textbf{Attuazione}: il \textit{TeamAFK} ha deciso di rendere il tool esterno accessibile da Grafana;
	\item Il tool deve essere un eseguibile o va bene come pagina web? \\
	\textbf{Risposta}: come preferite, non c'è un obbligo particolare; \\
	\textbf{Attuazione}: il \textit{TeamAFK} svilupperà il tool come pagina web interattiva;
	\item Che tipologia di plug-in dobbiamo sviluppare? Un plug-in App o Panel? \\
	\textbf{Risposta}: Panel;\\
	\textbf{Attuazione}: il plug-in verrà sviluppato come Panel plug-in;
	\item Dobbiamo usare le dashboard con i pannelli di Grafana o possiamo usare librerie javascript per i grafici? \\
	\textbf{Risposta}: utilizzare i pannelli di Grafana;\\
	\textbf{Attuazione}: verrà utilizzata la dashboard di Grafana per mostrare i vari grafici;
	\item Cosa vuole vedere lei nel PoC? \\
	\textbf{Risposta}: far vedere qualcosa che abbia una sua logica;\\
	\textbf{Attuazione}: il \textit{TeamAFK} prevede di sviluppare e quindi mostrare il tool esterno, in cui viene eseguito l'addestramento di una SVM o RL dato un \texttt{file.csv} in input contenente i dati utilizzati per tale addestramento. Sarà dunque possibile scaricare il \texttt{file.json} contenente la definizione dei predittori che verranno utilizzati per la previsione;
	\item Ha una documentazione più approfondita di Grafana? Quello che troviamo è poco chiaro e confuso.\\
	\textbf{Risposta}: no, purtroppo essendo un software open source è in continuo mutamento e di conseguenza la documentazione presente è scarna e poco approfondita;\\
	\textbf{Attuazione}: il \textit{TeamAFK} utilizza più fonti per ottenere le informazioni richieste.
\end{itemize}
A seguito delle risposte ricevute, il \textit{TeamAFK} è riuscito a risolvere i dubbi sopra descritti e quindi a continuare la progettazione e sviluppo dell'architettura software evitando errori progettuali.
\pagebreak

\end{document}