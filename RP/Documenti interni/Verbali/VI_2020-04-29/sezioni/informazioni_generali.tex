\section{Informazioni generali}
\subsection{Informazioni incontro}
\begin{itemize}
\item \textbf{Luogo}: Discord;
\item \textbf{Data}: 2020-04-29;
\item \textbf{Ora di inizio}: 9:00;
\item \textbf{Ora di fine}: 11:00;
\item \textbf{Partecipanti}:
	\begin{itemize}
		\item tutti i membri.
	\end{itemize}
\end{itemize}

\subsection{Topic}
\begin{itemize}
\item progettazione e sviluppo dell'architettura software a seguito dei chiarimenti ricevuti in data 2020-04-28 dal proponente;
\item correzioni da apportare ai documenti di progetto a seguito delle segnalazioni ricevute dal committente Prof. Tullio Vardanega in data 2020-04-27.
\end{itemize}

\subsubsection{Attuazione}

\paragraph{Progettazione e sviluppo dell'architettura software} \mbox{} \\ \mbox{} \\
I \textit{Progettisti} del \textit{TeamAFK} hanno deciso di attuare i consigli ricevuti dal proponente del progetto, consultabili nel verbale esterno del 2020-04-28. \\
L'architettura verrà realizzata come plug-in "panel", sviluppata come pagina web all'interno della dashboard di Grafana. Quest'ultima fornirà all'utente i grafici relativi ai dati contenuti nel database InfluxDB, modellati in relazione all'algoritmo di predizione contenuto nel \texttt{file.json} inserito in input.

\paragraph{Correzioni documenti}\mbox{} \\ \mbox{} \\
A seguito delle segnalazioni del committente Prof. Tullio Vardanega, il \textit{Responsabile di Progetto} ha subito provveduto alla suddivisione dei ruoli, per permettere sia la modifica e correzione dei documenti sia lo sviluppo del software senza incorrere in ritardi, rispettando quanto definito nel \textit{Piano di Progetto}.