\section{Informazioni generali}
\subsection{Informazioni incontro}
\begin{itemize}
\item \textbf{Luogo}: Discord;
\item \textbf{Data}: 2020-04-28;
\item \textbf{Ora di inizio}: 9:00;
\item \textbf{Ora di fine}: 11:30;
\item \textbf{Partecipanti}:
	\begin{itemize}
		\item tutti i membri.
	\end{itemize}
\end{itemize}

\subsection{Topic}
Durante questa fase sono sorti alcuni dubbi progettuali, pertanto è stato necessario fissare una riunione con il sig. Gregorio Piccoli.
I progettisti del \textit{TeamAFK} hanno formulato alcuni quesiti da esporre al proponente nella riunione di oggi, 2020-04-28, prevista alle ore 14:30.\\
Il chiarimento delle seguenti perplessità permette al team di sviluppo di procedere alla codifica del software senza rallentamenti e senza errori concettuali e, quindi, a rispettare gli obiettivi preposti nei tempi prefissati.
\begin{itemize}
\item Il tool esterno deve essere accessibile da Grafana?;
\item Il tool deve essere un eseguibile o va bene come pagina web?;
\item Che tipologia di plug-in dobbiamo sviluppare? Un plug-in App o Panel?;
\item Dobbiamo usare le dashboard con i pannelli di Grafana o possiamo usare librerie javascript per i grafici?;
\item Cosa vuole vedere lei nel PoC? Ha una richiesta particolare?;
\item Ha una documentazione più approfondita di Grafana? Quello che troviamo è poco chiaro e confuso.
\end{itemize}
