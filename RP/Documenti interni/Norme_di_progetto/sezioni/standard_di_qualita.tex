\section{Standard di qualità} \label{sez:standardQualita}
\label{sez:9126}
\subsection{ISO/IEC 9126}
Con la sigla ISO/IEC 9126 si individua una serie di normative e linee guida preposte a descrivere un modello di qualità del software. \\
La norma tecnica relativa alla qualità del software si compone in quattro parti:
\begin{itemize}
	\item modello della qualità del software;
	\item metriche per la qualità esterna;
	\item metriche per la qualità interna;
	\item metriche per la qualità in uso.
\end{itemize}
\subsubsection{Modello della qualità del software}
Il modello di qualità è classificato da sei caratteristiche generali e da varie sottocaratteristiche misurabili attraverso delle metriche.\\
Di seguito vengono definite tali caratteristiche.

\paragraph{Funzionalità} \mbox{} \\ \mbox{} \\
È la capacità del software di soddisfare determinate esigenze, stabilite nell' \textit{Analisi dei Requisiti}, necessarie per operare sotto condizioni specifiche. \\
In dettaglio il software deve soddisfare le seguenti caratteristiche:
\begin{itemize}
	\item \textbf{Appropriatezza}: le funzioni fornite dal software sono appropriate per i compiti ed obiettivi prefissati;
	\item \textbf{Accuratezza}: il software deve fornire risposte concordanti o i precisi effetti richiesti;
	\item \textbf{Interoperabilità}: capacità di interagire ed operare con uno o più sistemi specificati;
	\item \textbf{Conformità}: il prodotto deve essere in grado di aderire a standard, convenzioni e regolamentazioni;
	\item \textbf{Sicurezza}: è la capacità del software di proteggere informazioni e dati, negandone l'accesso a persone o sistemi non autorizzati e invece fornirlo a chi ne è effettivamente autorizzato.
\end{itemize}
\paragraph{Affidabilità}\mbox{} \\ \mbox{} \\
È la capacità del prodotto software di mantenere uno specificato livello di prestazioni quando usato in date condizioni. \\
Nello specifico il software deve soddisfare le seguenti caratteristiche:
\begin{itemize}
	\item \textbf{Maturità}: capacità di un prodotto software di evitare che si verifichino errori, malfunzionamenti o siano prodotti risultati non corretti;
	\item \textbf{Tolleranza agli errori}: il software deve mantenere dei predeterminati livelli di prestazioni anche in caso di un uso scorretto o di malfunzionamenti;
	\item \textbf{Recuperabilità}: capacità di ripristinare il livello appropriato di prestazioni a seguito di un malfunzionamento o di un errore. La caratteristica di recuperabilità è valutata dall'arco di tempo nel quale il software può risultare non accessibile;
	\item \textbf{Aderenza}: è la capacità di aderire a standard, regole e convenzioni riguardanti l'affidabilità.
\end{itemize}
\paragraph{Efficienza}\mbox{} \\ \mbox{} \\
Capacità del prodotto software di eseguire le proprie funzioni minimizzando il tempo necessario e sfruttando al meglio le risorse di cui necessita. \\
Nello specifico il software deve soddisfare le seguenti caratteristiche:
\begin{itemize}
	\item \textbf{Comportamento rispetto al tempo}: è la capacità di fornire adeguati tempi di risposta, elaborazione e velocità di attraversamento, sotto condizioni determinate;
	\item \textbf{Utilizzo delle risorse}: è la capacità di utilizzo di quantità e tipo di risorse in maniera adeguata;
	\item \textbf{Conformità}: è la capacità di aderire a standard e specifiche sull'efficienza\glo.
\end{itemize}
\paragraph{Usabilità}\mbox{} \\ \mbox{} \\
È la capacità del prodotto software di essere capito, appreso, usato e ben accettato dall'utente, anche quando usato sotto condizioni specificate. \\
Nello specifico il software deve soddisfare le seguenti caratteristiche:
\begin{itemize}
	\item \textbf{Comprensibilità}: esprime la facilità di comprensione dei concetti, funzionalità e utilizzo da parte dell'utente;
	\item \textbf{Apprendibilità}: è la capacità di ridurre l'impegno richiesto agli utenti per imparare ad usare l'applicazione;
	\item \textbf{Operabilità}: capacità di mettere in condizione gli utenti di farne uso per i propri scopi e controllarne l'uso;
	\item \textbf{Attrattivà:} è la capacità del software di essere piacevole all'uso per l'utente;
	\item \textbf{Conformità}: è la capacità del software di aderire a standard o convenzioni relativi all'usabilità.

\end{itemize}
\paragraph{Manutenibilità}\mbox{} \\ \mbox{} \\
La manutenibilità è la capacità del software di essere modificato, includendo correzioni, miglioramenti o adattamenti.
Nello specifico il software deve soddisfare le seguenti caratteristiche:
\begin{itemize}
	\item \textbf{Analizzabilità}: rappresenta la facilità con la quale è possibile analizzare il codice per localizzare un errore nello stesso;
	\item \textbf{Modificabilità}: è la capacità del prodotto software di permettere l'implementazione di una modifica;
	\item \textbf{Stabilità}: capacità del software di evitare effetti inaspettati a seguito di modifiche errate;
	\item \textbf{Testabilità}: è la capacità di essere facilmente testato per validare le modifiche apportate al software.
\end{itemize}
\paragraph{Portabilità}\mbox{} \\ \mbox{} \\
La portabilità è la capacità del software di essere trasportato da un ambiente\glo di lavoro ad un altro. Ambiente che può variare dall'hardware al software.
Nello specifico il software deve soddisfare le seguenti caratteristiche:
\begin{itemize}
	\item \textbf{Adattabilità}: la capacità del software di essere adattato per differenti ambienti senza dover applicare modifiche;
	\item \textbf{Installabilità}: capacità del software di essere installato in un diverso ambiente;
	\item \textbf{Conformità}: la capacità del prodotto software di aderire a standard e convenzioni relative alla portabilità;
	\item \textbf{Sostituibilità}: capacità di essere utilizzato al posto di un altro software per svolgere gli stessi compiti nello stesso ambiente.
\end{itemize}
\subsubsection{Metriche per la qualità esterna}
Servono a misurare i comportamenti del software sulla base dei test, in funzione degli obiettivi stabiliti. Viene rilevata tramite analisi dinamica\glo.

\subsubsection{Metriche per la qualità interna}
Metriche che si applicano al software non eseguibile durante le fasi di progettazione e codifica. Permettono di individuare eventuali problemi che potrebbero influire sulla qualità finale del prodotto prima che venga realizzato un eseguibile. Grazie alle misure effettuate tramite le metriche interne è possibile prevedere il livello di qualità esterna e di qualità in uso del prodotto finale, poiché entrambe vengono influenzate dalla qualità interna.\\
Viene rilevata tramite analisi statica\glo.

\subsubsection{Metriche per la qualità in uso}
La qualità in uso rappresenta il punto di vista dell'utente sul software. Il livello di qualità viene raggiunto dopo che sono stati raggiunti i livelli nella qualità esterna e interna. La qualità in uso, quindi, permette di abilitare specificati utenti ad ottenere specificati obiettivi con efficacia\glo, produttività, sicurezza e soddisfazione.


\subsection{ISO/IEC 15504}
Il modello ISO/IEC 15504, meglio conosciuto come SPICE (\textit{Software Process Improvement and Capability Determination}), è lo standard di riferimento per valutare in modo oggettivo la qualità dei processi nello sviluppo del software. \\
SPICE mette a disposizione una metrica per valutare diversi attributi per ogni processo ed
assegna un valore quantificabile ad ognuno di questi in modo tale da rendere esplicito come poter migliorare tale processo.

\subsubsection{Classificazione dei processi}
SPICE definisce una serie di sei livelli utilizzati per classificare la capacità\glo e la maturità\glo del processo software. Ogni livello è caratterizzato dal soddisfacimento degli attributi associati:
\begin{enumerate}
	\item \textbf{Incompleto}: il processo non è stato implementato o non ha raggiunto il successo desiderato;
	\item \textbf{Eseguito}: il processo è implementato e ha realizzato il suo obiettivo (conformità);\\
	Attributi:
	\begin{itemize}
		\item Process Performance.
	\end{itemize}
	\item \textbf{Gestito}: il processo è gestito e il prodotto finale è verificato, controllato e manutenuto (affidabilità);\\
	Attributi:
	\begin{itemize}
		\item  Performance Management;
		\item  Work Product Management.
	\end{itemize}
	\item \textbf{Stabilito}: il processo è basato sullo standard di processo (standardizzazione);\\
	Attributi:
	\begin{itemize}
		\item  Process Definition;
		\item  Process Deployment.
	\end{itemize}
	\item \textbf{Predicibile}: il processo è consistente e rispetta limiti definiti (strategico);\\
	Attributi:
	\begin{itemize}
		\item  Process Measurement;
		\item  Process Control.
	\end{itemize}
	\item \textbf{Ottimizzato}: il processo segue un miglioramento continuo per rispettare tutti gli obiettivi di
progetto;\\
Attributi:
	\begin{itemize}
		\item  Process Innovation;
		\item  Process Optimization.
	\end{itemize}
\end{enumerate}

Ogni attributo riceve una valutazione nella seguente scala, andando a definire il rispettivo livello di
capacità del processo:
\begin{itemize}
	\item N: non raggiunto (0 - 15\%);
	\item P: parzialmente raggiunto (>15\% - 50\%);
	\item L: largamente raggiunto (>50\%- 85\%);
	\item F: pienamente raggiunto (>85\% - 100\%).
\end{itemize}

\subsubsection{Linee guida}
Lo standard definisce delle linee guida per effettuare delle stime, tali linee guida sono:
\begin{itemize}
	\item processi di misurazione, descritti nel \textit{Piano di Progetto} ;
	\item modello di misurazione, descritto in questo documento;
	\item strumenti di misurazione, descritti nelle \textit{Norme di Progetto}.
\end{itemize}

\subsubsection{Competenze}
Per effettuare le misurazioni necessarie per determinare la qualità raggiunta, sono necessarie delle competenze. La mancanza di esperienza degli elementi del \textit{TeamAFK} fa sì che nessun membro possieda queste abilità, rendendo così impossibile la piena adesione allo standard. Tuttavia, ogni componente è chiamato a studiare SPICE e ad applicare al meglio le indicazioni descritte in questo documento e nelle \textit{Norme di Progetto}, al fine di perseguire un livello di qualità accettabile.






