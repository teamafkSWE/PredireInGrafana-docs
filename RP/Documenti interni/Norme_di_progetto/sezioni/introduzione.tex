\section{Introduzione}

\subsection{Scopo del documento}
Il seguente documento ha il compito di stabilire le regole che il team di sviluppo intende seguire in ogni attività di progetto, così da omologare il materiale prodotto.
I fornitori intendono adottare un approccio incrementale\glo, affinché lo sviluppo del prodotto si basi su decisioni prese di comune accordo. Perciò ogni componente del team deve far riferimento a questo documento per garantire la coesione e uniformità delle scelte prese.

\subsection{Scopo del prodotto}
Il capitolato {C4}\glo illustra il prodotto da fornire. Tale prodotto consiste in tool di addestramento ed un plugin\glo di Grafana\glo che prenderà il nome \textit{Predire in Grafana}. Entrambi gli applicativi verranno scritti in linguaggio JavaScript\glo. Il primo avrà il compito di produrre un file di estensione JSON\glo basato su dati di addestramento\glo. Il secondo invece si occuperà di leggere il file creato, effettuare previsioni basandosi su di esso e rendere disponibili al sistema i risultati ottenuti, in modo da poterli visualizzare in grafici e dashboard.
\subsection{Glossario}
Per evitare ambiguità nei documenti formali, viene fornito il documento \textit{Glossario\_v2.0.0}, contenente tutti i termini considerati di difficile comprensione. Perciò nella documentazione fornita ogni vocabolo contenuto in \textit{Glossario\_v2.0.0} è contrassegnato dalla lettera \textit{G} a pedice.
\subsection{Riferimenti}
\subsubsection{Riferimenti normativi}
\begin{itemize}
	\item \textbf{Capitolato d'appalto C4 - Predire in Grafana}: \\
	\url{https://www.math.unipd.it/~tullio/IS-1/2019/Progetto/C4.pdf}.	
\end{itemize}
\subsubsection{Riferimenti informativi}
\begin{itemize}
	\item \textbf{Change Management Process}: \\
	\href{https://www.blog-management.it/2018/04/10/change-management-project-management/}{https://www.blog-management.it/change-management-project-management}\\
	\href{https://www.digital4.biz/hr/hr-transformation/digital-transformation-e-change-management-vanno-avanti-di-pari-passo/}{https://www.digital4.biz/hr/hr-transformation/};
	\item \textbf{The three P's of Software Engineering}: \\
	\url{http://dwaynephillips.net/CutterPapers/ppp/ppp.htm}
	\item \textbf{Software Engineering - Ian Sommerville - 10th Edition}
	\item \textbf{Slide L05 del corso Ingegneria del Software - Ciclo di vita del software}: \\
	\url{https://www.math.unipd.it/~tullio/IS-1/2019/Dispense/L05.pdf};
	\item \textbf{Slide L06 del corso Ingegneria del Software - Gestione di Progetto}: \\
	\url{https://www.math.unipd.it/~tullio/IS-1/2019/Dispense/L06.pdf};	
	\item \textbf{Slide L12 del corso Ingegneria del Software - Qualità di Prodotto}: \\
	\url{https://www.math.unipd.it/~tullio/IS-1/2019/Dispense/L12.pdf};
	\item \textbf{Slide L13 del corso Ingegneria del Software - Qualità di Processo}: \\
	\url{https://www.math.unipd.it/~tullio/IS-1/2019/Dispense/L13.pdf};
\end{itemize}