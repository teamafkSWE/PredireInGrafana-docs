\section{Qualità di processo}
\subsection{Scopo}
Al fine di garantire la qualità del prodotto è necessario perseguire in primis la qualità dei processi che la definiscono. Si è deciso dunque di seguire il principio di miglioramento continuo (PDCA) e di adottare lo standard ISO/IEC 15504\footnote{ISO/IEC 15504: insieme di documenti di standard tecnici relativi ai processi di sviluppo del software e relative funzioni di business e, in particolare, alla loro valutazione.} denominato SPICE\glo: quest'ultimo permette di valutare il livello di maturità e capacità\glo (capability) dei processi, al fine di apportare modifiche migliorative. 
\subsubsection*{PDCA}
Le fasi descritte dal PDCA sono le seguenti: 
\begin{itemize}
\item \textbf{Plan}: fase di pianificazione dove si decidono e si individuano gli obiettivi di qualità e i risultati desiderati;
\item \textbf{Do}: si mette in atto il piano stabilito nella fase precedente;
\item \textbf{Check}: si verificano e si confrontano i dati in output dalla fase precedente con i risultati previsti in fase di Planning;
\item \textbf{Act}: si individuano le cause delle eventuali discordanze riscontrate in fase di Check e si determinano le azioni da intraprendere per risolverle e migliorare il processo.
\end{itemize}
\subsection{Obiettivi}
Sono fissati inoltre i seguenti obiettivi: \begin{itemize}
\item rispetto di tempi e costi descritti nel \textit{Piano\_di\_Progetto\_v1.0.0};
\item continuo miglioramento dei processi;
\item misurabilità dello stato dei processi.
\end{itemize}
\subsection{Metriche}
Per misurare la qualità, sono state scelte delle specifiche metriche che monitorano lo stato dei processi del progetto analizzando l'uso che essi fanno di tempo e denaro. Sono particolarmente utili per il \textit{Responsabile}, che può quindi decidere di apportare modifiche alla pianificazione quando necessario.\\
Ogni metrica conterrà:
\begin{itemize}
\item \textbf{Nome};
\item \textbf{Descrizione};
\item \textbf{Parametri}: range di valori su cui confrontare le misure ottenute. Sono definiti i seguenti intervalli: \begin{itemize}
\item \textbf{Accettabile}: intervallo in cui il valore misurato viene considerato sufficiente, seppur migliorabile;
\item \textbf{Ottimale}: intervallo in cui il valore misurato viene ritenuto ottimo.
\end{itemize}
Essi possono essere: \begin{itemize}
\item \textbf{Aperti}, se gli estremi non sono compresi. Esempio: (a, b) = $a < x < b$; 
\item \textbf{Chiusi}, se gli estremi sono compresi. Esempio: [a, b] = $a \leq x \leq b$;
\item \textbf{Limitati}, se gli estremi sono numeri finiti;
\item \textbf{Illimitati}, se almeno uno degli estremi è infinito.
\end{itemize}
\end{itemize}
\textbf{Attenzione}: in questo documento \textbf{non} saranno trattati la descrizione e gli strumenti per il calcolo delle metriche, reperibili invece nelle \textit{Norme\_di\_Progetto\_v1.0.0}.

\paragraph{MP01 - Schedule Variance} \mbox{} \\ \mbox{} \\
La Schedule Variance indica se una certa attività o processo è in anticipo, in pari, o in ritardo rispetto alla data di scadenza prevista. \\ \\ 
\textbf{Parametri adottati:} 
\begin{itemize}
\item range accettabile: ($ -\infty $, 2];
\item range ottimale: ($ -\infty $, 0].
\end{itemize}

\paragraph{MP02 - Budget Variance} \mbox{} \\ \mbox{} \\
Permette di controllare i costi sostenuti alla data corrente rispetto al budget preventivato. \\ \\ 
\textbf{Parametri adottati:}  
\begin{itemize}
\item range accettabile: [$-15\%$, $0\%$); 
\item range ottimale: $ \geq 0\%$.
\end{itemize}

\paragraph{MP03 - Produttività} \mbox{} \\ \mbox{} \\
Rappresenta la produttività media delle risorse impiegate, cioè delle persone coinvolte, nelle diverse fasi del progetto. \\ \\ 
\textbf{Parametri adottati:} 
\begin{itemize}
	\item range accettabile: [50, 100];
	\item range ottimale: $ > 100$.
\end{itemize}


