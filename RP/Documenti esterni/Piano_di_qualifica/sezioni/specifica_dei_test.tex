\section{Specifica dei test}
Per verificare la qualità del prodotto software, il gruppo fornitore ha deciso di adottare il \textbf{Modello di Sviluppo a V}\glo, sviluppando così una serie di test. Questi hanno lo scopo di controllare che tutte le unità di cui è composto il sistema siano state implementate correttamente, rispettando tutti gli aspetti del progetto.
Per semplificare la loro consultazione i test saranno suddivisi in categorie, per mezzo di tabelle, mostrando l'output prodotto e sottolineando se è un risultato atteso o non atteso.
\subsection{Stato dei test}
Per definire lo stato dei test, si usano le seguenti sigle:
\begin{itemize}
\item \textbf{I}: test implementato;
\item \textbf{NI}: test non implementato.
\end{itemize}

\begin{longtable}{C{2.5cm} C{2.5cm} L{8cm} C{2cm}}
\rowcolor{white}\caption{Tabella dei test} \\
		\rowcolor{redafk}
\textcolor{white}{\textbf{Codice}} &
\textcolor{white}{\textbf{Caso d'uso}} &
\textcolor{white}{\textbf{Descrizione}} &
\textcolor{white}{\textbf{Esito}} \\
		\endfirsthead
		\rowcolor{white}\caption[]{(continua)} \\
		\rowcolor{redafk}
\textcolor{white}{\textbf{Requisito}} &
\textcolor{white}{\textbf{Caso d'uso}} &
\textcolor{white}{\textbf{Descrizione}} &
\textcolor{white}{\textbf{Esito}} \\
		\endhead
%-------------------------------------------- Simo		
TSOF1 & UC1 &
L'utente  deve poter creare il file JSON\glo contenente il/i predittore/i\glo. \newline
All'utente viene chiesto di:
\begin{itemize}
	\item cliccare il pulsante “Carica Dati di Addestramento”;
	\item scegliere i dati di addestramento\glo da caricare;
	\item selezionare l’algoritmo di previsione\glo;
	\item conferma delle operazioni;
	\item salvataggio file JSON contenente i predittori.
\end{itemize} & NI \\

TSOF1.1 & UC1.1 &
L'utente  deve poter caricare i dati di addestramento. \newline All'utente viene chiesto di:
\begin{itemize}
 	\item cliccare il pulsante "Carica Dati di Addestramento";
 	\item verificare che si apra la finestra che visualizza il file system\glo.
\end{itemize} & I	\\


TSOF1.2 & UC1.2 &
L'utente  deve poter scegliere i dati di addestramento. \newline All'utente viene chiesto di:
\begin{itemize}
 	\item verificare che dalla finestra di dialogo siano visibili solo file CSV\glo;
	\item selezionare i dati di addestramento.
\end{itemize} 
& I \\
 
TSOF1.3 & UC1.3 & 
L'utente deve poter scegliere l'algoritmo di predizione. \newline All'utente viene chiesto di:
\begin{itemize}
	\item cliccare sulla Combo Box\glo con etichetta "Seleziona Algoritmo";
	\item scegliere uno degli algoritmi proposti (RL o SVM).
\end{itemize} & I \\
 
TSOF1.4 & UC1.4 & 
L'utente deve poter confermare la scelta dell'algoritmo. \newline All'utente viene chiesto di:
\begin{itemize}
	\item cliccare sul pulsante "Conferma".
\end{itemize} & NI \\

TSOF1.4.1 & UC9 & 
L'utente deve poter visualizzare un messaggio d'errore se la scelta dell'algoritmo non è compatibile con i dati di addestramento. \newline All'utente viene chiesto di:
\begin{itemize}
	\item verificare la visualizzazione dell'errore;
	\item verificare di essere rimandati al TSOF1.2.
\end{itemize} & I \\

TSOF1.5 & UC1.5 & 
L'utente deve poter denominare il file JSON e scegliere dove salvarlo. \newline All'utente viene chiesto di:
\begin{itemize}
	\item scegliere un nome per il file JSON;
	\item scegliere dove salvare il file JSON.
\end{itemize} & NI \\

TSOF1.5.1 & UC16 &
L'utente deve poter vedere la conferma dell'avvenuto salvataggio. \newline All'utente viene chiesto di:
\begin{itemize}
	\item visualizzare il messaggio di notifica "Avvenuto Successo Salvataggio File JSON";
	\item cliccare su "Conferma" per chiudere la notifica.
\end{itemize} & NI	\\

TSOF2 & UC2 &
L'utente  deve poter caricare il file JSON nel plug-in. \newline All'utente viene chiesto di:
\begin{itemize}
	\item cliccare il pulsante per caricare il file JSON;
	\item selezionare il file JSON;
	\item confermare il caricamento del file.
\end{itemize} & I	\\


TSOF2.1 & UC2.1 &
L'utente  deve poter caricare il file JSON. \newline All'utente viene chiesto di:
\begin{itemize}
	\item cliccare su "Carica JSON";
	\item verificare la visualizzazione della finestra di selezione file.
\end{itemize} & I	\\

TSOF2.1.1 & UC10 &
L'utente  deve poter visualizzare il messaggio di alert\glo del caricamento già avvenuto e caricare nuovamente il file. \newline All'utente viene chiesto di:
\begin{itemize}
	\item visualizzare il messaggio di alert "File JSON già caricato";
	\item cliccare su "Conferma" per sovrascrivere il file.
\end{itemize} & NI	\\

TSOF2.1.2 & UC10 &
L'utente  deve poter visualizzare il messaggio di alert del caricamento già avvenuto e annullare il caricamento. \newline All'utente viene chiesto di:
\begin{itemize}
	\item visualizzare il messaggio di alert "File JSON già caricato";
	\item cliccare su "Annulla" per tornare alla sezione di caricamento.
\end{itemize} & NI	\\

TSOF2.2 & UC2.2 &
L'utente  deve poter selezionare il file JSON. \newline All'utente viene chiesto di:
\begin{itemize}
	\item verificare che siano visibili solo file JSON;
	\item selezionare il file dalla finestra di dialogo.
\end{itemize} & I	\\

TSOF2.3 & UC2.3 &
L'utente  deve poter confermare il caricamento del file. \newline All'utente viene chiesto di:
\begin{itemize}
	\item cliccare sul pulsante "Conferma".
\end{itemize} & I	\\


TSOF2.3.1 & UC11 &
L'utente  deve poter visualizzare un messaggio d'errore in caso di problemi con il caricamento. \newline All'utente viene chiesto di:
\begin{itemize}
	\item visualizzare il messaggio d'errore "Struttura del file JSON non Supportata";
	\item cliccare il pulsante "Conferma";
	\item verificare di essere ritornato alla selezione del file.
\end{itemize} & NI	\\

TSOF2.3.2 & UC17 &
L'utente  deve poter visualizzare un messaggio di notifica di caricamento avvenuto con successo. \newline All'utente viene chiesto di:
\begin{itemize}
	\item visualizzare il messaggio di notifica "Avvenuto Successo Caricamento File JSON";
	\item cliccare il pulsante "Continua".
\end{itemize} & NI	\\
%-------------------------------------------- Olly
TSOF3 & 
UC3 &
L'utente  deve poter collegare un predittore ad un flusso. In particolare l'utente deve:
\begin{itemize}
	\item poter visualizzare la schermata di collegamento;
	\item poter selezionale il server di Grafana a cui collegarsi.
\end{itemize} &
NI \\ 

TSOF3.1 &
UC3.1 &
L'utente  deve poter selezionare il Database\glo. All'utente viene chiesto di:
\begin{itemize}
	\item verificare l'effettiva connessione al server;
	\item visualizzare la lista di Database disponibili;
	\item verificare di poter selezionare il Database desiderato.
\end{itemize}&
NI \\

TSOF3.2 &
UC3.2 &
L'utente  deve poter selezionare almeno un flusso di dati. All'utente viene chiesto di:
\begin{itemize}
	\item visualizzare le tabelle del Database;
	\item verificare di poter selezionare il flusso desiderato;
	\item verificare di poter utilizzare i dati del flusso selezionato.
\end{itemize}&
NI \\

TSOF3.3 &
UC3.3 &
L'utente  deve poter selezionare il predittore da associare al flusso. All'utente viene chiesto di:
\begin{itemize}
	\item visualizzare l'elenco dei predittori;
	\item verificare di poter selezionare il/i predittore/i desiderato/i;
	\item verificare la buona riuscita del collegamento.
\end{itemize}&
NI \\

TSOF3.4 &
UC3.4 &
L'utente  deve poter selezionare un nodo\glo del flusso. All'utente viene chiesto di:
\begin{itemize}
	\item verificare di poter selezionare il nodo desiderato;
	\item verificare di aver a disposizione il nodo desiderato.
\end{itemize}&
NI \\

TSOF3.5 &
UC3.5 &
L'utente  deve poter stabilire una o più soglie\glo al predittore. All'utente viene chiesto di:
\begin{itemize}
	\item  verificare se la funzionalità è disponibile;
	\item verificare se la soglia impostata è effettivamente quella desiderata.
\end{itemize}&
NI \\

TSOF3.5.1 &
UC12 &
L'utente  deve poter visualizzare il messaggio d'errore sulla soglia stabilita. All'utente viene chiesto di:
\begin{itemize}
	\item poter visualizzare il messaggio "Errore Impostazione Soglia Non Valida";
	\item poter cliccare il pulsante "Conferma";
	\item verificare che dopo il click sul pulsante "Conferma", sia possibile impostare la soglia.
\end{itemize} &
NI \\

TSOF3.6 &
UC3.6 &
L'utente  deve poter confermare il collegamento e vedere la lista dei collegamenti. All'utente viene chiesto di:
\begin{itemize}
	\item poter visualizzare e cliccare il pulsante etichettato "Conferma Collegamento";
	\item verificare l'effettiva conferma del collegamento;
	\item verificare la possibilità di effettuare un altro collegamento.
\end{itemize}&
NI \\

TSOF3.6.1 &
UC13 &
L'utente  deve poter visualizzare il messaggio d'errore sulle impostazioni di collegamento. All'utente viene chiesto di:
\begin{itemize}
	\item poter visualizzare il messaggio "Errore Impostazione di collegamento";
	\item poter cliccare il pulsante "Conferma";
	\item verificare che dopo il click sul pulsante "Conferma", sia possibile impostare il/i campi dato/i errato/i.
\end{itemize}&
NI \\
TSOF3.6.2 &
UC18 &
L'utente  deve poter visualizzare il messaggio di notifica per la buona riuscita del collegamento. All'utente viene chiesto di:
\begin{itemize}
	\item visualizzare il messaggio "Collegamento Avvenuto con Successo";
	\item poter visualizzare e cliccare il pulsante "Conferma".
\end{itemize} &
NI \\

TSOF3.6.3 &
UC19 &
L'utente  deve poter visualizzare l'elenco dei collegamenti. All'utente viene chiesto di:
\begin{itemize}
	\item poter visualizzare, per ogni collegamento, il predittore/i, il nodo del flusso dati e la soglia;
	\item poter visualizzare i pulsanti "Scollega Collegamento" e "Modifica Collegamento".
\end{itemize}&
NI \\

% ------------- FINE TEST 3 ------------------
% ------------- INIZIO TEST 4 ----------------

TSOF4 &
UC4 &
L'utente  deve poter scollegare il predittore. All'utente viene chiesto di:
\begin{itemize}
	\item poter visualizzare e cliccare il pulsante "Scollega Predittore";
	\item verificare l'effettiva e corretta esecuzione dello scollegamento.
\end{itemize}&
NI \\


TSOF4.1 &
UC20 &
L'utente  deve poter visualizzare il messaggio di alert in caso di scollegamento. All'utente viene chiesto di:
\begin{itemize}
	\item poter visualizzare il messaggio di alert "Procedere con lo scollegamento?";
	\item cliccare su "Conferma" se vuole procedere con lo scollegamento;
	\item cliccare su "Annulla" se non vuole scollegare il/i predittore/i;
	\item verificare che l’opzione scelta sia stata applicata.
\end{itemize}&
NI \\

% ------------- FINE TEST 4 ------------------

%-------------------------------------------- Davide		
TSOF5 & UC5 & L'utente  deve poter modificare un collegamento. \newline All'utente viene chiesto di: \begin{itemize}
\item cliccare il pulsante "Modifica collegamento";
\item verificare che la modifica sul collegamento, precedentemente effettuato, venga abilitata.
\end{itemize} & NI \\
TSOF6 & UC6 & L'utente  deve poter effettuare le operazioni di calcolo delle previsioni. \newline
All'utente viene chiesto di: \begin{itemize}
\item inserire la politica temporale\glo da applicare;
\item avviare il monitoraggio sul flusso di dati.
\end{itemize}& NI \\
TSOF6.1 & UC6.1 & L'utente  deve poter inserire la politica temporale.\newline All'utente viene chiesto di inserire: \begin{itemize}
\item il campo "Secondi";
\item il campo "Minuti";
\item il campo "Ore".
\end{itemize} & NI \\
TSOF6.1.1 & UC14 & L'utente  deve poter visualizzare il messaggio d’errore nel caso in cui la politica temporale non sia stata definita. \newline All’utente viene chiesto di:
\begin{itemize}
	\item poter visualizzare il messaggio d’errore "Errore Politica Temporale Non Definita";
	\item cliccare il pulsante "Conferma";
	\item verificare di essere ritornato all’inserimento della politica temporale.
\end{itemize}& NI \\
TSOF6.2 & UC6.2 & L'utente  deve poter avviare il monitoraggio sul flusso di dati. \newline All'utente viene chiesto di: \begin{itemize}
\item cliccare il pulsante "Avvia Monitoraggio".
\end{itemize} & NI \\
TSOF6.2.1 & UC15 & L'utente  deve poter visualizzare il messaggio d’errore nel caso in cui nessun predittore sia stato collegato. \newline All’utente viene chiesto di:
\begin{itemize}
	\item poter visualizzare il messaggio d’errore "Nessun Predittore Collegato";
	\item cliccare il pulsante "Conferma";
	\item verificare di essere ritornato all’impostazione di collegamento del predittore al flusso dati.
\end{itemize}& NI \\
TSOF6.2.2 & UC21 & L'utente  deve poter visualizzare il messaggio di notifica del corretto avvio del monitoraggio. \newline All'utente viene chiesto di: 
\begin{itemize}
	\item poter visualizzare il messaggio di notifica "Monitoraggio Avviato con Successo";
	\item cliccare il pulsante "Conferma".
\end{itemize} & NI \\
TSOF6.3 & UC6.3 & L'utente  deve poter salvare la previsione. \newline All'utente viene chiesto di: \begin{itemize}
\item cliccare il pulsante "Invia previsioni".
\end{itemize} & NI \\
TSOF6.3.1 & UC23 & L'utente  deve poter visualizzare il messaggio di notifica del corretto invio, e salvataggio, della previsione. \newline All'utente viene chiesto di: 
\begin{itemize}
	\item poter visualizzare il messaggio di notifica "Salvataggio Dati di Previsione Avvenuto con Successo";
	\item cliccare il pulsante "Conferma".
\end{itemize} & NI \\
TSOF7 & UC7 & L'utente  deve poter interrompere il monitoraggio. \newline All'utente viene chiesto di: \begin{itemize}
\item cliccare il pulsante "Interrompi Monitoraggio".
\end{itemize} & NI \\
TSOF7.1 & UC22 & L'utente  deve poter visualizzare il messaggio di notifica dell'interruzione del monitoraggio. \newline All'utente viene chiesto di: \begin{itemize}
\item poter visualizzare il messaggio di notifica "Monitoraggio Interrotto";
\item cliccare il pulsante "Conferma".
\end{itemize} & NI \\
TSOF8 & UC8 & L'utente  deve poter visualizzare le previsioni nella dashboard\glo. & NI \\
TSFF8.1 & UC24 & L'utente  deve poter visualizzare il messaggio di alert di avvenuto raggiungimento della soglia critica. \newline Per poter proseguire, all'utente viene chiesto di: \begin{itemize}
\item poter visualizzare il messaggio di alert "Soglia Critica Raggiunta";
\item cliccare il pulsante "Conferma".
\end{itemize}& NI \\


\end{longtable}