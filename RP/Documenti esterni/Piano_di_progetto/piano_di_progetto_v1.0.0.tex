\documentclass[a4paper, oneside, openany, dvipsnames, table, 12pt]{article}
\usepackage{../Template/AFKstyle}
\usepackage{hyperref}
\usepackage{verbatim} %per commenti di più righe \begin{comment} \end{comment}
\usepackage{amsmath}
\newcommand{\Titolo}{Piano di Progetto}

\newcommand{\Gruppo}{TeamAFK}

\newcommand{\Redattori}{Simone Meneghin \newline Davide Zilio}

\newcommand{\Verificatori}{Victor Dutca}

\newcommand{\pathimg}{../../../Template/img/logoAFK.png}

\newcommand{\Approvatore}{Alessandro Canesso}

\newcommand{\Distribuzione}{Prof. Vardanega Tullio \newline Prof. Cardin Riccardo \newline Gruppo AFK}

\newcommand{\Uso}{Esterno}

\newcommand{\NomeProgetto}{"Predire in Grafana"}

\newcommand{\Mail}{gruppoafk15@gmail.com}

\newcommand{\Versionedoc}{4.0.0}

\newcommand{\DescrizioneDoc}{Descrizione della pianificazione delle attività del gruppo \textit{TeamAFK} nella realizzazione del progetto \textit{Predire in Grafana}.}


\makeindex

\begin{document}
\copertina{}

%------------------ COLORI TABELLE 
\definecolor{pari}{RGB}{255, 207, 158} %{HTML}{E1F5FE} %azzurrino
\definecolor{dispari}{HTML}{FAFAFA} %bianco/grigetto 

%definizione colori per tabelle (tranne copertina)
\definecolor{redafk}{RGB}{255, 133, 51}
\definecolor{grey2}{RGB}{204, 204, 204}
\definecolor{greyRowafk}{RGB}{234, 234, 234}
\definecolor{lastrowcolor}{RGB}{176, 196, 222} %steel blue %{255,165,0} orange %{RGB}{255, 207, 158}
\rowcolors{2}{pari}{dispari}
\renewcommand{\arraystretch}{1.5}

%------------------

\newpage
\section*{Registro delle modifiche}
{
	\centering
	\begin{longtable}{ c c C{4cm}  C{4cm}  C{3cm} }
		\rowcolor{redafk}
		\textcolor{white}{\textbf{Versione}} & \textcolor{white}{\textbf{Data}} & \textcolor{white}{\textbf{Descrizione}} & \textcolor{white}{\textbf{Nominativo}} & \textcolor{white}{\textbf{Ruolo}}\\		
		1.0.0 & 2020-07-08 & Approvazione documento & Olivier Utshudi &\RdP{}\\		
		0.1.0 & 2020-07-08 & Stesura e verifica documento. & Davide Zilio \newline Olivier Utshudi &\reda{} \newline \ver{} \\		
		
	\end{longtable}

}

%Didascalia tabelle/immagini (prendono come riferimento la subsection)
\counterwithin{table}{subsection}
\counterwithin{figure}{subsection}
\newpage

%indice, indice figure e indice tabelle
\tableofcontents
\newpage
\listoffigures
\newpage
\listoftables
\newpage

\begin{comment}

ATTENZIONE
1) all'inizio e fino a metà maggio tutti hanno più tempo = TUTTI fanno le ore necessarie (base)	
2) le nuove distribuzioni verranno attuate dal 18/05 per le fasi più concitanti del progetto, ossia quelle di:	
- progettazione dettaglio e codifica		
- valutazione e collaudo	
contando che ci saranno da dare tutti, o quasi, gli arretrati da giugno in poi.

NUOVE DISTRIBUZIONI
x = esami arretrati

1) x > 3: -2 ore
2) 2 <= x <= 3: ore base definite per ogni fase
2a) caso Olivier: +1 ora (perchè hai da dare 2(+1) esami (calcolo + RO(+ TW proj)) ma essi sono comunque "meno complicati" di p2 proj e uno tra P3 e algo quindi il tempo da dedicarci è "minore")
3) x < 2: +2 ore

\end{comment}


\section{Introduzione}

\subsection{Premessa}
Per stabilire le varie attività, il gruppo si è basato sui processi, sui bisogni, e sui vincoli di dipendenza che intervengono nel progetto. In questo modo è stato possibile stabilire per ciascuna attività, il tempo e le persone da impiegare visto che sono risorse fondamentali per la vita di qualunque progetto.

\subsection{Scopo del documento}
Lo scopo del documento è quello di definire le attività da svolgere nel progetto, e di collocarle in una linea temporale.\\
Nello specifico il documento è così strutturato:
\begin{itemize}
\item analisi dei rischi;
\item descrizione modello di sviluppo;
\item collocazione membri nelle attività;
\item stima delle risorse per lo sviluppo del progetto.
\end{itemize}
\subsection{Scopo del prodotto}
Lo scopo del prodotto è quello di realizzare due plug-in per il software Grafana\glo, che permettano di monitorare e predire lo stato di un sistema in analisi. Grazie alle predizioni sarà possibile attivare degli allarmi così da poter gestire preventivamente eventuali situazioni di rischio. \\
I due plug-in\glo utilizzeranno la Support Vector Machine\glo (SVM) per poter effettuare regressione lineare o categorizzazione sui dati forniti.
\begin{comment}
I due plug-in\glo utilizzeranno la Support Vector Machine\glo (SVM) o la Regressione Lineare per classificazione o regressione sui dati forniti.
\end{comment}

\subsection{Glossario}
Per evitare ambiguità nei documenti formali, viene fornito il documento \textit{Glossario}, contenente tutti i termini considerati di difficile comprensione. Perciò nella documentazione fornita, ogni vocabolo contenuto in Glossario è contrassegnato dalla lettera G a pedice.

\subsection{Riferimenti}
\subsubsection{Riferimenti normativi}
\begin{itemize}
	\item Norme di Progetto: \textit{Norme\_di\_Progetto\_v1.0.0}.
\end{itemize}
\subsubsection{Riferimenti informativi}
\begin{itemize}
	\item Capitolato d'appalto C4: \url{https://www.math.unipd.it/~tullio/IS-1/2019/Progetto/C4.pdf}.
	\item \textbf{Slide L06 del corso Ingegneria del Software - Gestione di Progetto}: \\
	\url{https://www.math.unipd.it/~tullio/IS-1/2019/Dispense/L06.pdf};
	\item Ingegneria del Software - Ian Sommerville - 10\textsuperscript{a} Edizione.
\end{itemize}
\subsection{Scadenze}
Il gruppo \textit{TeamAFK} si impegna a presentare il proprio materiale nei seguenti appuntamenti:\\
\begin{itemize}
\item \textbf{Revisione dei Requisiti}: 2020-04-20;
\item \textbf{Revisione di Progettazione}: 2020-05-18;
\item \textbf{Revisione di Qualifica}: 2020-06-18;
\item \textbf{Revisione di Accettazione}: 2020-07-13. 
\end{itemize}

\pagebreak

\section{Gestione dei rischi}
I rischio viene inteso come l'evento che non vorremmo accadesse nel corso di un progetto, in quanto influenzerebbe in maniera negativa sulla qualità, o sulla riuscita stessa del prodotto. Inoltre, essendo un evento che può riguardare qualunque aspetto del progetto, la gestione dei rischi risulta fondamentale per la riuscita dello stesso. Per questo motivo il gruppo intende affrontare questo compito nel seguente modo:\\
\begin{itemize}
\item \textbf{Identificazione dei rischi}: vengono identificati i rischi, distinguendoli in rischi per il progetto, il prodotto e l'azienda;
\item \textbf{Analisi dei rischi}: viene valutata la probabilità dell'evento e la sua pericolosità;
\item \textbf{Pianificazione dei rischi}: viene stabilito un piano per la prevenzione del rischio annullandone gli effetti, quando possibile, o per lo meno mitigarne le conseguenze;
\item \textbf{Monitoraggio dei rischi}: ad ogni ridefinizione del \textit{Piano di Progetto}, i rischi vengono nuovamente controllati sulla base delle nuove informazioni.
\end{itemize}

\begin{longtable}{C{3cm} L{4.5cm} L{4.5cm} C{3.15cm}}
\rowcolor{white}\caption{Tabella dei rischi} \\
		\rowcolor{redafk}
\textcolor{white}{\textbf{Codice-Nome}} &
\textcolor{white}{\textbf{Descrizione}} &
\textcolor{white}{\textbf{Rilevamento}} &
\textcolor{white}{\textbf{Grado}}  \\
		\endfirsthead
		\rowcolor{white}\caption[]{(continua)} \\
		\rowcolor{redafk}
\textcolor{white}{\textbf{Codice-Nome}} &
\textcolor{white}{\textbf{Descrizione}} &
\textcolor{white}{\textbf{Rilevamento}} &
\textcolor{white}{\textbf{Grado}} \\
		\endhead
		
RiO40 - Emergenza sanitaria &
Un'epidemia riscontrata nel territorio, può costringere le autorità a porre restrizioni per ridurne l'espansione. &
Le restrizioni descritte dal DCPM 2020-03-08 permettono le sole interazioni telematiche tra gli stakeholders. & 
Probabilità: 2 
Pericolosità: 2 \\

Piano di contingenza &
\multicolumn{3}{L{13cm}}{Gli stakeholders dovranno decidere di utilizzare gli strumenti di comunicazione disponibili a tutti che limitino i disagi scaturiti dalle suddette restrizioni.} \\

RiT41 - Inesperienza Tecnologica &
Molte delle tecnologie adottate per lo sviluppo del progetto sono nuove per i componenti, che potrebbero usarle in modo non ottimale. &
Il \textit{Responsabile} ha il compito di essere al corrente delle conoscenze dei componenti. & 
Probabilità: 
2 
Pericolosità: 
2\\ 

Piano di contingenza &
\multicolumn{3}{L{13cm}}{Il \textit{Responsabile} una volta messo al corrente delle  conoscenze dei componenti, affiderà loro i ruoli che più li competono.} \\

RiO32 - Calcolo dei costi &
L'insesperienza del gruppo può portare alla sottovalutazione dei costi da sostenere. &
Il \textit{Responsabile} ha il compito di essere al corrente delle conoscenze dei componenti. & 
Probabilità: 
1 
Pericolosità: 
2\\ 

Piano di contingenza &
\multicolumn{3}{L{13cm}}{È consigliato comunicare tempestivamente al committente la variazione dei costi.} \\

RiO33 - Impegni accademici &
Essendo questo un progetto universitario, è probabile che in corso d'opera i componenti debbano sostenere attività accademiche che li sottrarrebbero dagli impegni di progetto. &
Ogni componente deve saper comunicare con chiarezza quelli che sono i propri impegni accademici. & 
Probabilità: 
2
Pericolosità: 
1 \\ 

Piano di contingenza &
\multicolumn{3}{L{13cm}}{È consigliato comunicare tempestivamente al \textit{Responsabile} i propri impegni accademici.} \\

RiO34 - Impegni personali &
\'E possibile che in corso d'opera i componeti debbano sostenere attività che li sottrarrebbero, dagli impegni di progetto. &
Ogni componente deve saper comunicare con chiarezza nel calendario quelli che sono i propri impegni. & 
Probabilità: 
2
Pericolosità: 
1 \\ 

Piano di contingenza &
\multicolumn{3}{L{13cm}}{È consigliato comunicare tempestivamente al \textit{Responsabile} i propri impegni.} \\


RiO15 - Ritardi &
Le problematiche sopracitate possono comportare ritardi non indifferenti ai fini di progetto. &
Per questo l'incaricato dell'attività deve comunicare tempestivamente il ritardo. & 
Probabilità: 
1
Pericolosità: 
0 \\ 

Piano di contingenza &
\multicolumn{3}{L{13cm}}{È consigliato riassegnare risorse laddove ce ne sia bisogno, e quindi risolvere il motivo del ritardo.} \\

RiI26 - Comunicazione interna &
Può essere che in determinati momenti un elemento del gruppo non sia raggiungibile. &
I membri del gruppo devono segnalare la momentanea assenza dell'interessato/a. & 
Probabilità: 
0
Pericolosità: 
2 \\ 

Piano di contingenza &
\multicolumn{3}{L{13cm}}{Il gruppo ha adottato diversi mezzi di comunicazione.} \\

RiI26 - Comunicazione esterna &
Se si presentano problematiche come RO40, il proponente potrebbe non sempre essere reperibile. &
I membri del gruppo organizzeranno le conferenze con il proponente con più largo anticipo. & 
Probabilità: 
0
Pericolosità: 
2 \\ 

Piano di contingenza &
\multicolumn{3}{L{13cm}}{Il gruppo ha adottato diversi mezzi di comunicazione per rimanere in contatto con il proponente.} \\

RiI37 - Contrasti interni &
Essendo l'attività di progetto un lavoro collaborativo, è possibile che i membri abbiano opinioni divergenti riguardo a determinate tematiche. &
Ciascun membro del team si impegnerà a limitare tali tensioni e fare in modo che esse non influiscano sul normale svolgersi delle attività. & 
Probabilità: 
0
Pericolosità: 
2 \\ 

Piano di contingenza &
\multicolumn{3}{L{13cm}}{Il responsabile avrà la funzione di gestire e fare da mediatore in tali divergente.} \\

\end{longtable}


\pagebreak

\section{Modello di sviluppo}
Il modello di sviluppo adottato dal gruppo è il \textbf{modello incrementale}.
\subsection{Modello incrementale}
Il modello di sviluppo incrementale vede il progetto come una serie di rilasci (interni e/o esterni), cosiché ad ogni scadenza, il materiale consegnato sia sempre più vicino al prodotto finale.
Questo approccio di sviluppo, vede la specifica del software, la sua implementazione, convalida ed evoluzione, come attività intrecciate tra loro e da sviluppare in parallelo. Quindi il prodotto è considerato tale solo all'ultimo rilascio. Motivo per cui, si relaziona bene con il versionamento adottato per il sistema.
L'adozione dello sviluppo incrementale porta i seguti vantaggi:
\begin{itemize}
\item costi ridotti di implementazione;
\item facilità nell'ottenere feedback;
\item possibilità di consegnare prototipi.
\end{itemize}
Svantaggi del modello incrementale:
\begin{itemize}
\item il processo non è visibile, e il manager deve richiedere consegne frequenti e regolari;
\item inclinazione alla degradazione del sistema, ovvero, la difficoltà di aggiungere funzionalità al sistema in un rilascio successivo, dopo averne integrata un'altra nella consegna attuale. Ad ogni incremento aumenta la complessità del codice e di conseguenza dei costi. È possibile rimediare tramite refactoring, anche se quest'ultimo muta il modello di sviluppo da incrementale a iterativo.
\end{itemize}
\begin{figure}[H]
	\centering
	\includegraphics[width=0.70\linewidth]{img/incremental_development.png}
	\caption{Modello di sviluppo incrementale}
\end{figure}

\pagebreak

\section{Pianificazione}
Sulla base delle scadenze fissate in §1.6, la ripartizione delle attività di progetto avviene tramite:
\begin{itemize}
\item \textbf{Analisi};
\item \textbf{Consolidamento requisiti};
\item \textbf{Progettazione e codifica per la Technology Baseline};
\item \textbf{Progettazione di dettaglio e codifica};
\item \textbf{Validazione e collaudo}.
\end{itemize}  
Le attività presenti sono una mera descrizione di ciò che bisognerà fare in ogni diversa fase del progetto, pertanto non vengono indicate nella \S 5. Quest'ultima si concentra sulla pianificazione generale dell'intero periodo, senza definire ogni attività, e sugli incrementi apportati al prodotto software.

\subsection{Analisi}
\textit{Periodo: da 2020-03-16 a 2020-04-13} \\
Durante questo periodo il \textit{TeamAFK} si occuperà principalmente dell’analisi di tutte le informazioni riguardanti il prodotto che deve sviluppare, l’organizzazione delle attività e la suddivisione delle risorse.

\subsubsection{Ruoli attivi} 
\begin{itemize}
\item \textit{Responsabile di Progetto};
\item \textit{Amministratore};
\item \textit{Analista};
\item \textit{Progettista};
\item \textit{Verificatore}.
\end{itemize}

\subsubsection{Attività}
\begin{itemize}
\item \textbf{Identificazione degli strumenti}: attività rivolta a determinare gli strumenti da utilizzare per le comunicazioni, stesura dei documenti, versionamento, sviluppo e verifica del sistema;
\item \textbf{Norme di Progetto}: sono l'insieme delle regole da seguire per lo svolgimento dei processi e la realizzazione del prodotto. Il documento \textit{Norme di Progetto} è redatto dall'\textit{Amministratore};
\item \textbf{Studio di Fattibilità}: attività svolta dagli \textit{Analisti} con lo scopo di analizzare i capitolati in linea generale per stabilire quale di essi sia una proposta realizzabile. Inoltre è un'attività propedeutica all'\textit{Analisi dei Requisiti};
\item \textbf{Analisi dei Requisiti}: sulla base dell'attività precedente, vengono identificati e definiti i requisiti del sistema. Come per il documento \textit{Studio di Fattibilità}, anche \textit{Analisi dei Requisiti} viene redatto dagli \textit{Analisti};
\item \textbf{Piano di Qualifica}: attività dell'\textit{Amministratore} e del \textit{Progettista} che si occupa di stabilire le metodologie per garantire la qualità del prodotto. In particolar modo la seconda figura si focalizza sulla parte programmatica;
\item \textbf{Piano di Progetto}: il lavoro da svolgere viene suddiviso in compiti, risorse e attività da parte del \textit{Responsabile} che ha anche il compito di calcolare il preventivo di periodo del progetto. Il tutto viene riportato sempre da parte del \textit{Responsabile} nel documento \textit{Piano di Progetto};
\item \textbf{Glossario}: tutti i vocaboli di difficile interpretazione vengono individuati e riportati nel documento \textit{Glossario}.
\end{itemize}

\begin{figure}[H]
\centering
\includegraphics[scale=0.24]{./img/gantt/analisi.png}
\caption{Diagramma di Gantt della fase di Analisi}
\end{figure}

\subsection{Consolidamento dei requisiti}
\textit{Periodo: da 2020-04-14 a 2020-04-20}\\
La fase di consolidamento è così suddivisa:
\begin{itemize}
\item \textbf{Approfondimento personale}: attività intenta a fissare ed approfondire le informazioni riguardanti i requisiti evidenziati nella precedente fase;
\item \textbf{Raccolta informazioni}: raccolta delle informazioni necessarie per la presentazione;
\item \textbf{Stesura presentazione}: preparazione del materiale necessario alla presentazione del 2020-04-20;
\item \textbf{Studio personale}: tempo dedicato ai membri del gruppo, per studiare le informazioni contenute nella presentazione.
\end{itemize}

\begin{figure}[H]
\centering
\includegraphics[scale=0.24]{./img/gantt/consolidamento_requisiti.png}
\caption{Diagramma di Gantt della fase di Consolidamento dei requisiti}
\end{figure}

\subsection{Progettazione e codifica per la Technology Baseline}
\textit{Periodo: da 2020-04-21 a 2020-05-11}\\
Questa fase coincide con il giorno successivo alla presentazione del 2020-04-20 e termina con la consegna del materiale per la \textbf{Revisone di Progettazione}.

\subsubsection{Ruoli attivi} \begin{itemize}
\item \textit{Responsabile di Progetto};
\item \textit{Amministratore};
\item \textit{Analista};
\item \textit{Progettista};
\item \textit{Programmatore};
\item \textit{Verificatore}.
\end{itemize}

\subsubsection{Attività}
\begin{itemize}
\item \textbf{Incrementi e verifica dei documenti}: sulla base dei feedback del committente e del proponente, viene migliorato e verificato il materiale del precedente rilascio;
\item \textbf{Progettazione e codifica della Proof of Concept}: vengono identificati i design pattern\glo necessari allo sviluppo del sistema e verranno riportati nell'allegato tecnico insieme al tracciamento dei requisiti. Inoltre viene presentato, al committente e al proponente, un prototipo per mezzo di un repository\glo. In questo periodo saranno implementati solo una parte di requisiti, ovvero quelli che ricoprono le funzionalità del tool di addestramento. La scrittura del codice per lo sviluppo di tali requisiti segue le indicazioni definite nel documento \textit{Norme di Progetto};
\item \textbf{Verifica della Proof of Concept}: vengono testati e verificati tutti gli incrementi sviluppati;
\item \textbf{Preparazione della presentazione}: vengono create le slide da utilizzare per la presentazione della PoC, fissata in data 2020/05/05.
\end{itemize}

\paragraph{Incremento 1}\mbox{} \\ \mbox{} \\ 
\textit{Periodo: da 2020/04/21 a 2020/04/30}\\
L’incremento 1 prevede lo sviluppo e l’implementazione del tool di addestramento, per ottenere il file JSON contenente i predittori. Per sviluppare questo incremento verrà utilizzata la libreria di JS \texttt{JSON.stringify}. \\
Verrà quindi implementata una pagina web che consentirà di: \begin{itemize}
\item caricare il file CSV contenente i dati per l'addestramento;
\item selezionare l'algoritmo di predizione desiderato;
\begin{itemize}
\item nel caso in cui l’utente selezioni un algoritmo incompatibile con il file CSV caricato, sarà mostrato il messaggio di errore "File CSV incompatibile";
\item nel caso in cui l'utente selezioni l'algoritmo senza aver caricato il file CSV, sarà mostrato il messaggio di errore "File CSV non inserito";
\end{itemize}
\item confermare tale algoritmo, per procedere con l'addestramento;
\begin{itemize}
\item l’utente visualizza il messaggio di notifica "Addestramento avvenuto successo", in cui viene notificato che l’addestramento confermato dell’algoritmo selezionato,a partire dai dati di addestramento, è avvenuto correttamente;
\end{itemize}
\item scaricare il file JSON contenente i predittori.
\end{itemize}
Lo sviluppo di questo incremento prevede il soddisfacimento completo dei seguenti requisiti funzionali:
\begin{itemize}
\item Re1F1;
\item Re1F1.1;
\item Re1F1.2;
\item Re1F1.3;
\item Re2F1.6;
\item Re2F1.7;
\item Re1F13.
\end{itemize}
\paragraph*{Verifica}\mbox{} \\ \mbox{} \\ 
\textit{Periodo: da 2020/05/01 a 2020/05/04}\\
Durante questo periodo verranno testate e verificate tutte le funzionalità di questo incremento. Qualora dovessero presentarsi dei bug\glo, sarà compito dei programmatori risolverli nel più breve tempo possibile.

\paragraph{Incremento 2}\mbox{} \\ \mbox{} \\ 
\textit{Periodo: da 2020/04/21 a 2020/04/30}\\
L’incremento 2 prevede lo sviluppo e l’implementazione dell'algoritmo che permette l'inserimento del file JSON, contenente i predittori, nel plug-in. Verrà usato React in sinergia con gli strumenti di sviluppo di plug-in offerti dalla piattaforma Grafana. \\
Sarà quindi possibile:
\begin{itemize}
	\item selezionare il file json da caricare, presente nel file system;
	\item confermare la selezione.
\end{itemize}
Lo sviluppo di questo incremento prevede il soddisfacimento completo dei seguenti requisiti funzionali:
\begin{itemize}
\item Re1F2;
\item Re1F2.1;
\item Re1F2.2;
\item Re1F2.4.
\end{itemize}
\paragraph*{Verifica}\mbox{} \\ \mbox{} \\ 
\textit{Periodo: da 2020/05/01 a 2020/05/03}\\
Durante questo periodo verranno testate e verificate tutte le funzionalità di questo incremento. Qualora dovessero presentarsi dei bug, sarà compito dei programmatori risolverli nel più breve tempo possibile.

\paragraph{Incremento 3}\mbox{} \\ \mbox{} \\ 
\textit{Periodo: da 2020/04/21 a 2020/04/30}\\
L’incremento 3 prevede lo sviluppo e l’implementazione dell'algoritmo che permette il collegamento del plug-in ad un flusso di dati. Verrà usato React in sinergia con gli strumenti di sviluppo di plug-in offerti dalla piattaforma Grafana. \\
Sarà quindi possibile:
\begin{itemize}
	\item selezionare i predittori;
	\item selezionare la query legata al flusso dei dati.
\end{itemize}
Lo sviluppo di questo incremento prevede il soddisfacimento completo dei seguenti requisiti funzionali:
\begin{itemize}
\item Re1F3.1;
\item Re1F3.2.
\end{itemize}
\paragraph*{Verifica}\mbox{} \\ \mbox{} \\ 
\textit{Periodo: da 2020/05/01 a 2020/05/03}\\
Durante questo periodo verranno testate e verificate tutte le funzionalità di questo incremento. Qualora dovessero presentarsi dei bug, sarà compito dei programmatori risolverli nel più breve tempo possibile.

\begin{figure}[H]
\centering
\includegraphics[scale=0.24]{./img/gantt/progettazione_architetturale.png}
\caption{Diagramma di Gantt della fase di Progettazione e codifica per la Technology Baseline}
\end{figure}

\subsection{Progettazine di dettaglio e codifica}
\textit{Periodo: da 2020-05-11 a 2020-06-11}\\
Questa fase è compresa tra il giorno successivo alla presentazione del 2020-05-11 e la consegna della \textit{Revisione di Qualifica}.

\subsubsection{Ruoli attivi} \begin{itemize}
\item \textit{Responsabile di Progetto};
\item \textit{Amministratore};
\item \textit{Analista};
\item \textit{Progettista};
\item \textit{Programmatore};
\item \textit{Verificatore}.
\end{itemize}

\subsubsection{Attività}
\begin{itemize}
\item \textbf{Incrementi e verifica dei documenti}: sulla base dei feedback del committente e del proponente, viene migliorato e verificato il materiale del precedente rilascio;
\item \textbf{Codifica degli incrementi}: 
\item \textbf{Verifica degli incrementi}: 
\item \textbf{Preparazione della presentazione}: vengono create le slide da utilizzare per la presentazione della Product Baseline.
\end{itemize}

\paragraph{Incremento 4}\mbox{} \\ \mbox{} \\ 
\textit{Periodo: da 2020-05-22 a 2020-05-26}\\
L’incremento 4 prevede il completamento dello sviluppo del tool di addestramento, aggiungendo i messaggi di alert mancanti.\\
Sarà quindi possibile:
\begin{itemize}
	\item visualizzare il messaggio di addestramento avvenuto con successo;
	\item visualizzare l'errore di addestramento non riuscito;
	\item visualizzare l'errore CSV non caricato;
	\item visualizzare l'alert algoritmo non scelto. 
\end{itemize}
Lo sviluppo di questo incremento prevede il soddisfacimento completo dei seguenti requisiti funzionali:
\begin{itemize}
\item Re1F1.4;
\item Re2F1.5;
\item Re1F11;
\item Re1F12.
\end{itemize}
\paragraph*{Verifica}\mbox{} \\ \mbox{} \\ 
\textit{Periodo: da 2020-05-27 a 2020-05-28}\\
Durante questo periodo verranno testate e verificate tutte le funzionalità di questo incremento. Qualora dovessero presentarsi dei bug, sarà compito dei programmatori risolverli nel più breve tempo possibile.

\paragraph{Incremento 5}\mbox{} \\ \mbox{} \\ 
\textit{Periodo: da 2020-05-22 a 2020-05-25}\\
L’incremento 5 prevede il completamento dello sviluppo del caricamento del file json nel plug-in, aggiungendo i messaggi di alert mancanti.\\
Sarà quindi possibile:
\begin{itemize}
	\item visualizzare il messaggio di avvenuto caricamento del file json.
\end{itemize}
Lo sviluppo di questo incremento prevede il soddisfacimento completo dei seguenti requisiti funzionali:
\begin{itemize}
\item Re2F2.3.
\end{itemize}
\paragraph*{Verifica}\mbox{} \\ \mbox{} \\ 
\textit{Periodo: da 2020-05-26 a 2020-05-27}\\
Durante questo periodo verranno testate e verificate tutte le funzionalità di questo incremento. Qualora dovessero presentarsi dei bug, sarà compito dei programmatori risolverli nel più breve tempo possibile.

\paragraph{Incremento 6}\mbox{} \\ \mbox{} \\ 
\textit{Periodo: da 2020-05-23 a 2020-05-27}\\
L’incremento 6 prevede l'inserimento del collegamento e la visualizzazione della lista dei predittori collegati. \\
Sarà quindi possibile:
\begin{itemize}
	\item confermare le impostazioni di collegamento;
	\item visualizzare la lista dei predittori precedentemente collegati.
\end{itemize}
Lo sviluppo di questo incremento prevede il soddisfacimento completo dei seguenti requisiti funzionali:
\begin{itemize}
\item Re1F3.5;
\item Re1F4.
\end{itemize}
\paragraph*{Verifica}\mbox{} \\ \mbox{} \\ 
\textit{Periodo: da 2020-05-30 a 2020-05-30}\\
Durante questo periodo verranno testate e verificate tutte le funzionalità di questo incremento. Qualora dovessero presentarsi dei bug, sarà compito dei programmatori risolverli nel più breve tempo possibile.

\paragraph{Incremento 7}\mbox{} \\ \mbox{} \\ 
\textit{Periodo: da 2020-05-28 a 2020-05-29}\\
L’incremento 7 prevede il completamente del pannello di collegamento. \\
Sarà quindi possibile:
\begin{itemize}
	\item collegare il predittore al flusso dati;
	\item aggiungere il nome ad un collegamento.
\end{itemize}
Lo sviluppo di questo incremento prevede il soddisfacimento completo dei seguenti requisiti funzionali:
\begin{itemize}
\item Re1F3;
\item Re1F3.4.
\end{itemize}
\paragraph*{Verifica}\mbox{} \\ \mbox{} \\ 
\textit{Periodo: da 2020-05-30 a 2020-05-30}\\
Durante questo periodo verranno testate e verificate tutte le funzionalità di questo incremento. Qualora dovessero presentarsi dei bug, sarà compito dei programmatori risolverli nel più breve tempo possibile.

\paragraph{Incremento 8}\mbox{} \\ \mbox{} \\ 
\textit{Periodo: da 2020-05-28 a 2020-05-30}\\
L’incremento 8 prevede lo sviluppo della funzionalità di scollegamento dei predittori.\\
Sarà quindi possibile:
\begin{itemize}
	\item scollegare il/i predittore/i;
	\item visualizzare il messaggio di conferma del scollegamento;
	\item selezionare se confermare o annullare lo scollegamento;
	\item visualizzare il messaggio di avvenuto scollegamento.
\end{itemize}
Lo sviluppo di questo incremento prevede il soddisfacimento completo dei seguenti requisiti funzionali:
\begin{itemize}
\item Re1F5;
\item Re1F5.1;
\item Re1F5.2;
\item Re1F5.3;
\item Re2F5.4;
\item Re1F5.5.
\end{itemize}
\paragraph*{Verifica}\mbox{} \\ \mbox{} \\ 
\textit{Periodo: da 2020-05-31 a 2020-05-31}\\
Durante questo periodo verranno testate e verificate tutte le funzionalità di questo incremento. Qualora dovessero presentarsi dei bug, sarà compito dei programmatori risolverli nel più breve tempo possibile.

\paragraph{Incremento 9}\mbox{} \\ \mbox{} \\ 
\textit{Periodo: da 2020-05-28 a 2020-06-01}\\
L’incremento 9 prevede lo sviluppo della funzionalità di modifica dei predittori. \\
Sarà quindi possibile:
\begin{itemize}
	\item modificare il nome del collegamento;
	\item modificare il flusso di dati associato al collegamento;
	\item salvare/annullare le modifiche effettuate;
	\item visualizzare i messaggi di alert/errori dedicati.
\end{itemize}
Lo sviluppo di questo incremento prevede il soddisfacimento completo dei seguenti requisiti funzionali:
\begin{itemize}
\item Re1F6;
\item Re1F6.1;
\item Re1F6.2;
\item Re1F6.3;
\item Re2F6.4;
\item Re1F6.5.
\end{itemize}
\paragraph*{Verifica}\mbox{} \\ \mbox{} \\ 
\textit{Periodo: da 2020-06-01 a 2020-06-03}\\
Durante questo periodo verranno testate e verificate tutte le funzionalità di questo incremento. Qualora dovessero presentarsi dei bug, sarà compito dei programmatori risolverli nel più breve tempo possibile.

\paragraph{Incremento 10}\mbox{} \\ \mbox{} \\ 
\textit{Periodo: da 2020-06-01 a 2020-06-02}\\
L’incremento 10 prevede lo sviluppo della funzionalità di monitoraggio delle previsioni. \\
Sarà quindi possibile:
\begin{itemize}
	\item avviare il monitoraggio;
	\item visualizzare il messaggio di conferma di avvio corretto del monitoraggio.
\end{itemize}
Lo sviluppo di questo incremento prevede il soddisfacimento completo dei seguenti requisiti funzionali:
\begin{itemize}
\item Re1F7;
\item Re1F7.1;
\item Re2F7.2.
\end{itemize}
\paragraph*{Verifica}\mbox{} \\ \mbox{} \\ 
\textit{Periodo: da 2020-06-03 a 2020-06-03}\\
Durante questo periodo verranno testate e verificate tutte le funzionalità di questo incremento. Qualora dovessero presentarsi dei bug, sarà compito dei programmatori risolverli nel più breve tempo possibile.

\paragraph{Incremento 11}\mbox{} \\ \mbox{} \\ 
\textit{Periodo: da 2020-05-30 a 2020-05-31}\\
L’incremento 11 prevede lo sviluppo del collegamento al database per il salvataggio delle previsioni. Per poter svolgere questo incremento è necessario l'aver implementato il calcolo delle previsioni. \\
Sarà quindi possibile:
\begin{itemize}
	\item salvare le previsione nel database;
	\item selezionare il datasource;
	\item assegnare un nome alla tabella;
	\item abilitare/disabilitare il salvataggio;
	\item visualizzare il messaggio di conferma/errore dedicati.
\end{itemize}
Lo sviluppo di questo incremento prevede il soddisfacimento completo dei seguenti requisiti funzionali:
\begin{itemize}
\item Re1F8;
\item Re1F8.1;
\item Re1F8.2;
\item Re1F8.3;
\item Re2F8.4;
\item Re1F8.5;
\item Re2F8.6.
\end{itemize}
\paragraph*{Verifica}\mbox{} \\ \mbox{} \\ 
\textit{Periodo: da 2020-06-01 a 2020-06-01}\\
Durante questo periodo verranno testate e verificate tutte le funzionalità di questo incremento. Qualora dovessero presentarsi dei bug, sarà compito dei programmatori risolverli nel più breve tempo possibile.

\paragraph{Incremento 12}\mbox{} \\ \mbox{} \\ 
\textit{Periodo: da 2020-06-01 a 2020-06-03}\\
L’incremento 12 prevede la visualizzazione della dashboard nella lista dei plug-in disponibili di Grafana. \\
Sarà quindi possibile:
\begin{itemize}
	\item visualizzare la dashboard del plug-in.
\end{itemize}
Lo sviluppo di questo incremento prevede il soddisfacimento completo dei seguenti requisiti funzionali:
\begin{itemize}
\item Re1F10;
\item Re1F10.1.
\end{itemize}
\paragraph*{Verifica}\mbox{} \\ \mbox{} \\ 
\textit{Periodo: da 2020-06-04 a 2020-06-05}\\
Durante questo periodo verranno testate e verificate tutte le funzionalità di questo incremento. Qualora dovessero presentarsi dei bug, sarà compito dei programmatori risolverli nel più breve tempo possibile.

\begin{figure}[H]
\centering
\includegraphics[scale=0.24]{./img/gantt/progettazione_dettaglio_codifica.png}
\caption{Diagramma di Gantt della fase di Progettazione di dettaglio e codifica}
\end{figure}

\subsection{Validazione e collaudo}
\textit{Periodo: da 2020-06-19 a 2020-07-17}\\
La seguente fase inizia il giorno seguente la \textit{Revisione di Qualifica} e termina con la consegna del materiale richiesto per la \textit{Revisione di Accettazione}.

\subsubsection{Ruoli attivi} \begin{itemize}
\item \textit{Responsabile di Progetto};
\item \textit{Amministratore};
\item \textit{Analista};
\item \textit{Progettista};
\item \textit{Programmatore};
\item \textit{Verificatore}.
\end{itemize}

\subsubsection{Attività}
\begin{itemize}
\item \textbf{Codifica degli incrementi}:\\
\textit{Periodo: da 2020-06-22 a 2020-06-26} \\
nel caso risultasse necessario vengono effettuati miglioramenti sulla base di feedback e/o requisiti obbligatori mancanti;
\item \textbf{Validazione e collaudo}: \\
\textit{Periodo: da 2020-06-27 a 2020-07-15} \\
la validazione effettua test sul prodotto, mentre la convalidazione controlla se viene rispettata la coerenza tra il prodotto e le specifiche evidenziate nel documento \textit{Analisi dei Requisiti};
\item \textbf{Manuale Sviluppatore}: viene aggiornato ed incrementato il documento \textit{Manuale dello Sviluppatore}, il quale conterrà le informazioni necessarie allo sviluppo, mantenimento e manutenzione del prodotto;
\item \textbf{Manuale Utente}: viene aggiornato ed incrementato il documento \textit{Manuale dell'Utente}, il quale conterrà le informazioni necessarie all'utilizzo del prodotto.
\end{itemize}

\paragraph{Incremento 13}\mbox{} \\ \mbox{} \\ 
\textit{Periodo: da 2020-06-22 a 2020-06-24} \\
L'incremento 13 prevede lo sviluppo della funzionalità di interruzione del monitoraggio. Sarà quindi possibile: \begin{itemize}
\item interrompere il monitoraggio del flusso dati.
\item visualizzare i messaggi di conferma/errore dedicati.
\end{itemize}
Lo sviluppo di questo incremento prevede il soddisfacimento completo dei seguenti requisiti funzionali: \begin{itemize}
\item Re1F9;
\item Re1F9.1;
\item Re2F9.2.
\end{itemize}

\paragraph{Incremento 14}\mbox{} \\ \mbox{} \\ 
\textit{Periodo: da 2020-06-25 a 2020-06-26}\\
L'incremento 14 prevede l'inserimento di messaggi di notifica/errori mancanti nel plug-in. Sarà quindi possibile visualizzare: \begin{itemize}
\item visualizzare il messaggio d'errore "Collega tutti i predittori";
\item visualizzare il messaggio d'errore "Inserisci un nome per la connessione";
\item visualizzare il messaggio d'errore "Attenzione! Monitoraggio attivo";
\item visualizzare il messaggio d'errore "Nessun predittore collegato".
\end{itemize}
Lo sviluppo di questo incremento prevede il soddisfacimento completo dei seguenti requisiti funzionali: \begin{itemize}
\item Re1F14;
\item Re1F16;
\item Re1F17;
\item Re1F18.
\end{itemize}

\begin{figure}[H]
\centering
\includegraphics[scale=0.24]{./img/gantt/validazione_collaudo.png}
\caption{Diagramma di Gantt della fase di Validazione e collaudo}
\end{figure}
\pagebreak

\section{Preventivo}
Per facilitare la lettura delle tabelle vengono utilizzate le seguenti sigle per identificare i diversi ruoli e per ognuno di essi vengono indicati i relativi costi/h: \begin{itemize}
\item \textbf{Re}: \textit{Responsabile} 30€/h;
\item \textbf{Am}: \textit{Amministratore} 20€/h;
\item \textbf{An}: \textit{Analista} 25€/h;
\item \textbf{Pt}: \textit{Progettista} 22€/h;
\item \textbf{Pm}: \textit{Programmatore} 15€/h;
\item \textbf{Ve}: \textit{Verificatore} 15€/h.
\end{itemize}
Inoltre, se le ore ricoperte in un determinato ruolo fossero nulle, la cella presenterà il simbolo "-" per indicarne l'assenza.
 
\subsection{Periodo di Analisi}
\subsubsection{Distribuzione oraria}
In questa fase, i ruoli sono così suddivisi:
\begin{table}[H]
\centering\renewcommand{\arraystretch}{1.5}
\caption{Distribuzione delle ore nella fase di Analisi}
\vspace{0.2cm}
\begin{tabular}{ c c c c c c c c }
\rowcolor{redafk}
\textcolor{white}{\textbf{Nominativo}} & \textcolor{white}{\textbf{Re}} &
\textcolor{white}{\textbf{Am}} & \textcolor{white}{\textbf{An}} &
\textcolor{white}{\textbf{Pt}} & \textcolor{white}{\textbf{Pm}} &
\textcolor{white}{\textbf{Ve}} & \textcolor{white}{\textbf{Totale}} \\
Simone Federico Bergamin & 6 & 7 & 20 & - & - & 9 & 42 \\
Alessandro Canesso & 8 & 6 & 16 & - & - & 12 & 42 \\
Victor Dutca & 9 & - & 15 & - & - & 16 & 40 \\
Fouad Farid & 7 & 7 & 12 & 6 & - & 8 & 40 \\
Simone Meneghin & - & 8 & 14 & 10 & - & 10 & 42 \\
Olivier Utshudi & - & 8 & 13 & 8 & - & 13 & 42 \\
Davide Zilio & 4 & 5 & 17 & - & - & 14 & 40 \\
\rowcolor{lastrowcolor}
\textbf{Ore totali ruolo} & \textbf{34} & \textbf{41} & \textbf{107} & \textbf{24} & \textbf{0} & \textbf{82} & \textbf{288} \\
\end{tabular}
\end{table}
 
\pagebreak
 
I dati ottenuti sono riassunti nel seguente istogramma:
\begin{figure}[H]
\centering
\includegraphics[scale=0.60]{img/grafici/tabella_fase_analisi.png}
\caption{Istogramma della ripartizione delle ore per ruolo nella fase di Analisi}
\end{figure}
 
\subsubsection{Prospetto economico}
In questa fase il costo per ogni ruolo è il seguente:
 
%tabella costi
\begin{table}[H]
\centering\renewcommand{\arraystretch}{1.5}
\caption{Prospetto dei costi nella fase di Analisi}
\vspace{0.2cm}
\begin{tabular}{ c c c  }
\rowcolor{redafk}
\textcolor{white}{\textbf{Ruolo}} & \textcolor{white}{\textbf{Ore}} &
\textcolor{white}{\textbf{Costo}}  \\
Responsabile & 34 & 1020€ \\
Amministratore & 41 & 820€ \\
Analista & 107 & 2675€ \\
Progettista & 24 & 528€ \\
Programmatore & 0 & 0€  \\
Verificatore & 82 & 1230€  \\
\rowcolor{lastrowcolor}
\textbf{Totale} & \textbf{288} & \textbf{6273€}  \\
\end{tabular}
\end{table}
 
I dati ottenuti sono riassunti nel seguente areogramma:
\begin{figure}[H]
\centering
\includegraphics[scale=0.60]{img/grafici/torta_fase_analisi_prospetto_economico.png}
\caption{Areogramma della ripartizione dei costi per ruolo nella fase di Analisi}
\end{figure}
 
%--------------------------------------------------
 
\subsection{Periodo di Progettazione e codifica per la Technology Baseline}
\subsubsection{Distribuzione oraria}
In questa fase i ruoli sono così suddivisi:
\begin{table}[H]
\centering\renewcommand{\arraystretch}{1.5}
\caption{Distribuzione delle ore nella fase di Progettazione e codifica per la Technology Baseline}
\vspace{0.2cm}
\begin{tabular}{ c c c c c c c c }
\rowcolor{redafk}
\textcolor{white}{\textbf{Nominativo}} & \textcolor{white}{\textbf{Re}} &
\textcolor{white}{\textbf{Am}} & \textcolor{white}{\textbf{An}} &
\textcolor{white}{\textbf{Pt}} & \textcolor{white}{\textbf{Pm}} &
\textcolor{white}{\textbf{Ve}} & \textcolor{white}{\textbf{Totale}} \\
Simone Federico Bergamin & - & - & 10 & 7 & 5 & 8 & 30 \\
Alessandro Canesso & - & 5 & - & 10 & 9 & 8 & 32 \\
Victor Dutca & 3 & 6 & 4 & 10 & 7 & - & 30 \\
Fouad Farid & - & 5 & - & 14 & - & 11 & 30 \\
Simone Meneghin & 6 & - & 8 & 10 & 6 & - & 30 \\
Olivier Utshudi & - & 4 & - & 8 & 6 & 12 & 30 \\
Davide Zilio & 3 & - & 13 & - & - & 14 & 30 \\
\rowcolor{lastrowcolor}
\textbf{Ore totali ruolo} & \textbf{12} & \textbf{20} & \textbf{35} & \textbf{59} & \textbf{33} & \textbf{53} & \textbf{212} \\
\end{tabular}
\end{table}
 
I dati ottenuti sono riassunti nel seguente istogramma:
\begin{figure}[H]
\centering
\includegraphics[scale=0.60]{img/grafici/tabella_fase_prog_architetturale.png}
\caption{Istogramma della ripartizione delle ore per ruolo nella fase di Progettazione e codifica per la Technology Baseline}
\end{figure}
 
\subsubsection{Prospetto economico}
In questa fase il costo per ogni ruolo è il seguente:
 
%tabella costi
\begin{table}[H]
\centering\renewcommand{\arraystretch}{1.5}
\caption{Prospetto dei costi nella fase di Progettazione e codifica per la Technology Baseline}
\vspace{0.2cm}
\begin{tabular}{ c c c }
\rowcolor{redafk}
\textcolor{white}{\textbf{Ruolo}} & \textcolor{white}{\textbf{Ore}} &
\textcolor{white}{\textbf{Costo}}  \\
Responsabile & 12 & 360€ \\
Amministratore & 20 & 400€ \\
Analista & 35 & 875€ \\
Progettista & 59 & 1298€ \\
Programmatore & 33 & 495€  \\
Verificatore & 53 & 795€  \\
\rowcolor{lastrowcolor}
\textbf{Totale} & \textbf{212} & \textbf{4223€}  \\
\end{tabular}
\end{table}
 
I dati ottenuti sono riassunti nel seguente areogramma:
\begin{figure}[H]
\centering
\includegraphics[scale=0.60]{img/grafici/torta_fase_prog_architetturale.png}
\caption{Areogramma della ripartizione dei costi per ruolo nella fase di Progettazione e codifica per la Technology Baseline}
\end{figure}
 
%--------------------------------------------------
 
\subsection{Periodo di Progettazione di dettaglio e codifica}
\subsubsection{Distribuzione oraria}
In questa fase i ruoli sono così suddivisi:
\begin{table}[H]
\centering\renewcommand{\arraystretch}{1.5}
\caption{Distribuzione delle ore nella fase di Progettazione di dettaglio e codifica}
\vspace{0.2cm}
\begin{tabular}{ c c c c c c c c }
\rowcolor{redafk}
\textcolor{white}{\textbf{Nominativo}} & \textcolor{white}{\textbf{Re}} &
\textcolor{white}{\textbf{Am}} & \textcolor{white}{\textbf{An}} &
\textcolor{white}{\textbf{Pt}} & \textcolor{white}{\textbf{Pm}} &
\textcolor{white}{\textbf{Ve}} & \textcolor{white}{\textbf{Totale}} \\
Simone Federico Bergamin & - & 6 & - & 12 & 18 & 12 & 48 \\
Alessandro Canesso & 4 & 3 & - & 10 & 18 & 11 & 46 \\
Victor Dutca & - & 8 & - & 10 & 20 & 10 & 48 \\
Fouad Farid & 4 & - & - & 12 & 20 & 12 & 48 \\
Simone Meneghin & 2 & - & - & 12 & 22 & 14 & 50 \\
Olivier Utshudi & 8 & - & - & 8 & 22 & 12 & 50 \\
Davide Zilio & - & 6 & - & 10 & 20 & 12 & 48 \\
\rowcolor{lastrowcolor}
\textbf{Ore totali ruolo} & \textbf{18} & \textbf{23} & \textbf{0} & \textbf{74} & \textbf{140} & \textbf{83} & \textbf{338} \\
\end{tabular}
\end{table}
 
I dati ottenuti sono riassunti nel seguente istogramma:
\begin{figure}[H]
\centering
\includegraphics[scale=0.60]{img/grafici/tabella_fase_prog_cod.png}
\caption{Istogramma della ripartizione delle ore per ruolo nella fase di Progettazione di dettaglio e codifica}
\end{figure}

\paragraph{Distribuzione oraria incrementi}
\paragraph*{Incremento 4} \mbox{} \\
\begin{table}[H]
\centering\renewcommand{\arraystretch}{1.5}
\caption{Distribuzione delle ore per lo sviluppo dell'incremento 4}
\vspace{0.2cm}
\begin{tabular}{ c c c c c c c c }
\rowcolor{redafk}
\textcolor{white}{\textbf{Nominativo}} & \textcolor{white}{\textbf{Re}} &
\textcolor{white}{\textbf{Am}} & \textcolor{white}{\textbf{An}} &
\textcolor{white}{\textbf{Pt}} & \textcolor{white}{\textbf{Pm}} &
\textcolor{white}{\textbf{Ve}} & \textcolor{white}{\textbf{Totale}} \\
Simone Federico Bergamin & - & - & - & - & - & - & 0 \\
Alessandro Canesso & - & - & - & - & - & - & 0\\
Victor Dutca & - & 2 & - & - & 5 & - & 7 \\
Fouad Farid & - & - & - & - & 11 & - & 11 \\
Simone Meneghin & 2 & - & - & - & - & 6 & 8 \\
Olivier Utshudi & - & - & - & - & - & - & 0 \\
Davide Zilio & - & 2 & - & 4 & - & - & 6 \\
\rowcolor{lastrowcolor}
\textbf{Ore totali ruolo} & \textbf{2} & \textbf{4} & \textbf{0} & \textbf{4} & \textbf{16} & \textbf{6} & \textbf{32} \\
\end{tabular}
\end{table}

I dati ottenuti sono riassunti nel seguente istogramma:
\begin{figure}[H]
\centering
\includegraphics[scale=0.60]{img/grafici/tabella_inc4.png}
\caption{Istogramma della ripartizione delle ore per lo sviluppo dell'incremento 4}
\end{figure}

\paragraph*{Incremento 5}\mbox{} \\
\begin{table}[H]
\centering\renewcommand{\arraystretch}{1.5}
\caption{Distribuzione delle ore per lo sviluppo dell'incremento 5}
\vspace{0.2cm}
\begin{tabular}{ c c c c c c c c }
\rowcolor{redafk}
\textcolor{white}{\textbf{Nominativo}} & \textcolor{white}{\textbf{Re}} &
\textcolor{white}{\textbf{Am}} & \textcolor{white}{\textbf{An}} &
\textcolor{white}{\textbf{Pt}} & \textcolor{white}{\textbf{Pm}} &
\textcolor{white}{\textbf{Ve}} & \textcolor{white}{\textbf{Totale}} \\
Simone Federico Bergamin & - & - & - & - & - & - & 0 \\
Alessandro Canesso & 1 & - & - & 1 & - & 2 & 4 \\
Victor Dutca & - & 1 & - & - & - & - & 1 \\
Fouad Farid & - & - & - & - & - & - & 0 \\
Simone Meneghin & - & - & - & - & - & - & 0 \\
Olivier Utshudi & - & - & - & - & 3 & - & 3 \\
Davide Zilio & - & - & - & - & - & - & 0 \\
\rowcolor{lastrowcolor}
\textbf{Ore totali ruolo} & \textbf{1} & \textbf{1} & \textbf{0} & \textbf{1} & \textbf{3} & \textbf{2} & \textbf{8} \\
\end{tabular}
\end{table}

I dati ottenuti sono riassunti nel seguente istogramma:
\begin{figure}[H]
\centering
\includegraphics[scale=0.60]{img/grafici/tabella_inc5.png}
\caption{Istogramma della ripartizione delle ore per lo sviluppo dell'incremento 5}
\end{figure}

\paragraph*{Incremento 6}\mbox{} \\
\begin{table}[H]
\centering\renewcommand{\arraystretch}{1.5}
\caption{Distribuzione delle ore per lo sviluppo dell'incremento 6}
\vspace{0.2cm}
\begin{tabular}{ c c c c c c c c }
\rowcolor{redafk}
\textcolor{white}{\textbf{Nominativo}} & \textcolor{white}{\textbf{Re}} &
\textcolor{white}{\textbf{Am}} & \textcolor{white}{\textbf{An}} &
\textcolor{white}{\textbf{Pt}} & \textcolor{white}{\textbf{Pm}} &
\textcolor{white}{\textbf{Ve}} & \textcolor{white}{\textbf{Totale}} \\
Simone Federico Bergamin & - & 2 & - & - & 8 & - & 10 \\
Alessandro Canesso & - & - & - & - & - & 6 & 6 \\
Victor Dutca & - & 2 & - & 2 & - & - & 4 \\
Fouad Farid & - & - & - & 4 & - & - & 4 \\
Simone Meneghin & - & - & - & - & - & - & 0 \\
Olivier Utshudi & 3 & - & - & - & - & - & 3 \\
Davide Zilio & - & - & - & - & 7 & - & 7 \\
\rowcolor{lastrowcolor}
\textbf{Ore totali ruolo} & \textbf{3} & \textbf{4} & \textbf{-} & \textbf{6} & \textbf{15} & \textbf{6} & \textbf{34} \\
\end{tabular}
\end{table}

I dati ottenuti sono riassunti nel seguente istogramma:
\begin{figure}[H]
\centering
\includegraphics[scale=0.60]{img/grafici/tabella_inc6.png}
\caption{Istogramma della ripartizione delle ore per lo sviluppo dell'incremento 6}
\end{figure}

\paragraph*{Incremento 7}\mbox{} \\
\begin{table}[H]
\centering\renewcommand{\arraystretch}{1.5}
\caption{Distribuzione delle ore per lo sviluppo dell'incremento 7}
\vspace{0.2cm}
\begin{tabular}{ c c c c c c c c }
\rowcolor{redafk}
\textcolor{white}{\textbf{Nominativo}} & \textcolor{white}{\textbf{Re}} &
\textcolor{white}{\textbf{Am}} & \textcolor{white}{\textbf{An}} &
\textcolor{white}{\textbf{Pt}} & \textcolor{white}{\textbf{Pm}} &
\textcolor{white}{\textbf{Ve}} & \textcolor{white}{\textbf{Totale}} \\
Simone Federico Bergamin & - & 2 & - & - & - & - & 2 \\
Alessandro Canesso & 2 & - & - & - & - & - & 2 \\
Victor Dutca & - & - & - & - & - & - & 0 \\
Fouad Farid & - & - & - & - & - & - & 0 \\
Simone Meneghin & - & - & - & 6 & - & - & 6 \\
Olivier Utshudi & - & - & - & 4 & 6 & 4 & 8 \\
Davide Zilio & - & - & - & - & - & - & 0 \\
\rowcolor{lastrowcolor}
\textbf{Ore totali ruolo} & \textbf{2} & \textbf{2} & \textbf{0} & \textbf{10} & \textbf{6} & \textbf{4} & \textbf{24} \\
\end{tabular}
\end{table}

I dati ottenuti sono riassunti nel seguente istogramma:
\begin{figure}[H]
\centering
\includegraphics[scale=0.60]{img/grafici/tabella_inc7.png}
\caption{Istogramma della ripartizione delle ore per lo sviluppo dell'incremento 7}
\end{figure}

\paragraph*{Incremento 8}\mbox{} \\
\begin{table}[H]
\centering\renewcommand{\arraystretch}{1.5}
\caption{Distribuzione delle ore per lo sviluppo dell'incremento 8}
\vspace{0.2cm}
\begin{tabular}{ c c c c c c c c }
\rowcolor{redafk}
\textcolor{white}{\textbf{Nominativo}} & \textcolor{white}{\textbf{Re}} &
\textcolor{white}{\textbf{Am}} & \textcolor{white}{\textbf{An}} &
\textcolor{white}{\textbf{Pt}} & \textcolor{white}{\textbf{Pm}} &
\textcolor{white}{\textbf{Ve}} & \textcolor{white}{\textbf{Totale}} \\
Simone Federico Bergamin & - & - & - & 4 & - & - & 4 \\
Alessandro Canesso & - & - & - & - & 6 & - & 6 \\
Victor Dutca & - & - & - & - & - & 6 & 6 \\
Fouad Farid & - & - & - & - & - & 5 & 5 \\
Simone Meneghin & - & - & - & - & 8 & - & 8 \\
Olivier Utshudi & 3 & - & - & 4 & - & - & 7 \\
Davide Zilio & - & 4 & - & - & 6 & - & 10 \\
\rowcolor{lastrowcolor}
\textbf{Ore totali ruolo} & \textbf{3} & \textbf{4} & \textbf{0} & \textbf{8} & \textbf{20} & \textbf{11} & \textbf{46} \\
\end{tabular}
\end{table}

I dati ottenuti sono riassunti nel seguente istogramma:
\begin{figure}[H]
\centering
\includegraphics[scale=0.60]{img/grafici/tabella_inc8.png}
\caption{Istogramma della ripartizione delle ore per lo sviluppo dell'incremento 8}
\end{figure}

\paragraph*{Incremento 9}\mbox{} \\
\begin{table}[H]
\centering\renewcommand{\arraystretch}{1.5}
\caption{Distribuzione delle ore per lo sviluppo dell'incremento 9}
\vspace{0.2cm}
\begin{tabular}{ c c c c c c c c }
\rowcolor{redafk}
\textcolor{white}{\textbf{Nominativo}} & \textcolor{white}{\textbf{Re}} &
\textcolor{white}{\textbf{Am}} & \textcolor{white}{\textbf{An}} &
\textcolor{white}{\textbf{Pt}} & \textcolor{white}{\textbf{Pm}} &
\textcolor{white}{\textbf{Ve}} & \textcolor{white}{\textbf{Totale}} \\
Simone Federico Bergamin & - & 2 & - & 4 & 8 & - & 14\\
Alessandro Canesso & - & 2 & - & 5 & - & - & 7 \\
Victor Dutca & - & - & - & - & 10 & - & 10 \\
Fouad Farid & - & - & - & - & 9 & - & 9 \\
Simone Meneghin & - & - & - & - & - & 8 & 8 \\
Olivier Utshudi & 2 & - & - & - & - & 8 & 10 \\
Davide Zilio & - & - & - & 6 & - & 4 & 10 \\
\rowcolor{lastrowcolor}
\textbf{Ore totali ruolo} & \textbf{2} & \textbf{4} & \textbf{0} & \textbf{15} & \textbf{27} & \textbf{20} & \textbf{68} \\
\end{tabular}
\end{table}

I dati ottenuti sono riassunti nel seguente istogramma:
\begin{figure}[H]
\centering
\includegraphics[scale=0.60]{img/grafici/tabella_inc9.png}
\caption{Istogramma della ripartizione delle ore per lo sviluppo dell'incremento 9}
\end{figure}

\paragraph*{Incremento 10}\mbox{} \\
\begin{table}[H]
\centering\renewcommand{\arraystretch}{1.5}
\caption{Distribuzione delle ore per lo sviluppo dell'incremento 10}
\vspace{0.2cm}
\begin{tabular}{ c c c c c c c c }
\rowcolor{redafk}
\textcolor{white}{\textbf{Nominativo}} & \textcolor{white}{\textbf{Re}} &
\textcolor{white}{\textbf{Am}} & \textcolor{white}{\textbf{An}} &
\textcolor{white}{\textbf{Pt}} & \textcolor{white}{\textbf{Pm}} &
\textcolor{white}{\textbf{Ve}} & \textcolor{white}{\textbf{Totale}} \\
Simone Federico Bergamin & - & - & - & - & 2 & 5 & 7 \\
Alessandro Canesso & 1 & 1 & - & 4 & - & - & 6 \\
Victor Dutca & - & - & - & - & 5 & 4 & 9 \\
Fouad Farid & - & - & - & 3 & - & - & 3  \\
Simone Meneghin & - & - & - & - & 4 & - & 4 \\
Olivier Utshudi & - & - & - & - & - & - & 0 \\
Davide Zilio & - & - & - & - & - & - & 0 \\
\rowcolor{lastrowcolor}
\textbf{Ore totali ruolo} & \textbf{1} & \textbf{1} & \textbf{0} & \textbf{7} & \textbf{11} & \textbf{9} & \textbf{29} \\
\end{tabular}
\end{table}

I dati ottenuti sono riassunti nel seguente istogramma:
\begin{figure}[H]
\centering
\includegraphics[scale=0.60]{img/grafici/tabella_inc10.png}
\caption{Istogramma della ripartizione delle ore per lo sviluppo dell'incremento 10}
\end{figure}

\paragraph*{Incremento 11}\mbox{} \\
\begin{table}[H]
\centering\renewcommand{\arraystretch}{1.5}
\caption{Distribuzione delle ore per lo sviluppo dell'incremento 11}
\vspace{0.2cm}
\begin{tabular}{ c c c c c c c c }
\rowcolor{redafk}
\textcolor{white}{\textbf{Nominativo}} & \textcolor{white}{\textbf{Re}} &
\textcolor{white}{\textbf{Am}} & \textcolor{white}{\textbf{An}} &
\textcolor{white}{\textbf{Pt}} & \textcolor{white}{\textbf{Pm}} &
\textcolor{white}{\textbf{Ve}} & \textcolor{white}{\textbf{Totale}} \\
Simone Federico Bergamin & - & - & - & - & - & 3 & 3\\
Alessandro Canesso & - & - & - & - & 6 & - & 6 \\
Victor Dutca & - & 3 & - & 5 & - & - & 8 \\
Fouad Farid & 2 & - & - & - & - & 7 & 9 \\
Simone Meneghin & - & - & - & 6 & - & - & 6 \\
Olivier Utshudi & - & - & - & - & 5 & - & 5 \\
Davide Zilio & - & - & - & - & 7 & - & 7 \\
\rowcolor{lastrowcolor}
\textbf{Ore totali ruolo} & \textbf{2} & \textbf{3} & \textbf{0} & \textbf{11} & \textbf{18} & \textbf{10} & \textbf{44} \\
\end{tabular}
\end{table}

I dati ottenuti sono riassunti nel seguente istogramma:
\begin{figure}[H]
\centering
\includegraphics[scale=0.60]{img/grafici/tabella_inc11.png}
\caption{Istogramma della ripartizione delle ore per lo sviluppo dell'incremento 11}
\end{figure}

\paragraph*{Incremento 12}\mbox{} \\
\begin{table}[H]
\centering\renewcommand{\arraystretch}{1.5}
\caption{Distribuzione delle ore per lo sviluppo dell'incremento 12}
\vspace{0.2cm}
\begin{tabular}{ c c c c c c c c }
\rowcolor{redafk}
\textcolor{white}{\textbf{Nominativo}} & \textcolor{white}{\textbf{Re}} &
\textcolor{white}{\textbf{Am}} & \textcolor{white}{\textbf{An}} &
\textcolor{white}{\textbf{Pt}} & \textcolor{white}{\textbf{Pm}} &
\textcolor{white}{\textbf{Ve}} & \textcolor{white}{\textbf{Totale}} \\
Simone Federico Bergamin & - & - & - & 4 & - & 4 & 8 \\
Alessandro Canesso & - & - & - & - & 6 & 3 & 9 \\
Victor Dutca & - & - & - & 3 & - & - & 3 \\
Fouad Farid & 2 & - & - & 5 & - & - & 7 \\
Simone Meneghin & - & - & - & - & 10 & - & 10 \\
Olivier Utshudi & - & - & - & - & 8 & - & 8 \\
Davide Zilio & - & - & - & - & - & 8 & 8 \\
\rowcolor{lastrowcolor}
\textbf{Ore totali ruolo} & \textbf{2} & \textbf{0} & \textbf{0} & \textbf{12} & \textbf{24} & \textbf{15} & \textbf{53} \\
\end{tabular}
\end{table}

I dati ottenuti sono riassunti nel seguente istogramma:
\begin{figure}[H]
\centering
\includegraphics[scale=0.60]{img/grafici/tabella_inc12.png}
\caption{Istogramma della ripartizione delle ore per lo sviluppo dell'incremento 12}
\end{figure}

\subsubsection{Prospetto economico}
In questa fase il costo per ogni ruolo è il seguente:
 
%tabella costi
\begin{table}[H]
\centering\renewcommand{\arraystretch}{1.5}
\caption{Prospetto dei costi nella fase di Progettazione di dettaglio e codifica}
\vspace{0.2cm}
\begin{tabular}{ c c c }
\rowcolor{redafk}
\textcolor{white}{\textbf{Ruolo}} & \textcolor{white}{\textbf{Ore}} &
\textcolor{white}{\textbf{Costo}}  \\
Responsabile & 18 & 540€ \\
Amministratore & 23 & 460€ \\
Analista & 0 & 0€ \\
Progettista & 74 & 1628€ \\
Programmatore & 140 & 2100€  \\
Verificatore & 83 & 1245€  \\
\rowcolor{lastrowcolor}
\textbf{Totale} & \textbf{338} & \textbf{5973€}  \\
\end{tabular}
\end{table}
 
I dati ottenuti sono riassunti nel seguente areogramma:
\begin{figure}[H]
\centering
\includegraphics[scale=0.60]{img/grafici/torta_fase_prog_cod.png}
\caption{Areogramma della ripartizione dei costi per ruolo nella fase di Progettazione di dettaglio e codifica}
\end{figure}

\paragraph{Prospetto economico incrementi}
\paragraph*{Incremento 4}\mbox{} \\
\begin{table}[H]
\centering\renewcommand{\arraystretch}{1.5}
\caption{Prospetto dei costi per lo sviluppo dell'incremento 4}
\vspace{0.2cm}
\begin{tabular}{ c c c }
\rowcolor{redafk}
\textcolor{white}{\textbf{Ruolo}} & \textcolor{white}{\textbf{Ore}} &
\textcolor{white}{\textbf{Costo}}  \\
Responsabile & 2 & 60€ \\
Amministratore & 4 & 80€ \\
Analista & 0 & 0€ \\
Progettista & 4 & 88€ \\
Programmatore & 16 & 240€  \\
Verificatore & 6 & 90€  \\
\rowcolor{lastrowcolor}
\textbf{Totale} & \textbf{32} & \textbf{558€}  \\
\end{tabular}
\end{table}
 
I dati ottenuti sono riassunti nel seguente areogramma:
\begin{figure}[H]
\centering
\includegraphics[scale=0.60]{img/grafici/torta_inc4.png}
\caption{Areogramma della ripartizione dei costi per ruolo per lo sviluppo dell'incremento 4}
\end{figure}

\paragraph*{Incremento 5}\mbox{} \\
\begin{table}[H]
\centering\renewcommand{\arraystretch}{1.5}
\caption{Prospetto dei costi per lo sviluppo dell'incremento 5}
\vspace{0.2cm}
\begin{tabular}{ c c c }
\rowcolor{redafk}
\textcolor{white}{\textbf{Ruolo}} & \textcolor{white}{\textbf{Ore}} &
\textcolor{white}{\textbf{Costo}}  \\
Responsabile & 1 & 30€ \\
Amministratore & 1 & 20€ \\
Analista & 0 & 0€ \\
Progettista & 1 & 22€ \\
Programmatore & 3 & 45€  \\
Verificatore & 2 & 30€  \\
\rowcolor{lastrowcolor}
\textbf{Totale} & \textbf{8} & \textbf{147€}  \\
\end{tabular}
\end{table}
 
I dati ottenuti sono riassunti nel seguente areogramma:
\begin{figure}[H]
\centering
\includegraphics[scale=0.60]{img/grafici/torta_inc5.png}
\caption{Areogramma della ripartizione dei costi per ruolo per lo sviluppo dell'incremento 5}
\end{figure}

\paragraph*{Incremento 6}\mbox{} \\
\begin{table}[H]
\centering\renewcommand{\arraystretch}{1.5}
\caption{Prospetto dei costi per lo sviluppo dell'incremento 6}
\vspace{0.2cm}
\begin{tabular}{ c c c }
\rowcolor{redafk}
\textcolor{white}{\textbf{Ruolo}} & \textcolor{white}{\textbf{Ore}} &
\textcolor{white}{\textbf{Costo}}  \\
Responsabile & 3 & 90€ \\
Amministratore & 4 & 80€ \\
Analista & 0 & 0€ \\
Progettista & 6 & 132€ \\
Programmatore & 15 & 225€  \\
Verificatore & 6 & 90€  \\
\rowcolor{lastrowcolor}
\textbf{Totale} & \textbf{34} & \textbf{617€}  \\
\end{tabular}
\end{table}
 
I dati ottenuti sono riassunti nel seguente areogramma:
\begin{figure}[H]
\centering
\includegraphics[scale=0.60]{img/grafici/torta_inc6.png}
\caption{Areogramma della ripartizione dei costi per ruolo per lo sviluppo dell'incremento 6}
\end{figure}

\paragraph*{Incremento 7}\mbox{} \\
\begin{table}[H]
\centering\renewcommand{\arraystretch}{1.5}
\caption{Prospetto dei costi per lo sviluppo dell'incremento 7}
\vspace{0.2cm}
\begin{tabular}{ c c c }
\rowcolor{redafk}
\textcolor{white}{\textbf{Ruolo}} & \textcolor{white}{\textbf{Ore}} &
\textcolor{white}{\textbf{Costo}}  \\
Responsabile & 2 & 60€ \\
Amministratore & 2 & 40€ \\
Analista & 0 & 0€ \\
Progettista & 10 & 220€ \\
Programmatore & 6 & 90€  \\
Verificatore & 4 & 60€  \\
\rowcolor{lastrowcolor}
\textbf{Totale} & \textbf{24} & \textbf{470€}  \\
\end{tabular}
\end{table}
 
I dati ottenuti sono riassunti nel seguente areogramma:
\begin{figure}[H]
\centering
\includegraphics[scale=0.60]{img/grafici/torta_inc7.png}
\caption{Areogramma della ripartizione dei costi per ruolo per lo sviluppo dell'incremento 7}
\end{figure}

\paragraph*{Incremento 8}\mbox{} \\
\begin{table}[H]
\centering\renewcommand{\arraystretch}{1.5}
\caption{Prospetto dei costi per lo sviluppo dell'incremento 8}
\vspace{0.2cm}
\begin{tabular}{ c c c }
\rowcolor{redafk}
\textcolor{white}{\textbf{Ruolo}} & \textcolor{white}{\textbf{Ore}} &
\textcolor{white}{\textbf{Costo}}  \\
Responsabile & 3 & 90€ \\
Amministratore & 4 & 80€ \\
Analista & 0 & 0€ \\
Progettista & 8 & 176€ \\
Programmatore & 20 & 300€  \\
Verificatore & 11 & 165€  \\
\rowcolor{lastrowcolor}
\textbf{Totale} & \textbf{46} & \textbf{811€}  \\
\end{tabular}
\end{table}
 
I dati ottenuti sono riassunti nel seguente areogramma:
\begin{figure}[H]
\centering
\includegraphics[scale=0.60]{img/grafici/torta_inc8.png}
\caption{Areogramma della ripartizione dei costi per ruolo per lo sviluppo dell'incremento 8}
\end{figure}

\paragraph*{Incremento 9}\mbox{} \\
\begin{table}[H]
\centering\renewcommand{\arraystretch}{1.5}
\caption{Prospetto dei costi per lo sviluppo dell'incremento 9}
\vspace{0.2cm}
\begin{tabular}{ c c c }
\rowcolor{redafk}
\textcolor{white}{\textbf{Ruolo}} & \textcolor{white}{\textbf{Ore}} &
\textcolor{white}{\textbf{Costo}}  \\
Responsabile & 2 & 60€ \\
Amministratore & 4 & 80€ \\
Analista & 0 & 0€ \\
Progettista & 15 & 330€ \\
Programmatore & 27 & 405€  \\
Verificatore & 20 & 300€  \\
\rowcolor{lastrowcolor}
\textbf{Totale} & \textbf{68} & \textbf{1175€}  \\
\end{tabular}
\end{table}
 
I dati ottenuti sono riassunti nel seguente areogramma:
\begin{figure}[H]
\centering
\includegraphics[scale=0.60]{img/grafici/torta_inc9.png}
\caption{Areogramma della ripartizione dei costi per ruolo per lo sviluppo dell'incremento 9}
\end{figure}

\paragraph*{Incremento 10}\mbox{} \\
\begin{table}[H]
\centering\renewcommand{\arraystretch}{1.5}
\caption{Prospetto dei costi per lo sviluppo dell'incremento 10}
\vspace{0.2cm}
\begin{tabular}{ c c c }
\rowcolor{redafk}
\textcolor{white}{\textbf{Ruolo}} & \textcolor{white}{\textbf{Ore}} &
\textcolor{white}{\textbf{Costo}}  \\
Responsabile & 1 & 30€ \\
Amministratore & 1 & 20€ \\
Analista & 0 & 0€ \\
Progettista & 7 & 154€ \\
Programmatore & 11 & 165€  \\
Verificatore & 9 & 135€  \\
\rowcolor{lastrowcolor}
\textbf{Totale} & \textbf{29} & \textbf{504€}  \\
\end{tabular}
\end{table}
 
I dati ottenuti sono riassunti nel seguente areogramma:
\begin{figure}[H]
\centering
\includegraphics[scale=0.60]{img/grafici/torta_inc10.png}
\caption{Areogramma della ripartizione dei costi per ruolo per lo sviluppo dell'incremento 10}
\end{figure}

\paragraph*{Incremento 11}\mbox{} \\
\begin{table}[H]
\centering\renewcommand{\arraystretch}{1.5}
\caption{Prospetto dei costi per lo sviluppo dell'incremento 11}
\vspace{0.2cm}
\begin{tabular}{ c c c }
\rowcolor{redafk}
\textcolor{white}{\textbf{Ruolo}} & \textcolor{white}{\textbf{Ore}} &
\textcolor{white}{\textbf{Costo}}  \\
Responsabile & 2 & 60€ \\
Amministratore & 3 & 60€ \\
Analista & 0 & 0€ \\
Progettista & 11 & 242€ \\
Programmatore & 18 & 270€  \\
Verificatore & 10 & 150€  \\
\rowcolor{lastrowcolor}
\textbf{Totale} & \textbf{44} & \textbf{782€}  \\
\end{tabular}
\end{table}
 
I dati ottenuti sono riassunti nel seguente areogramma:
\begin{figure}[H]
\centering
\includegraphics[scale=0.60]{img/grafici/torta_inc11.png}
\caption{Areogramma della ripartizione dei costi per ruolo per lo sviluppo dell'incremento 11}
\end{figure}

\paragraph*{Incremento 12}\mbox{} \\
\begin{table}[H]
\centering\renewcommand{\arraystretch}{1.5}
\caption{Prospetto dei costi per lo sviluppo dell'incremento 12}
\vspace{0.2cm}
\begin{tabular}{ c c c }
\rowcolor{redafk}
\textcolor{white}{\textbf{Ruolo}} & \textcolor{white}{\textbf{Ore}} &
\textcolor{white}{\textbf{Costo}}  \\
Responsabile & 2 & 60€ \\
Amministratore & 0 & 0€ \\
Analista & 0 & 0€ \\
Progettista & 12 & 264€ \\
Programmatore & 24 & 360€  \\
Verificatore & 15 & 225€  \\
\rowcolor{lastrowcolor}
\textbf{Totale} & \textbf{53} & \textbf{909€}  \\
\end{tabular}
\end{table}
 
I dati ottenuti sono riassunti nel seguente areogramma:
\begin{figure}[H]
\centering
\includegraphics[scale=0.60]{img/grafici/torta_inc12.png}
\caption{Areogramma della ripartizione dei costi per ruolo per lo sviluppo dell'incremento 12}
\end{figure}

%------------------------------------

\subsection{Periodo di Validazione e collaudo}
\subsubsection{Distribuzione oraria}
In questa fase i ruoli sono così suddivisi:

%tabella ore
\begin{table}[H]
\centering\renewcommand{\arraystretch}{1.5}
\caption{Validazione e Collaudo}
\vspace{0.2cm}
\begin{tabular}{ c c c c c c c c }
\rowcolor{redafk}
\textcolor{white}{\textbf{Nominativo}} & \textcolor{white}{\textbf{Re}} & 
\textcolor{white}{\textbf{Am}} & \textcolor{white}{\textbf{An}} &
\textcolor{white}{\textbf{Pt}} & \textcolor{white}{\textbf{Pm}} &
\textcolor{white}{\textbf{Ve}} & \textcolor{white}{\textbf{Totale}} \\
Simone Federico Bergamin 	& 5 	& - 	& - 	& - 	& 8 	& 12 	& 25 \\
Alessandro Canesso 			& 4 	& 4 	& - 	& - 	& - 	& 15 	& 23 \\
Victor Dutca 				& 5 	& - 	& - 	& - 	& 5 	& 15 	& 25 \\
Fouad Farid					& - 	& 6 	& - 	& - 	& 7 	& 12 	& 25 \\
Simone Meneghin 			& - 	& 9 	& - 	& - 	& - 	& 16 	& 25 \\
Olivier Utshudi 			& - 	& 4 	& - 	& 4 	& 5 	& 12 	& 25 \\
Davide Zilio 				& 6 	& - 	& - 	& 8 	& - 	& 11 	& 25 \\
\rowcolor{lastrowcolor}
\textbf{Ore totali ruolo} & \textbf{20} & \textbf{23} & \textbf{0} & \textbf{12} & \textbf{25} & \textbf{93} & \textbf{173} \\
\end{tabular}
\end{table}

I dati ottenuti sono riassunti nel seguente istogramma: 
\begin{figure}[H]
\centering
\includegraphics[scale=0.60]{img/grafici/tabella_fase_val_col.png}
\caption{Istogramma della ripartizione delle ore per ruolo nella fase di Validazione e collaudo}
\end{figure}

\paragraph{Distribuzione oraria incrementi}
\paragraph*{Incremento 13} \mbox{} \\
\begin{table}[H]
\centering\renewcommand{\arraystretch}{1.5}
\caption{Distribuzione delle ore per lo sviluppo dell'incremento 13}
\vspace{0.2cm}
\begin{tabular}{ c c c c c c c c }
\rowcolor{redafk}
\textcolor{white}{\textbf{Nominativo}} & \textcolor{white}{\textbf{Re}} &
\textcolor{white}{\textbf{Am}} & \textcolor{white}{\textbf{An}} &
\textcolor{white}{\textbf{Pt}} & \textcolor{white}{\textbf{Pm}} &
\textcolor{white}{\textbf{Ve}} & \textcolor{white}{\textbf{Totale}} \\
Simone Federico Bergamin & - & - & - & - & 2 & - & 2 \\
Alessandro Canesso & 2 & 1 & - & - & - & - & 3\\
Victor Dutca & - & - & - & - & - & - & 0 \\
Fouad Farid & - & - & - & - & - & 2 & 2 \\
Simone Meneghin & - & - & - & - & - & - & 0 \\
Olivier Utshudi & - & - & - & - & 2 & 3 & 5 \\
Davide Zilio & - & - & - & - & - & - & 0 \\
\rowcolor{lastrowcolor}
\textbf{Ore totali ruolo} & \textbf{2} & \textbf{1} & \textbf{0} & \textbf{2} & \textbf{5} & \textbf{2} & \textbf{12} \\
\end{tabular}
\end{table}

I dati ottenuti sono riassunti nel seguente istogramma:
\begin{figure}[H]
\centering
\includegraphics[scale=0.60]{img/grafici/tabella_inc13.png}
\caption{Istogramma della ripartizione delle ore per lo sviluppo dell'incremento 13}
\end{figure}

\paragraph*{Incremento 14} \mbox{} \\
\begin{table}[H]
\centering\renewcommand{\arraystretch}{1.5}
\caption{Distribuzione delle ore per lo sviluppo dell'incremento 14}
\vspace{0.2cm}
\begin{tabular}{ c c c c c c c c }
\rowcolor{redafk}
\textcolor{white}{\textbf{Nominativo}} & \textcolor{white}{\textbf{Re}} &
\textcolor{white}{\textbf{Am}} & \textcolor{white}{\textbf{An}} &
\textcolor{white}{\textbf{Pt}} & \textcolor{white}{\textbf{Pm}} &
\textcolor{white}{\textbf{Ve}} & \textcolor{white}{\textbf{Totale}} \\
Simone Federico Bergamin & - & - & - & - & - & - & 0 \\
Alessandro Canesso & 2 & 1 & - & - & - & - & 3\\
Victor Dutca & - & - & - & - & - & - & 0 \\
Fouad Farid & - & - & - & - & - & - & 0 \\
Simone Meneghin & - & - & - & - & - & - & 0 \\
Olivier Utshudi & - & - & - & 2 & 2 & - & 4 \\
Davide Zilio & - & - & - &- & - & 3 & 3 \\
\rowcolor{lastrowcolor}
\textbf{Ore totali ruolo} & \textbf{2} & \textbf{1} & \textbf{0} & \textbf{2} & \textbf{2} & \textbf{3} & \textbf{10} \\
\end{tabular}
\end{table}

I dati ottenuti sono riassunti nel seguente istogramma:
\begin{figure}[H]
\centering
\includegraphics[scale=0.60]{img/grafici/tabella_inc14.png}
\caption{Istogramma della ripartizione delle ore per lo sviluppo dell'incremento 14}
\end{figure}

\paragraph{Distribuzione oraria sviluppo test per il collaudo del prodotto finale}\mbox{} \\
\begin{table}[H]
\centering\renewcommand{\arraystretch}{1.5}
\caption{Distribuzione delle ore per lo sviluppo e verifica dei test e del prodotto finale}
\vspace{0.2cm}
\begin{tabular}{ c c c c c c c c }
\rowcolor{redafk}
\textcolor{white}{\textbf{Nominativo}} & \textcolor{white}{\textbf{Re}} &
\textcolor{white}{\textbf{Am}} & \textcolor{white}{\textbf{An}} &
\textcolor{white}{\textbf{Pt}} & \textcolor{white}{\textbf{Pm}} &
\textcolor{white}{\textbf{Ve}} & \textcolor{white}{\textbf{Totale}} \\
Simone Federico Bergamin & 5 & - & - & - & 6 & 12 & 23 \\
Alessandro Canesso & - & 2 & - & - & - & 15 & 17\\
Victor Dutca & 5 & - & - & - & 5 & 15 & 25 \\
Fouad Farid & - & 6 & - & - & 7 & 10 & 23 \\
Simone Meneghin & - & 9 & - & - & - & 16 & 25 \\
Olivier Utshudi & - & 4 & - & - & - & 12 & 16 \\
Davide Zilio & 6 & - & - & 8 & - & 8 & 22 \\
\rowcolor{lastrowcolor}
\textbf{Ore totali ruolo} & \textbf{16} & \textbf{21} & \textbf{0} & \textbf{8} & \textbf{18} & \textbf{88} & \textbf{151} \\
\end{tabular}
\end{table}

I dati ottenuti sono riassunti nel seguente istogramma:
\begin{figure}[H]
\centering
\includegraphics[scale=0.60]{img/grafici/tabella_test.png}
\caption{Istogramma della ripartizione delle ore per lo sviluppo e verifica dei test e del prodotto finale}
\end{figure}

\subsubsection{Prospetto economico}
In questa fase il costo per ogni ruolo è il seguente:

%tabella costi
\begin{table}[H]
\centering\renewcommand{\arraystretch}{1.5}
\caption{Prospetto dei costi nella fase di Validazione e collaudo}
\vspace{0.2cm}
\begin{tabular}{ c c c }
\rowcolor{redafk}
\textcolor{white}{\textbf{Ruolo}} & \textcolor{white}{\textbf{Ore}} & 
\textcolor{white}{\textbf{Costo}}  \\
Responsabile & 20 & 600€ \\
Amministratore & 23 & 460€ \\
Analista & - & - \\
Progettista	& 12 & 264€ \\
Programmatore & 25 & 375€  \\
Verificatore & 93 & 1395€  \\
\rowcolor{lastrowcolor}
\textbf{Totale} & \textbf{173} & \textbf{3094€}  \\
\end{tabular}
\end{table}

I dati ottenuti si possono riassumere nel seguente areogramma:
\begin{figure}[H]
\centering
\includegraphics[scale=0.60]{img/grafici/torta_fase_val_col.png}
\caption{Areogramma della ripartizione dei costi per ruolo nella fase di Validazione e collaudo}
\end{figure}

\paragraph{Prospetto economico incrementi}
\paragraph*{Incremento 13}\mbox{} \\
\begin{table}[H]
\centering\renewcommand{\arraystretch}{1.5}
\caption{Prospetto dei costi per lo sviluppo dell'incremento 13}
\vspace{0.2cm}
\begin{tabular}{ c c c }
\rowcolor{redafk}
\textcolor{white}{\textbf{Ruolo}} & \textcolor{white}{\textbf{Ore}} &
\textcolor{white}{\textbf{Costo}}  \\
Responsabile & 2 & 60€ \\
Amministratore & 1 & 20€ \\
Analista & 0 & 0€ \\
Progettista & 2 & 44€ \\
Programmatore & 5 & 75€  \\
Verificatore & 2 & 30€  \\
\rowcolor{lastrowcolor}
\textbf{Totale} & \textbf{12} & \textbf{229€}  \\
\end{tabular}
\end{table}
 
I dati ottenuti sono riassunti nel seguente areogramma:
\begin{figure}[H]
\centering
\includegraphics[scale=0.60]{img/grafici/torta_inc13.png}
\caption{Areogramma della ripartizione dei costi per ruolo per lo sviluppo dell'incremento 13}
\end{figure}

\paragraph*{Incremento 14}\mbox{} \\
\begin{table}[H]
\centering\renewcommand{\arraystretch}{1.5}
\caption{Prospetto dei costi per lo sviluppo dell'incremento 14}
\vspace{0.2cm}
\begin{tabular}{ c c c }
\rowcolor{redafk}
\textcolor{white}{\textbf{Ruolo}} & \textcolor{white}{\textbf{Ore}} &
\textcolor{white}{\textbf{Costo}}  \\
Responsabile & 2 & 60€ \\
Amministratore & 1 & 20€ \\
Analista & 0 & 0€ \\
Progettista & 2 & 44€ \\
Programmatore & 2 & 30€  \\
Verificatore & 3 & 45€  \\
\rowcolor{lastrowcolor}
\textbf{Totale} & \textbf{10} & \textbf{199€}  \\
\end{tabular}
\end{table}
 
I dati ottenuti sono riassunti nel seguente areogramma:
\begin{figure}[H]
\centering
\includegraphics[scale=0.60]{img/grafici/torta_inc14.png}
\caption{Areogramma della ripartizione dei costi per ruolo per lo sviluppo dell'incremento 14}
\end{figure}

\paragraph{Prospetto economico sviluppo test per il collaudo del prodotto finale}\mbox{} \\
\begin{table}[H]
\centering\renewcommand{\arraystretch}{1.5}
\caption{Prospetto dei costi per lo sviluppo e verifica dei test e del prodotto finale}
\vspace{0.2cm}
\begin{tabular}{ c c c }
\rowcolor{redafk}
\textcolor{white}{\textbf{Ruolo}} & \textcolor{white}{\textbf{Ore}} &
\textcolor{white}{\textbf{Costo}}  \\
Responsabile & 16 & 480€ \\
Amministratore & 21 & 420€ \\
Analista & 0 & 0€ \\
Progettista & 8 & 176€ \\
Programmatore & 18 & 270€  \\
Verificatore & 88 & 1320€  \\
\rowcolor{lastrowcolor}
\textbf{Totale} & \textbf{151} & \textbf{2666€}  \\
\end{tabular}
\end{table}
 
I dati ottenuti sono riassunti nel seguente areogramma:
\begin{figure}[H]
\centering
\includegraphics[scale=0.60]{img/grafici/torta_test.png}
\caption{Areogramma della ripartizione dei costi per ruolo per lo sviluppo e verifica dei test e del prodotto finale}
\end{figure}

\subsection{Riepilogo}
\subsubsection{Ore rendicontate con investimento}
\paragraph{Distribuzione oraria} \mbox{} \\ \mbox{} \\
Vengono riportate il totale delle ore del progetto in cui sono presenti le ore di investimento e le
ore rendicontate a carico del committente:

%tabella ore
\begin{table}[H]
\centering\renewcommand{\arraystretch}{1.5}
\caption{Distribuzione totale delle ore dell'intero progetto con investimento}
\vspace{0.2cm}
\begin{tabular}{ c c c c c c c c }
\rowcolor{redafk}
\textcolor{white}{\textbf{Nominativo}} & \textcolor{white}{\textbf{Re}} & 
\textcolor{white}{\textbf{Am}} & \textcolor{white}{\textbf{An}} &
\textcolor{white}{\textbf{Pt}} & \textcolor{white}{\textbf{Pm}} &
\textcolor{white}{\textbf{Ve}} & \textcolor{white}{\textbf{Totale}} \\
Simone Federico Bergamin 	& 11 	& 13 	& 30 	& 19 	& 31 	& 41 	& 145 \\
Alessandro Canesso 			& 16 	& 18 	& 16 	& 20 	& 27 	& 46 	& 143 \\
Victor Dutca 				& 17	& 14 	& 19 	& 20 	& 32 	& 41 	& 143 \\
Fouad Farid					& 11	& 18 	& 12 	& 32 	& 27 	& 43 	& 143 \\
Simone Meneghin 			& 8 	& 17 	& 22 	& 32 	& 28 	& 40 	& 147 \\
Olivier Utshudi 			& 8 	& 16 	& 13 	& 28 	& 33 	& 49 	& 147 \\
Davide Zilio 				& 13 	& 11 	& 30 	& 18 	& 20 	& 51 	& 143 \\
\rowcolor{lastrowcolor}
\textbf{Ore totali ruolo} & \textbf{84} & \textbf{107} & \textbf{142} & \textbf{169} & \textbf{198} & \textbf{311} & \textbf{1011} \\
\end{tabular}
\end{table}

Una rappresentazione visiva della suddivisione oraria viene data dal seguente grafico:
\begin{figure}[H]
\centering
\includegraphics[scale=0.60]{img/grafici/tabella_tot_con_analisi.png}
\caption{Istogramma della ripartizione delle ore totali per ruolo con investimento}
\end{figure}

\paragraph{Prospetto economico} \mbox{} \\ \mbox{} \\
Il costo totale con investimento è riportato nella seguente tabella:

%tabella costi
\begin{table}[H]
\centering\renewcommand{\arraystretch}{1.5}
\caption{Costi totali con investimento}
\vspace{0.2cm}
\begin{tabular}{ c c c }
\rowcolor{redafk}
\textcolor{white}{\textbf{Ruolo}} & \textcolor{white}{\textbf{Ore}} & 
\textcolor{white}{\textbf{Costo}}  \\
Responsabile & 84 & 2520€ \\
Amministratore & 107 & 2140€ \\
Analista & 142 & 3550€ \\
Progettista	& 169 & 3718€ \\
Programmatore & 198 & 2970€  \\
Verificatore & 311 & 4665€  \\
\rowcolor{lastrowcolor}
\textbf{Totale} & \textbf{1011} & \textbf{19563€}  \\
\end{tabular}
\end{table}

I dati ottenuti si possono riassumere nel seguente areogramma:
\begin{figure}[H]
\centering
\includegraphics[scale=0.60]{img/grafici/torta_tot_con_analisi.png}
\caption{Areogramma della ripartizione dei costi totali per ruolo con investimento}
\end{figure}

\subsubsection{Ore rendicontate senza investimento}
\paragraph{Distribuzione oraria} \mbox{} \\ \mbox{} \\
Le ore rendicontate sono riassunte nella seguente tabella:

%tabella ore
\begin{table}[H]
\centering\renewcommand{\arraystretch}{1.5}
\caption{Distribuzione totale delle ore dell'intero progetto senza investimento}
\vspace{0.2cm}
\begin{tabular}{ c c c c c c c c }
\rowcolor{redafk}
\textcolor{white}{\textbf{Nominativo}} & \textcolor{white}{\textbf{Re}} & 
\textcolor{white}{\textbf{Am}} & \textcolor{white}{\textbf{An}} &
\textcolor{white}{\textbf{Pt}} & \textcolor{white}{\textbf{Pm}} &
\textcolor{white}{\textbf{Ve}} & \textcolor{white}{\textbf{Totale}} \\
Simone Federico Bergamin 	& 5 	& 6 	& 10	& 19	& 31	& 32 	& 103 \\
Alessandro Canesso 			& 8 	& 12	& - 	& 20	& 27	& 34 	& 101 \\
Victor Dutca 				& 8 	& 14	& 4 	& 20	& 32	& 25 	& 103 \\
Fouad Farid					& 4 	& 11	& - 	& 26	& 27	& 35 	& 103 \\
Simone Meneghin 			& 8 	& 9 	& 8 	& 22	& 28	& 30 	& 105 \\
Olivier Utshudi 			& 8 	& 8 	& - 	& 20	& 33	& 36 	& 105 \\
Davide Zilio 				& 9 	& 6 	& 13	& 18	& 20	& 37 	& 103 \\
\rowcolor{lastrowcolor}
\textbf{Ore totali ruolo} & \textbf{50} & \textbf{66} & \textbf{35} & \textbf{145} & \textbf{198} & \textbf{229} & \textbf{723} \\
\end{tabular}
\end{table}

Una rappresentazione visiva della suddivisione oraria viene data dal seguente grafico:
\begin{figure}[H]
\centering
\includegraphics[scale=0.60]{img/grafici/tabella_tot_no_analisi.png}
\caption{Istogramma della ripartizione delle ore totali per ruolo con investimento}
\end{figure}

\paragraph{Prospetto economico} \mbox{} \\ \mbox{} \\
Il costo totale senza investimento è riportato nella seguente tabella:

%tabella costi
\begin{table}[H]
\centering\renewcommand{\arraystretch}{1.5}
\caption{Costi totali senza investimento}
\vspace{0.2cm}
\begin{tabular}{ c c c }
\rowcolor{redafk}
\textcolor{white}{\textbf{Ruolo}} & \textcolor{white}{\textbf{Ore}} & 
\textcolor{white}{\textbf{Costo}}  \\
Responsabile & 50 & 1500€ \\
Amministratore & 66 & 1320€ \\
Analista & 35 & 875€ \\
Progettista	& 145 & 3190€ \\
Programmatore & 198 & 2970€  \\
Verificatore & 229 & 3435€  \\
\rowcolor{lastrowcolor}
\textbf{Totale} & \textbf{723} & \textbf{13290€}  \\
\end{tabular}
\end{table}

I dati ottenuti si possono riassumere nel seguente areogramma:
\begin{figure}[H]
\centering
\includegraphics[scale=0.60]{img/grafici/torta_tot_no_analisi.png}
\caption{Areogramma della ripartizione dei costi totali per ruolo senza investimento}
\end{figure}

\subsection{Conclusioni}
Il costo totale preventivato per il progetto è 13.290,00€
\pagebreak

\section{Consuntivo di periodo}
Di seguito verranno indicate le spese effettivamente sostenute da ogni ruolo. Il bilancio di consuntivo potrà risultare: \begin{itemize}
\item \textbf{Positivo}: se il preventivo supera il consuntivo;
\item \textbf{Pari}: se preventivo e consuntivo sono uguali;
\item \textbf{Negativo}: se il consuntivo supera il preventivo.
\end{itemize}

\subsection{Analisi}
%tabella costi
\begin{table}[H]
\centering\renewcommand{\arraystretch}{1.5}
\caption{Consuntivo del periodo di Analisi}
\vspace{0.2cm}
\begin{tabular}{ c c c }
\rowcolor{redafk}
\textcolor{white}{\textbf{Ruolo}} & \textcolor{white}{\textbf{Ore}} & 
\textcolor{white}{\textbf{Costo}}  \\
Responsabile & 34 & 1020€ \\
Amministratore & 41 (+13) & 820€ (+260€) \\
Analista & 107 (+8) & 2675€ (+200€) \\
Progettista	& 24 & 528€  \\
Programmatore & 0 & 0€  \\
Verificatore & 82 (+9) & 1230€ (+135€)  \\
\textbf{Totale preventivo} & \textbf{288} & \textbf{6273€}  \\
\textbf{Totale consuntivo} & \textbf{318} & \textbf{6868€}  \\
\rowcolor{lastrowcolor}
\textbf{Differenza} & \textbf{+30} & \textbf{+595€}  \\
\end{tabular}
\end{table}

\subsubsection{Conclusioni}
Come emerge dai dati riportati nella tabella soprastante è stato necessario investire più tempo del previsto nei ruoli di \textit{Amministratore}, \textit{Analista} e \textit{Verificatore}. Per questo motivo il bilancio risultante è negativo. Le cause di tali ritardi sono riportate di seguito:
\begin{itemize}
\item \textbf{\textit{Amministratore}}: è servito più tempo del previsto per riuscire ad individuare i software più adatti per la gestione del progetto e per la loro  configurazione. Inoltre sono state aggiunte ed aggiornate alcune sezioni nelle \textit{Norme di Progetto}, necessarie al chiarimento di alcune problematiche sorte durante la stesura dei documenti;
\item \textbf{\textit{Analista}}: alcuni requisiti si sono rivelati di non facile comprensione, e sono state necessarie più ore di lavoro per la discussione interna tra gli \textit{Analisti} ed esterna con il proponente;
\item \textbf{\textit{Verificatore}}: l’aggiunta di nuove sezioni nelle \textit{Norme di Progetto} e l'inesperienza dei membri hanno implicato un maggiore lavoro anche per questo ruolo.
\end{itemize}
Il notevole quantitativo di ore che il gruppo ha dovuto impiegare nel primo periodo non deve ripetersi durante il lavoro rendicontato. Per le problematiche riscontrate verranno adottate le seguenti contromisure:
\begin{itemize}
	\item amministrazione degli strumenti: il gruppo ha ricercato e configurato in anticipo gli strumenti che verranno usati. In caso venissero individuati nuovi strumenti avere già un ambiente di sviluppo impostato correttamente per tutti i membri semplificherà la nuova configurazione e ridurrà l'insorgere di problemi;
	\item comprensione dei requisiti: i requisiti sono stati ampiamente discussi con il proponente durante questa fase, non si prevede di incorrere ulteriormente in tale problema;
	\item applicazione delle norme: i membri del gruppo hanno studiato attentamente le norme, in modo tale da poter redigere fin da subito nuove sezioni dei documenti già normate, semplificando il lavoro ai verificatori.
\end{itemize}

\subsubsection{Preventivo a finire}
Essendo questo periodo non rendicontato, non vengono a generarsi problemi nel monte ore totale, nonché nel preventivo economico. Nonostante ciò \textit{TeamAFK} si impegnerà a integrare altre misure di contenimento ad eventuali nuovi problemi, facendo esperienza dei problemi riscontrati durante questo primo periodo

\subsection{Progettazione e codifica per la Technology Baseline}

\begin{table}[H]
\centering\renewcommand{\arraystretch}{1.5}
\caption{Consuntivo del periodo di Progettazione e codifica per la Technology Baseline}
\vspace{0.2cm}
\begin{tabular}{ c c c }
\rowcolor{redafk}
\textcolor{white}{\textbf{Ruolo}} & \textcolor{white}{\textbf{Ore}} &
\textcolor{white}{\textbf{Costo}}  \\
Responsabile 	& 12 & 360€ \\
Amministratore 	& 20 (+9) 	& 400€ (+180€) \\
Analista 		& 35 (-20) 	& 875€ (-500€) \\
Progettista		& 59 (+8) 	& 1298€ (+176€)\\
Programmatore	& 33 (+13) 	& 495€ (+195€)\\
Verificatore 	& 53 & 795€ \\
\textbf{Totale preventivo} & \textbf{212} & \textbf{4223€}  \\
\textbf{Totale consuntivo} & \textbf{222} & \textbf{4274€}  \\
\rowcolor{lastrowcolor}
\textbf{Differenza} & \textbf{+10} & \textbf{+51€}  \\
\end{tabular}
\end{table}

\subsubsection{Analisi degli incrementi}
Al fine di garantire uno sviluppo del progetto congruo con quanto preventivato
nei tempi e nei costi, gli incrementi individuati nella pianificazione sono stati sviluppati in parallelo. Terminata la fase di codifica, si rilevano eventuali problemi riscontrati ed eventualmente si modifica e si dettaglia ulteriormente la pianificazione futura in modo da mitigare gli effetti di questi imprevisti.

\paragraph*{Incremento 1} \mbox{} \\ \mbox{} \\
Il tempo dedicato alla codifica di questo incremento e le risorse assegnategli sono risultati ben bilanciati. I programmatori dedicati sono riusciti a sviluppare quanto pianificato entro la scadenza prefissata. \\ 
Terminato il proprio compito, quest'ultimi si sono messi a disposizione dei propri colleghi.

\paragraph*{Incremento 2} \mbox{} \\ \mbox{} \\
La codifica di questo incremento ha avuto dei rallentamenti dovuti all'inesperienza dei programmatori con tale tecnologia unita alla scarsa documentazione fornita da Grafana. \\ Il \textit{TeamAFK} ha sfruttato questo periodo per incrementare le proprie conoscenze ed abilità in relazione all'ambiente di sviluppo e ai linguaggi di programmazione necessari allo sviluppo di tale e future funzionalità. 

\paragraph*{Incremento 3} \mbox{} \\ \mbox{} \\
La codifica di questo incremento, come per l'incremento 2, ha avuto dei rallentamenti dovuti all'inesperienza dei programmatori con tale tecnologia unita alla scarsa documentazione fornita da Grafana. \\ 
Il \textit{TeamAFK} ha pertanto sfruttato questo periodo per incrementare le proprie conoscenze ed abilità in relazione all'ambiente di sviluppo e ai linguaggi di programmazione necessari allo sviluppo di tale e future funzionalità.

\subsubsection{Conclusioni}
Come emerge dai dati riportati nella tabella soprastante è stato necessario investire più tempo nei ruoli di \textit{Amministratore}, \textit{Progettista} e \textit{Programmatore} mentre l'\textit{Analista} ha visto una riduzione delle sue ore. Le cause di tali scostamenti sono riportate di seguito:
\begin{itemize}
	\item \textbf{\textit{Amministratore}}: la causa di questo aumento di ore è dovuto all'aggiunta e modifica di alcune parti delle \textit{Norme di Progetto};
	\item \textbf{\textit{Analista}}: l'elevata comunicazione con il proponente nel periodo di analisi ha permesso un'ottima comprensione del prodotto da sviluppare, questo ha permesso di concentrarsi principalmente sulla correzione dell'\textit{Analisi dei Requisiti};
	\item \textbf{\textit{Progettista}}: le ore aggiuntive sono state richieste per la correzione del documento \textit{Priano di Qualifica};
	\item \textbf{\textit{Programmatore}}: data l'inesperienza con le tecnologie utilizzate per lo sviluppo del software, sono state richieste più ore di programmazione per comprendere e quindi correggere i problemi che si sono presentati durante la codifica delle funzionalità previste per la PoC.
\end{itemize}

Rispetto alla fase di analisi, le ore aggiunte sono decisamente ridotte, però in questo caso le ore sono rendicontate, quindi lo sforamento è ben più grave. Per le problematiche riscontrate verranno adottate le seguenti contromisure: 
\begin{itemize}
	\item mancanza ed errata stesura di alcune sezioni delle norme: è stata prestata particolare attenzione durante la correzione, in modo tale che non si debbano correggere ulteriormente le \textit{Norme di Progetto} in futuro;
	\item correzione dei documenti: durante la stesura e la verifica si è stati più meticolosi, così da ridurre il più possibile eventuali nuove correzioni;
	\item inesperienza tecnologica: durante questa fase si è analizzato le componenti del prodotto che potrebbero essere più complicate, ricercando in anticipo informazioni ed possibili soluzioni.
\end{itemize}

\subsubsection{Preventivo a finire}
Il bilancio risultante è negativo, in quanto sono stati spesi 51€ in più rispetto a quanto preventivato. Per questo motivo sarà necessario impegnarsi per ridurre il costo dei successivi periodi senza però intaccare la qualità del prodotto finale.

\subsection{Periodo di progettazione di dettaglio e codifica}
\begin{table}[H]
\centering\renewcommand{\arraystretch}{1.5}
\caption{Consuntivo del periodo di progettazione di dettaglio e codifica}
\vspace{0.2cm}
\begin{tabular}{ c c c }
\rowcolor{redafk}
\textcolor{white}{\textbf{Ruolo}} & \textcolor{white}{\textbf{Ore}} &
\textcolor{white}{\textbf{Costo}}  \\
Responsabile 	& 18 & 540€ \\
Amministratore 	&  23	& 460€ \\
Analista 		& 0 (+6)  & 0 (+150€) \\
Progettista		&  74 (-8) & 1628€ (-176€)\\
Programmatore	&  	140 (-4)& 2100€ (-60€)\\
Verificatore 	& 83 (+2) &  1245€ (+30€)\\
\textbf{Totale preventivo} & \textbf{338} & \textbf{5973€}  \\
\textbf{Totale consuntivo} & \textbf{334} & \textbf{5917€} \\
\rowcolor{lastrowcolor}
\textbf{Differenza} & \textbf{-4} & \textbf{-56€} \\
\end{tabular}
\end{table}

\subsubsection{Analisi degli incrementi}
Di seguito sono analizzati tutti gli incrementi singolarmente, ed ognuno contiene una breve descrizione dei costi, negativi o positivi, presenti nella tabella.

\paragraph*{Incremento 4} \mbox{} \\ \mbox{} \\
Come riportato nella tabella seguente, la codifica di questo incremento ha richiesto meno tempo di quello preventivato, in quanto i \textit{Programmatori} avevano acquisito le conoscenze necessarie allo sviluppo di quest'ultimo nella fase precedente di Progettazione e codifica per la Technology Baseline (in particolare durante lo sviluppo dell'incremento 1). Sono state apportate inoltre delle piccole modifiche alla struttura del codice.
\begin{table}[H]
\centering\renewcommand{\arraystretch}{1.5}
\caption{Consuntivo dell'incremento 4}
\vspace{0.2cm}
\begin{tabular}{ c c c }
\rowcolor{redafk}
\textcolor{white}{\textbf{Ruolo}} & \textcolor{white}{\textbf{Ore}} &
\textcolor{white}{\textbf{Costo}}  \\
Responsabile 	& 2 & 60€ \\
Amministratore 	&  4 & 80€ \\
Analista 		&  0 & 0€ \\
Progettista		&  4 (-1) & 88€ (-22€)\\
Programmatore	&  14 (-2) & 240€ (-30€)\\
Verificatore 	& 6 & 90€ \\
\textbf{Totale preventivo} & \textbf{32} & \textbf{558€}  \\
\textbf{Totale consuntivo} & \textbf{29} & \textbf{506€}  \\
\rowcolor{lastrowcolor}
\textbf{Differenza} & \textbf{-3} & \textbf{-52€} \\
\end{tabular}
\end{table}

\paragraph*{Incremento 5} \mbox{} \\ \mbox{} \\
Come riportato nella tabella seguente, la codifica di questo incremento ha rispettato le tempistiche e i costi preventivati.
\begin{table}[H]
\centering\renewcommand{\arraystretch}{1.5}
\caption{Consuntivo dell'incremento 5}
\vspace{0.2cm}
\begin{tabular}{ c c c }
\rowcolor{redafk}
\textcolor{white}{\textbf{Ruolo}} & \textcolor{white}{\textbf{Ore}} &
\textcolor{white}{\textbf{Costo}}  \\
Responsabile 	& 1 & 30€ \\
Amministratore 	& 1 & 20€ \\
Analista 		& 0 & 0€ \\
Progettista		& 1 & 22€ \\
Programmatore	& 3 & 45€ \\
Verificatore 	& 2 & 30€ \\
\textbf{Totale preventivo} & \textbf{8} & \textbf{147€}  \\
\textbf{Totale consuntivo} & \textbf{8} & \textbf{147€}  \\
\rowcolor{lastrowcolor}
\textbf{Differenza} & \textbf{0} & \textbf{0€} \\
\end{tabular}
\end{table}

\paragraph*{Incremento 6} \mbox{} \\ \mbox{} \\
Come riportato nella tabella seguente, la codifica di questo incremento ha subito una leggera variazione, positiva, in quanto è stato implementato quanto necessario leggermente più velocemente rispetto a quanto pianificato.
\begin{table}[H]
\centering\renewcommand{\arraystretch}{1.5}
\caption{Consuntivo dell'incremento 6}
\vspace{0.2cm}
\begin{tabular}{ c c c }
\rowcolor{redafk}
\textcolor{white}{\textbf{Ruolo}} & \textcolor{white}{\textbf{Ore}} &
\textcolor{white}{\textbf{Costo}}  \\
Responsabile 	& 3 & 90€ \\
Amministratore 	& 4 & 80€ \\
Analista 		& 0  & 0€ \\
Progettista		& 6  &  132€\\
Programmatore	& 14 & 225€ (-15€) \\
Verificatore 	& 6 & 90€ \\
\textbf{Totale preventivo} & \textbf{34} & \textbf{617€}   \\
\textbf{Totale consuntivo} & \textbf{33} & \textbf{602€}  \\
\rowcolor{lastrowcolor}
\textbf{Differenza} & \textbf{-1} & \textbf{-15€} \\
\end{tabular}
\end{table}

\paragraph*{Incremento 7} \mbox{} \\ \mbox{} \\
Come riportato nella tabella seguente, la codifica di questo incremento ha richiesto meno ore di progettazione, in quanto l'architettura necessaria al suo sviluppo è stata ben definita dai \textit{Progettisti} e i \textit{Programmatori} non hanno avuto necessità di chiedere maggiori informazioni.
\begin{table}[H]
\centering\renewcommand{\arraystretch}{1.5}
\caption{Consuntivo dell'incremento 7}
\vspace{0.2cm}
\begin{tabular}{ c c c }
\rowcolor{redafk}
\textcolor{white}{\textbf{Ruolo}} & \textcolor{white}{\textbf{Ore}} &
\textcolor{white}{\textbf{Costo}}  \\
Responsabile 	& 2 & 60€ \\
Amministratore 	& 2 &  40€ \\
Analista 		& 0  & 0€ \\
Progettista		& 10 (-2)  & 220€ (-44€) \\
Programmatore	&  6 & 90€ \\
Verificatore 	&  4 & 60€ \\
\textbf{Totale preventivo} & \textbf{24} & \textbf{470€}  \\
\textbf{Totale consuntivo} & \textbf{22} & \textbf{426€}  \\
\rowcolor{lastrowcolor}
\textbf{Differenza} & \textbf{-2} & \textbf{-44€} \\
\end{tabular}
\end{table}

\paragraph*{Incremento 8} \mbox{} \\ \mbox{} \\
Come riportato nella tabella seguente, la codifica di questo incremento ha richiesto piu soldi (e ore) di quanto preventivato, in quanto sono stati riscontrati dei problemi nell'\textit{Analisi dei Requisiti}, da risolvere subito e nel minor tempo possibile. Tal'ultimi hanno richiesto 6 ore di analisi in più di quanto preventivato inizialmente.\\
La progettazione, invece, ha richiesto 1 ora in meno per mettere a punto e sviluppare l'idea architetturale dietro a questo incremento.
\begin{table}[H]
\centering\renewcommand{\arraystretch}{1.5}
\caption{Consuntivo dell'incremento 8}
\vspace{0.2cm}
\begin{tabular}{ c c c }
\rowcolor{redafk}
\textcolor{white}{\textbf{Ruolo}} & \textcolor{white}{\textbf{Ore}} &
\textcolor{white}{\textbf{Costo}}  \\
Responsabile 	& 3 & 90€ \\
Amministratore 	& 4 & 80€ \\
Analista 		& 0 (+6)  & 0 (+150€) \\
Progettista		& 8 (-1)  & 176€ (-22€) \\
Programmatore	& 20 & 300€ \\
Verificatore 	& 11 & 165€  \\
\textbf{Totale preventivo} & \textbf{46} & \textbf{811€}  \\
\textbf{Totale consuntivo} & \textbf{51} & \textbf{939€} \\
\rowcolor{lastrowcolor}
\textbf{Differenza} & \textbf{+5} & \textbf{+128€} \\
\end{tabular}
\end{table}

\paragraph*{Incremento 9} \mbox{} \\ \mbox{} \\
Come riportato nella tabella seguente, la codifica di questo incremento ha richiesto meno ore di progettazione, in quanto la sua struttura era già ben definita. 
\begin{table}[H]
\centering\renewcommand{\arraystretch}{1.5}
\caption{Consuntivo dell'incremento 9}
\vspace{0.2cm}
\begin{tabular}{ c c c }
\rowcolor{redafk}
\textcolor{white}{\textbf{Ruolo}} & \textcolor{white}{\textbf{Ore}} &
\textcolor{white}{\textbf{Costo}}  \\
Responsabile 	& 2 & 60€ \\
Amministratore 	& 4 & 80€ \\
Analista 		&  0 & 0€ \\
Progettista		&  15 (-4) & 330€ (-88€) \\
Programmatore	&  27 (-1) & 405€ (-15€) \\
Verificatore 	&  20 & 300€ \\
\textbf{Totale preventivo} & \textbf{68} & \textbf{1175€}  \\
\textbf{Totale consuntivo} & \textbf{63} & \textbf{1072€}  \\
\rowcolor{lastrowcolor}
\textbf{Differenza} & \textbf{-5} & \textbf{-103€} \\
\end{tabular}
\end{table}

\paragraph*{Incremento 10} \mbox{} \\ \mbox{} \\
Come riportato nella tabella seguente, la codifica di questo incremento ha rispetto quanto preventivato.
\begin{table}[H]
\centering\renewcommand{\arraystretch}{1.5}
\caption{Consuntivo dell'incremento 1o}
\vspace{0.2cm}
\begin{tabular}{ c c c }
\rowcolor{redafk}
\textcolor{white}{\textbf{Ruolo}} & \textcolor{white}{\textbf{Ore}} &
\textcolor{white}{\textbf{Costo}}  \\
Responsabile 	& 1 & 30€ \\
Amministratore 	& 1 & 20€ \\
Analista 		& 0 & 0€ \\
Progettista		& 7 & 154€ \\
Programmatore	& 11 & 165€\\
Verificatore 	& 9 &  135€\\
\textbf{Totale preventivo} & \textbf{29} & \textbf{504€}  \\
\textbf{Totale consuntivo} & \textbf{29} & \textbf{504€}  \\
\rowcolor{lastrowcolor}
\textbf{Differenza} & \textbf{0} & \textbf{0€} \\
\end{tabular}
\end{table}

\paragraph*{Incremento 11} \mbox{} \\ \mbox{} \\
Come riportato nella tabella seguente, la codifica di questo incremento ha richiesto due ore in più di verifica in quanto lo sviluppo di alcuni test ha richiesto più tempo del previsto data l'inesperienza del \textit{TeamAFK} con questo tipologia di codifica.
\begin{table}[H]
\centering\renewcommand{\arraystretch}{1.5}
\caption{Consuntivo dell'incremento 11}
\vspace{0.2cm}
\begin{tabular}{ c c c }
\rowcolor{redafk}
\textcolor{white}{\textbf{Ruolo}} & \textcolor{white}{\textbf{Ore}} &
\textcolor{white}{\textbf{Costo}}  \\
Responsabile 	& 2 & 60€  \\
Amministratore 	& 3	& 60€\\
Analista 		& 0  & 0€ \\
Progettista		& 11 & 242€ \\
Programmatore	& 18 & 270€ \\
Verificatore 	& 12 (+2) & 150€ (+30€) \\
\textbf{Totale preventivo} & \textbf{44} & \textbf{782€}  \\
\textbf{Totale consuntivo} & \textbf{46} & \textbf{812€}  \\
\rowcolor{lastrowcolor}
\textbf{Differenza} & \textbf{+2} & \textbf{+30€} \\
\end{tabular}
\end{table}

\paragraph*{Incremento 12} \mbox{} \\ \mbox{} \\
Come riportato nella tabella seguente, la codifica di questo incremento ha rispetto quanto preventivato.
\begin{table}[H]
\centering\renewcommand{\arraystretch}{1.5}
\caption{Consuntivo dell'incremento 12}
\vspace{0.2cm}
\begin{tabular}{ c c c }
\rowcolor{redafk}
\textcolor{white}{\textbf{Ruolo}} & \textcolor{white}{\textbf{Ore}} &
\textcolor{white}{\textbf{Costo}}  \\
Responsabile 	& 2 & 60€ \\
Amministratore 	& 0 & 0€ \\
Analista 		&  0 & 0€ \\
Progettista		&  12 & 264€ \\
Programmatore	&  24 & 360€ \\
Verificatore 	&  15 & 225€ \\
\textbf{Totale preventivo} & \textbf{53} & \textbf{909€}  \\
\textbf{Totale consuntivo} & \textbf{53} & \textbf{909€}  \\
\rowcolor{lastrowcolor}
\textbf{Differenza} & \textbf{0} & \textbf{0€} \\
\end{tabular}
\end{table}


\subsubsection{Conclusioni generali}
Come emerge dai dati riportati nella tabella è stato necessario investire più tempo nei ruoli di \textit{Analista} e \textit{Verificatore} mentre il \textit{Programmatore} e il \textit{Progettista} hanno visto una riduzione delle loro ore. Le cause di tali scostamenti sono riportate di seguito:
\begin{itemize}
	\item \textbf{\textit{Analista}}: è stato necessario rivedere la struttura dei casi d'uso e l'importanza di alcuni requisiti, in modo da soddisfare e rispettare il \textit{Piano di Progetto}. Infine, sono state apportate le correzioni indicate dal committente;
	\item \textbf{\textit{Progettista}}: i \textit{Progettisti} avevano studiato, descritto e impostato in modo esaustivo l'architettura del prodotto nella fase precedente; è stato richiesto quindi meno tempo per progettare quanto codificato e mostrato durante la Product Baseline e la Revisione di Qualifica; 
	\item \textbf{\textit{Programmatore}}: sono state richieste meno ore di programmazione per sviluppare quanto descritto in questo documento, data l'esperienza acquisita durante la Technology Baseline;
	\item \textbf{\textit{Verificatore}}: sono state richieste più ore di verifica, in quanto sono stati riscontrati dei problemi nella stesura dei test; nessun membro del \textit{TeamAFK} aveva esperienza con questo tipo di codifica.
\end{itemize}
Rispetto alla fase di Progettazione e codifica per la Technology Baseline, le ore aggiunte sono dovute principalmente all'inesperienza del \textit{TeamAFK} nella stesura di codice di test. \\
Per le problematiche riscontrate verranno adottate le seguenti contromisure: 
\begin{itemize}
	\item correzione dei documenti: durante la stesura e la verifica si è stati più meticolosi, così da ridurre il più possibile eventuali nuove correzioni;
	\item inesperienza tecnologica: questo periodo ha permesso al \textit{TeamAFK} di sviluppare nuove conoscenze. Quest'ultime velocizzeranno il processo di codifica del periodo successivo, in cui il progetto dovrà essere terminato del tutto.
\end{itemize}


\subsubsection{Preventivo a finire generale}
Le modifiche apportate alla pianificazione ci hanno permesso di avere un
bilancio positivo. Tale bilancio risulta tale in quanto i costi che abbiamo effettivamente rilevato sono inferiori a quelli che erano stati preventivati (-56€). Questi soldi risparmiati vengono utilizzati per coprire le spese in eccesso del periodo precedente.\\
Per quanto riguarda lo sviluppo degli incrementi, il \textit{TeamAFK} si ritiene soddisfatto di quanto prodotto in questo periodo.
\pagebreak

\appendix
\section{Organigramma}
\subsection{Redazione} 
\begin{table}[H]
	\begin{center}
	\begin{tabular}{ c c C{6.6cm} }
		\rowcolor{redafk}
		\textcolor{white}{\textbf{Nominativo}} & \textcolor{white}{\textbf{Data di redazione}} & \textcolor{white}{\textbf{Firma}} \\
		Olivier Utshudi & 2020-04-10 & \includegraphics[scale=0.3, width=0.25\textwidth]{img/firme/outshudi.png}\\
		Simone Meneghin & 2020-04-10 & \includegraphics[scale=0.3, width=0.25\textwidth]{img/firme/meneghin.png}\\
		Davide Zilio & 2020-04-10 & \includegraphics[scale=0.2, width=0.2\textwidth]{img/firme/zilio.png}\\
	\end{tabular}
	\end{center}	
\end{table}

\subsection{Approvazione} 
\begin{table}[H]
	\begin{center}
	\begin{tabular}{ c c C{6cm} }
		\rowcolor{redafk}
		\textcolor{white}{\textbf{Nominativo}} & \textcolor{white}{\textbf{Data di approvazione}} & \textcolor{white}{\textbf{Firma}} \\
		Victor Dutca & 2020-04-12 &  \includegraphics[scale=0.3, width=0.25\textwidth]{img/firme/dutca.png} \\
		Tullio Vardanega &  & \\
		Riccardo Cardin &  & \\
	\end{tabular}
	\end{center}	
\end{table}

\subsection{Accettazione dei componenti}
\begin{table}[H]	
	\begin{center}
	\begin{tabular}{ C{5cm} C{4cm} C{6cm}}
		\rowcolor{redafk}
		\textcolor{white}{\textbf{Nominativo}} & \textcolor{white}{\textbf{Data di accettazione}} & \textcolor{white}{\textbf{Firma}} \\
		Simone Federico Bergamin & 2020-03-09 & \includegraphics[scale=0.4]{img/firme/bergamin.png}\\
		Alessandro Canesso & 2020-03-09 & \includegraphics[scale=0.3, width=0.25\textwidth]{img/firme/canesso.png}\\
		Victor Dutca & 2020-03-09 & \includegraphics[scale=0.3, width=0.25\textwidth]{img/firme/dutca.png}\\
		Fouad Farid & 2020-03-09 & \includegraphics[scale=0.2, width=0.2\textwidth]{img/firme/farid.png}\\
		Simone Meneghin & 2020-03-09 & \includegraphics[scale=0.2, width=0.2\textwidth]{img/firme/meneghin.png}\\
		Olivier Utshudi & 2020-03-09 & \includegraphics[scale=0.3, width=0.25\textwidth]{img/firme/outshudi.png}\\
		Davide Zilio & 2020-03-09 & \includegraphics[scale=0.2, width=0.2\textwidth]{img/firme/zilio.png}\\
	\end{tabular}
	\end{center}
\end{table}

\subsection{Componenti}
\begin{table}[H]	
	\begin{center}
	\begin{tabular}{ C{4cm} C{3cm} C{8cm} }
		\rowcolor{redafk}
		\textcolor{white}{\textbf{Nominativo}} & \textcolor{white}{\textbf{Matricola}} & \textcolor{white}{\textbf{Indirizzo email}} \\
		Simone Federico Bergamin & 1144724  & simonefederico.bergamin@studenti.unipd.it \\
		Alessandro Canesso & 1122701 & alessandro.canesso@studenti.unipd.it\\
		Victor Dutca & 1122137 & victor.dutca@studenti.unipd.it\\
		Fouad Farid & 1122195 & fouad.farid@studenti.unipd.it\\
		Simone Meneghin & 1174926 & simone.meneghin@studenti.unipd.it\\
		Olivier Utshudi & 1143556 & olivier.utshudi@studenti.unipd.it\\
		Davide Zilio & 1149807 & davide.zilio.3@studenti.unipd.it\\
	\end{tabular}
	\end{center}
\end{table}
\pagebreak


\end{document}