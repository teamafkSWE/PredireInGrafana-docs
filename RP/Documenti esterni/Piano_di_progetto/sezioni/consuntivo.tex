\section{Consuntivo di periodo}
Di seguito verranno indicate le spese effettivamente sostenute da ogni ruolo. Il bilancio di consuntivo potrà risultare: \begin{itemize}
\item \textbf{Positivo}: se il preventivo supera il consuntivo;
\item \textbf{Pari}: se preventivo e consuntivo sono uguali;
\item \textbf{Negativo}: se il consuntivo supera il preventivo.
\end{itemize}

\subsection{Analisi}
%tabella costi
\begin{table}[H]
\centering\renewcommand{\arraystretch}{1.5}
\caption{Consuntivo del periodo di Analisi}
\vspace{0.2cm}
\begin{tabular}{ c c c }
\rowcolor{redafk}
\textcolor{white}{\textbf{Ruolo}} & \textcolor{white}{\textbf{Ore}} & 
\textcolor{white}{\textbf{Costo}}  \\
Responsabile & 34 & 1020€ \\
Amministratore & 41 (+13) & 820€ (+260€) \\
Analista & 107 (+8) & 2675€ (+200€) \\
Progettista	& 24 & 528€  \\
Programmatore & 0 & 0€  \\
Verificatore & 82 (+9) & 1230€ (+135€)  \\
\textbf{Totale preventivo} & \textbf{288} & \textbf{6273€}  \\
\textbf{Totale consuntivo} & \textbf{318} & \textbf{6868€}  \\
\rowcolor{lastrowcolor}
\textbf{Differenza} & \textbf{30} & \textbf{+595€}  \\
\end{tabular}
\end{table}

\subsubsection{Conclusioni}
Come emerge dai dati riportati nella tabella soprastante è stato necessario investire più tempo del previsto nei ruoli di \textit{Amministratore}, \textit{Analista} e \textit{Verificatore}. Per questo motivo il bilancio risultante è negativo. Le cause di tali ritardi sono riportate di seguito:
\begin{itemize}
\item \textbf{\textit{Amministratore}}: è servito più tempo del previsto per riuscire ad individuare i software più adatti per la gestione del progetto e per la loro  configurazione. Inoltre sono state aggiunte ed aggiornate alcune sezioni nelle \textit{Norme di Progetto}, necessarie al chiarimento di alcune problematiche sorte durante la stesura dei documenti;
\item \textbf{\textit{Analista}}: alcuni requisiti si sono rivelati di non facile comprensione, e sono state necessarie più ore di lavoro per la discussione interna tra gli \textit{Analisti} ed esterna con il proponente;
\item \textbf{\textit{Verificatore}}: l’aggiunta di nuove sezioni nelle \textit{Norme di Progetto} e l'inesperienza dei membri hanno implicato un maggiore lavoro anche per questo ruolo.
\end{itemize}
Il notevole quantitativo di ore che il gruppo ha dovuto impiegare nel primo periodo non deve ripetersi durante il lavoro rendicontato. Per le problematiche riscontrate verranno adottate le seguenti contromisure:
\begin{itemize}
	\item amministrazione degli strumenti: il gruppo ha ricercato e configurato in anticipo gli strumenti che verranno usati. In caso venissero individuati nuovi strumenti avere già un ambiente di sviluppo impostato correttamente per tutti i membri semplificherà la nuova configurazione e l'insorgere di problemi;
	\item comprensione dei requisiti: i requisiti sono stati ampiamente discussi con il proponente durante questa fase, non si prevede di incorrere ulteriormente in tale problema;
	\item applicazione delle norme: i membri del gruppo hanno studiato attentamente le norme, in modo tale da poter redigere fin da subito nuove sezioni dei documenti già normate, semplificando il lavoro ai verificatori e precludendo di dover tornare a correggere il documento
\end{itemize}

\subsubsection{Preventivo a finire}
Essendo questo periodo non rendicontato, non vengono a generarsi problemi nel monte ore totale, nonché nel preventivo economico. Nonostante ciò \textit{TeamAFK} si impegnerà a seguire le contromisure sopra riportate, e facendo esperienza dei problemi riscontrati durante questo primo periodo