\section{Gestione dei rischi}
I rischio viene inteso come un evento che non vorremmo accadesse nel corso del progetto, in quanto influenzerebbe negativamente la qualità o la riuscita stessa del prodotto. Inoltre, essendo un evento che può riguardare qualunque aspetto del progetto, la gestione dei rischi risulta fondamentale per la riuscita dello stesso. Per questo motivo il gruppo intende affrontare il compito nel seguente modo:\\
\begin{itemize}
\item \textbf{Identificazione dei rischi}: vengono identificati i rischi, distinguendoli in rischi per il progetto, il prodotto e l'azienda;
\item \textbf{Analisi dei rischi}: viene valutata la probabilità dell'evento e la sua pericolosità;
\item \textbf{Pianificazione dei rischi}: viene stabilito un piano per la prevenzione del rischio annullandone gli effetti, quando possibile, o per lo meno mitigarne le conseguenze;
\item \textbf{Monitoraggio dei rischi}: ad ogni ridefinizione del \textit{Piano di Progetto}, i rischi vengono nuovamente controllati sulla base delle nuove informazioni.
\end{itemize}

\begin{longtable}{C{3cm} L{4.5cm} L{4.5cm} C{3.15cm}}
\rowcolor{white}\caption{Tabella dei rischi} \\
		\rowcolor{redafk}
\textcolor{white}{\textbf{Codice-Nome}} &
\textcolor{white}{\textbf{Descrizione}} &
\textcolor{white}{\textbf{Rilevamento}} &
\textcolor{white}{\textbf{Grado}}  \\
		\endfirsthead
		\rowcolor{white}\caption[]{(continua)} \\
		\rowcolor{redafk}
\textcolor{white}{\textbf{Codice-Nome}} &
\textcolor{white}{\textbf{Descrizione}} &
\textcolor{white}{\textbf{Rilevamento}} &
\textcolor{white}{\textbf{Grado}} \\
		\endhead
		
RiO01 - Emergenza sanitaria &
Un'epidemia riscontrata nel territorio, può costringere le autorità a porre restrizioni per ridurne l'espansione. &
Le restrizioni descritte dal DCPM 2020-03-08 permettono le sole interazioni telematiche tra gli stakeholders. & 
Probabilità: 
Alta 
Pericolosità: 
Alta \\

Piano di contingenza &
\multicolumn{3}{L{13cm}}{Gli stakeholders dovranno decidere di utilizzare gli strumenti di comunicazione disponibili a tutti che limitino i disagi scaturiti dalle suddette restrizioni.} \\

RiT02 - Inesperienza Tecnologica &
Molte delle tecnologie adottate per lo sviluppo del progetto sono nuove per i componenti, che potrebbero usarle in modo non ottimale. &
Il \textit{Responsabile} ha il compito di essere al corrente delle conoscenze dei componenti. & 
Probabilità: 
Alta
Pericolosità: 
Alta\\ 

Piano di contingenza &
\multicolumn{3}{L{13cm}}{Il \textit{Responsabile} una volta messo al corrente delle  conoscenze dei componenti, affiderà loro i ruoli che più li competono.} \\

RiO03 - Calcolo dei costi &
L'insesperienza del gruppo può portare alla sottovalutazione dei costi da sostenere. &
Il \textit{Responsabile} ha il compito di essere al corrente delle conoscenze dei componenti. & 
Probabilità: 
Media 
Pericolosità: 
Alta\\ 

Piano di contingenza &
\multicolumn{3}{L{13cm}}{È consigliato comunicare tempestivamente al committente la variazione dei costi.} \\

RiO04 - Impegni accademici &
Essendo questo un progetto universitario, è probabile che in corso d'opera i componenti debbano sostenere attività accademiche che li sottrarrebbero dagli impegni di progetto. &
Ogni componente deve saper comunicare con chiarezza i propri impegni accademici. & 
Probabilità: 
Alta
Pericolosità: 
Media \\ 

Piano di contingenza &
\multicolumn{3}{L{13cm}}{È consigliato comunicare tempestivamente al \textit{Responsabile} i propri impegni accademici.} \\

RiO05 - Impegni personali &
\'E possibile che in corso d'opera i componenti debbano sostenere attività che li sottrarrebbero dagli impegni di progetto. &
Ogni componente deve saper comunicare con chiarezza nel calendario i propri impegni. & 
Probabilità: 
Alta
Pericolosità: 
Media \\ 

Piano di contingenza &
\multicolumn{3}{L{13cm}}{È consigliato comunicare tempestivamente al \textit{Responsabile} i propri impegni.} \\


RiO06 - Ritardi &
Le problematiche sopracitate possono comportare ritardi non indifferenti ai fini di progetto. &
Per questo l'incaricato dell'attività deve comunicare tempestivamente il ritardo. & 
Probabilità: 
Media
Pericolosità: 
Bassa \\ 

Piano di contingenza &
\multicolumn{3}{L{13cm}}{È consigliato riassegnare risorse laddove ce ne sia bisogno, e quindi risolvere il motivo del ritardo.} \\

RiP07 - Comunicazione interna &
Può essere che in determinati momenti un elemento del gruppo non sia raggiungibile. &
I membri del gruppo devono segnalare la momentanea assenza dell'interessato/a. & 
Probabilità: 
Bassa
Pericolosità: 
Alta \\ 

Piano di contingenza &
\multicolumn{3}{L{13cm}}{Il gruppo ha adottato diversi mezzi di comunicazione.} \\

RiP08 - Comunicazione esterna &
Se si presentano problematiche come RiO01, il proponente potrebbe non sempre essere reperibile. &
I membri del gruppo organizzeranno le conferenze con il proponente con più largo anticipo. & 
Probabilità: 
Bassa
Pericolosità: 
Alta \\ 

Piano di contingenza &
\multicolumn{3}{L{13cm}}{Il gruppo ha adottato diversi mezzi di comunicazione per rimanere in contatto con il proponente.} \\

RiP09 - Contrasti interni &
Essendo l'attività di progetto un lavoro collaborativo, è possibile che i membri abbiano opinioni divergenti riguardo a determinate tematiche. &
Ciascun membro del team si impegnerà a limitare tali tensioni e fare in modo che esse non influiscano sul normale svolgersi delle attività. & 
Probabilità: 
Bassa
Pericolosità: 
Alta \\ 

Piano di contingenza &
\multicolumn{3}{L{13cm}}{Il responsabile avrà la funzione di gestire e fare da mediatore in tali divergente.} \\

RiT10 - Errori nelle dipendenze &
Il progetto richiede parecchie dipendenze esterne, potrebbero essere presenti errori all'interno di esse. & 
Al momento dell'aggiunta di una nuova dipendenza ci si deve informare su errori già conosciuti. &
Probabilità:
Bassa
Pericolosità:
Alta \\

Piano di contingenza &
\multicolumn{3}{L{13cm}}{Sarà necessario cercare una versione di tale dipendenza che non contenga l'errore. Valutare anche la possibilità di cambiare dipendenza con una analoga.}\\

RiT11 - Mancanza di documentazione &
La piattaforma di Grafana non fornisce una vasta documentazione. & 
Prima dello sviluppo di un nuovo componente si deve verificare che sia presente la relativa documentazione o un esempio del codice. &
Probabilità:
Media
Pericolosità:
Media \\

Piano di contingenza &
\multicolumn{3}{L{13cm}}{Consultarsi con gli altri membri per suggerimenti sullo sviluppo e cercare un metodo alternativo per l'implementazione} \\
\end{longtable}
\pagebreak
\subsection{Attuazione dei rischi}
Nella seguente tabella vengono riportati i rischi in cui il gruppo si è imbattuto:

\begin{longtable}{C{3cm} L{5cm} L{7cm}}
\rowcolor{white}\caption{Attuazione dei rischi} \\
		\rowcolor{redafk}
\textcolor{white}{\textbf{Rischio}} &
\textcolor{white}{\textbf{Descrizione}} &
\textcolor{white}{\textbf{Contromisura}}\\
		\endfirsthead
		\rowcolor{white}\caption[]{(continua)} \\
		\rowcolor{redafk}
\textcolor{white}{\textbf{Rischio}} &
\textcolor{white}{\textbf{Descrizione}} &
\textcolor{white}{\textbf{Contromisura}}\\
		\endhead
RiO01 - Emergenza sanitaria	& L'epidemia ha costretto gli stakeholders ad attuare lo smart working. & Sono stati usati vari mezzi di comunicazione, in particolare si ha optato per applicazioni che permettessero comunicazioni rapide e già conosciute così da ridurre il disagio al minimo.
\\
RiT02 - Inespreienza tecnologica & I programmatori non conoscevano a pieno i linguaggi e le librerie che sono state utilizzate & \'E stato suddiviso il lavoro in modo da rispettare le conoscenze dei membri. In caso di nessuna conoscenza precedente, si è suddiviso il compito di studiare le documentazioni, per poi spiegarle agli altri membri.
\\
RiO04 - Impegni accademici & Un membro del gruppo ha dovuto svolgere un esame & Durante la breve mancanza di un membro il resto del gruppo si è dedicato all'approfondimento e allo studio delle tecnologie utilizzate.
\\
RiT10 - Errori nelle dipendenze & Un cambiamento all'interno degli strumenti forniti da Grafana per sviluppare il plugin ha causato l'impossibilità di effettuare la build del prodotto. & Si è proceduto al passaggio ad una versione precedente di tali strumenti.
\end{longtable}