\documentclass[a4paper, oneside, openany, dvipsnames, table, 12pt]{article}
\usepackage{../../../../Template/AFKstyle}
\usepackage{hyperref}
\usepackage{verbatim} %per commenti di più righe \begin{comment} \end{comment}
\usepackage{amsmath}
\newcommand{\Titolo}{Piano di Progetto}

\newcommand{\Gruppo}{TeamAFK}

\newcommand{\Redattori}{Simone Meneghin \newline Davide Zilio}

\newcommand{\Verificatori}{Victor Dutca}

\newcommand{\pathimg}{../../../Template/img/logoAFK.png}

\newcommand{\Approvatore}{Alessandro Canesso}

\newcommand{\Distribuzione}{Prof. Vardanega Tullio \newline Prof. Cardin Riccardo \newline Gruppo AFK}

\newcommand{\Uso}{Esterno}

\newcommand{\NomeProgetto}{"Predire in Grafana"}

\newcommand{\Mail}{gruppoafk15@gmail.com}

\newcommand{\Versionedoc}{4.0.0}

\newcommand{\DescrizioneDoc}{Descrizione della pianificazione delle attività del gruppo \textit{TeamAFK} nella realizzazione del progetto \textit{Predire in Grafana}.}


\makeindex

\begin{document}
\copertina{}

%------------------ COLORI TABELLE 
\definecolor{pari}{RGB}{255, 207, 158} %{HTML}{E1F5FE} %azzurrino
\definecolor{dispari}{HTML}{FAFAFA} %bianco/grigetto 

%definizione colori per tabelle (tranne copertina)
\definecolor{redafk}{RGB}{255, 133, 51}
\definecolor{grey2}{RGB}{204, 204, 204}
\definecolor{greyRowafk}{RGB}{234, 234, 234}
\definecolor{lastrowcolor}{RGB}{176, 196, 222} %steel blue %{255,165,0} orange %{RGB}{255, 207, 158}
\rowcolors{2}{pari}{dispari}
\renewcommand{\arraystretch}{1.5}

%------------------

\newpage
\section*{Registro delle modifiche}
{
	\centering
	\begin{longtable}{ c c C{4cm}  C{4cm}  C{3cm} }
		\rowcolor{redafk}
		\textcolor{white}{\textbf{Versione}} & \textcolor{white}{\textbf{Data}} & \textcolor{white}{\textbf{Descrizione}} & \textcolor{white}{\textbf{Nominativo}} & \textcolor{white}{\textbf{Ruolo}}\\		
		1.0.0 & 2020-07-08 & Approvazione documento & Olivier Utshudi &\RdP{}\\		
		0.1.0 & 2020-07-08 & Stesura e verifica documento. & Davide Zilio \newline Olivier Utshudi &\reda{} \newline \ver{} \\		
		
	\end{longtable}

}


%Didascalia tabelle/immagini (prendono come riferimento la subsection)
\counterwithin{table}{subsection}
\counterwithin{figure}{subsection}
\newpage

%indice, indice figure e indice tabelle
\tableofcontents
\newpage
\begin{comment}
\listoffigures
\newpage
\listoftables
\newpage
\end{comment}

\section{Informazioni generali}
\subsection{Informazioni incontro}
\begin{itemize}
\item \textbf{Luogo}: Discord;
\item \textbf{Data}: 2020-07-01;
\item \textbf{Ora di inizio}: 15:00;
\item \textbf{Ora di fine}: 18:00;
\item \textbf{Partecipanti}:
	\begin{itemize}
		\item tutti i membri.
	\end{itemize}
\end{itemize}

\subsection{Topic}
Durante questo incontro il gruppo ha discusso sulle seguenti tematiche:
\begin{enumerate}
\item sviluppo degli incrementi rimanenti;
\item domande da porre al proponente;
\item scegliere se tradurre tutto in inglese o tenere il prodotto in italiano.
\end{enumerate}

\subsubsection{Punto 1}
Il gruppo ha discusso lo sviluppo degli ultimi 2 incrementi rimanenti, i quali rappresentano le ultime funzionalità da inserire nel plug-in di Grafana\glo.
L'incremento 13 riguarda il completamento del pannello di collegamento mentre l'incremento 14 implementa tutti i messaggi di errori o notifica mancanti per guidare l'utente passo passo in ogni sua operazione. \\
È stato deciso pertanto di inserire: \begin{itemize}
\item alert quando si tenta di aggiungere un file json nel plug-in ma uno è gia stato inserito: "Watch out! A file is already imported!";
\item alert "Disconnection done.";
\item alert "Saving started." quando si clicca "Enable saving.";
\item alert "Saving stopped." quando si clicca "Disable saving.";
\item alert "Monitoring started.";
\item alert "Monitoring stopped.".
\end{itemize}

\subsubsection{Punto 2}
Le seguenti domande verranno poste al proponente nella prossima riunione: \begin{enumerate}
\item Il tool lo preferisce come pagina web o come eseguibile?
\item Le soglie devono essere implementate? 
\end{enumerate}
Verrà inoltre mostrato al proponente quanto sviluppato finora.

\subsubsection{Punto 3}
Il \textit{TeamAFK} ha deciso di tradurre il tool esterno e il plug-in di Grafana in lingua inglese, per uniformare il prodotto finale e renderlo utilizzabile da chiunque.

\subsection{Tracciamento delle decisioni}
\begin{longtable}{ c C{12cm} }
\rowcolor{redafk}
\textcolor{white}{\textbf{Codice}} & \textcolor{white}{\textbf{Decisione}}\\	
		VI\_2020-07-01\_1 & Revisione e sviluppo incrementi 13 e 14. Inserimento alert mancanti (definiti al punto 1).\\
		VI\_2020-07-01\_2 & Domande da porre al proponente (definite al punto 2). \\
		VI\_2020-07-01\_3 & Scelto l'inglese come lingua principale del prodotto finale.
\end{longtable}
\pagebreak

\end{document}