\section{Informazioni generali}
\subsection{Informazioni incontro}
\begin{itemize}
\item \textbf{Luogo}: Skype;
\item \textbf{Data}: 2020-04-29;
\item \textbf{Ora di inizio}: 14:30;
\item \textbf{Ora di fine}: 14:45;
\item \textbf{Partecipanti}:
	\begin{itemize}
		\item tutti i membri;
		\item Gregorio Piccoli (proponente).
	\end{itemize}
\end{itemize}

\subsection{Topic}
Esposizione di alcune domande e conseguente richiesta di chiarimenti, relativi allo sviluppo della Proof of Concept da presentare in Revisione di Progettazione, prevista per il 2020-05-18.
In particolare le domande riguardano:
\begin{itemize}
	\item il tool esterno deve essere accessibile da grafana?\\
	Risposta: non necessariamente, deve essere sviluppato nel modo più facile per il team;
	\item il tool deve essere un eseguibile o va bene come pagina web? \\
	Risposta: come preferiamo noi, non c'è un obbligo particolare;
	\item che tipologia di plug-in dobbiamo sviluppare? Un plug-in App o Panel? \\
	Risposta: Panel;
	\item dobbiamo usare le dashboard con i pannelli di grafana o possiamo usare librerie javascript per i grafici? \\
	Risposta: utilizzare i pannelli di Grafana;
	\item cosa vuole vedere lei nel PoC? \\
	Risposta: far vedere qualcosa che abbia una sua logica;
	\item ha una documentazione più approfondita di Grafana? Quello che troviamo è poco chiaro e confuso.\\
	Risposta: no, purtroppo essendo un software open source è in continuo mutamento e di conseguenza la documentazione presente è scarna e poco approfondita.
\end{itemize}
A seguito delle risposte ricevute, il \textit{TeamAFK} è riuscito a risolvere i dubbi e quindi a continuare la progettazione e sviluppo dell'architettura software.