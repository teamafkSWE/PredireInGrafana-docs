\section{Gestione dei rischi}
I rischio viene inteso come l'evento che non vorremmo accadesse nel corso di un progetto, in quanto influenzerebbe in maniera negativa sulla qualità, o sulla riuscita stessa del prodotto. Inoltre, essendo un evento che può riguardare qualunque aspetto del progetto, la gestione dei rischi risulta fondamentale per la riuscita dello stesso. Per questo motivo il gruppo intende affrontare questo compito nel seguente modo:\\
\begin{itemize}
\item \textbf{Identificazione dei rischi}: vengono identificati i rischi, distinguendoli in rischi per il progetto, il prodotto e l'azienda;
\item \textbf{Analisi dei rischi}: viene valutata la probabilità dell'evento e la sua pericolosità;
\item \textbf{Pianificazione dei rischi}: viene stabilito un piano per la prevenzione del rischio annullandone gli effetti, quando possibile, o per lo meno mitigarne le conseguenze;
\item \textbf{Monitoraggio dei rischi}: ad ogni ridefinizione del \textit{Piano di Progetto}, i rischi vengono nuovamente controllati sulla base delle nuove informazioni.
\end{itemize}

\begin{longtable}{C{3cm} L{4.5cm} L{4.5cm} C{3.15cm}}
\rowcolor{white}\caption{Tabella dei rischi} \\
		\rowcolor{redafk}
\textcolor{white}{\textbf{Codice-Nome}} &
\textcolor{white}{\textbf{Descrizione}} &
\textcolor{white}{\textbf{Rilevamento}} &
\textcolor{white}{\textbf{Grado}}  \\
		\endfirsthead
		\rowcolor{white}\caption[]{(continua)} \\
		\rowcolor{redafk}
\textcolor{white}{\textbf{Codice-Nome}} &
\textcolor{white}{\textbf{Descrizione}} &
\textcolor{white}{\textbf{Rilevamento}} &
\textcolor{white}{\textbf{Grado}} \\
		\endhead
		
RiO40 - Emergenza sanitaria &
Un'epidemia riscontrata nel territorio, può costringere le autorità a porre restrizioni per ridurne l'espansione. &
Le restrizioni descritte dal DCPM 2020-03-08 permettono le sole interazioni telematiche tra gli stakeholders. & 
Probabilità: 2 
Pericolosità: 2 \\

Piano di contingenza &
\multicolumn{3}{L{13cm}}{Gli stakeholders dovranno decidere di utilizzare gli strumenti di comunicazione disponibili a tutti che limitino i disagi scaturiti dalle suddette restrizioni.} \\

RiT41 - Inesperienza Tecnologica &
Molte delle tecnologie adottate per lo sviluppo del progetto sono nuove per i componenti, che potrebbero usarle in modo non ottimale. &
Il \textit{Responsabile} ha il compito di essere al corrente delle conoscenze dei componenti. & 
Probabilità: 
2 
Pericolosità: 
2\\ 

Piano di contingenza &
\multicolumn{3}{L{13cm}}{Il \textit{Responsabile} una volta messo al corrente delle  conoscenze dei componenti, affiderà loro i ruoli che più li competono.} \\

RiO32 - Calcolo dei costi &
L'insesperienza del gruppo può portare alla sottovalutazione dei costi da sostenere. &
Il \textit{Responsabile} ha il compito di essere al corrente delle conoscenze dei componenti. & 
Probabilità: 
1 
Pericolosità: 
2\\ 

Piano di contingenza &
\multicolumn{3}{L{13cm}}{È consigliato comunicare tempestivamente al committente la variazione dei costi.} \\

RiO33 - Impegni accademici &
Essendo questo un progetto universitario, è probabile che in corso d'opera i componenti debbano sostenere attività accademiche che li sottrarrebbero dagli impegni di progetto. &
Ogni componente deve saper comunicare con chiarezza quelli che sono i propri impegni accademici. & 
Probabilità: 
2
Pericolosità: 
1 \\ 

Piano di contingenza &
\multicolumn{3}{L{13cm}}{È consigliato comunicare tempestivamente al \textit{Responsabile} i propri impegni accademici.} \\

RiO34 - Impegni personali &
\'E possibile che in corso d'opera i componeti debbano sostenere attività che li sottrarrebbero, dagli impegni di progetto. &
Ogni componente deve saper comunicare con chiarezza nel calendario quelli che sono i propri impegni. & 
Probabilità: 
2
Pericolosità: 
1 \\ 

Piano di contingenza &
\multicolumn{3}{L{13cm}}{È consigliato comunicare tempestivamente al \textit{Responsabile} i propri impegni.} \\


RiO15 - Ritardi &
Le problematiche sopracitate possono comportare ritardi non indifferenti ai fini di progetto. &
Per questo l'incaricato dell'attività deve comunicare tempestivamente il ritardo. & 
Probabilità: 
1
Pericolosità: 
0 \\ 

Piano di contingenza &
\multicolumn{3}{L{13cm}}{È consigliato riassegnare risorse laddove ce ne sia bisogno, e quindi risolvere il motivo del ritardo.} \\

RiI26 - Comunicazione interna &
Può essere che in determinati momenti un elemento del gruppo non sia raggiungibile. &
I membri del gruppo devono segnalare la momentanea assenza dell'interessato/a. & 
Probabilità: 
0
Pericolosità: 
2 \\ 

Piano di contingenza &
\multicolumn{3}{L{13cm}}{Il gruppo ha adottato diversi mezzi di comunicazione.} \\

RiI26 - Comunicazione esterna &
Se si presentano problematiche come RO40, il proponente potrebbe non sempre essere reperibile. &
I membri del gruppo organizzeranno le conferenze con il proponente con più largo anticipo. & 
Probabilità: 
0
Pericolosità: 
2 \\ 

Piano di contingenza &
\multicolumn{3}{L{13cm}}{Il gruppo ha adottato diversi mezzi di comunicazione per rimanere in contatto con il proponente.} \\

RiI37 - Contrasti interni &
Essendo l'attività di progetto un lavoro collaborativo, è possibile che i membri abbiano opinioni divergenti riguardo a determinate tematiche. &
Ciascun membro del team si impegnerà a limitare tali tensioni e fare in modo che esse non influiscano sul normale svolgersi delle attività. & 
Probabilità: 
0
Pericolosità: 
2 \\ 

Piano di contingenza &
\multicolumn{3}{L{13cm}}{Il responsabile avrà la funzione di gestire e fare da mediatore in tali divergente.} \\

\end{longtable}

