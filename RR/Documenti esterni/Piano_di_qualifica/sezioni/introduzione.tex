\section{Introduzione}
\subsection{Premessa}
Il \textit{Piano di Qualifica} è un documento su cui si prevedono continui aggiornamenti o modifiche durante l'intera durata del progetto. Molti dei contenuti del documento sono di natura instabile. Ad esempio molte delle metriche scelte non sono applicabili nella fase iniziale e solo con il loro utilizzo pratico si può valutarne l'effettiva utilità. Anche i processi selezionati possono essere soggetti a cambiamenti, rivelandosi insufficienti o inadeguati agli scopi del progetto e al modo di lavorare del team. Il documento è stato scritto in diversi periodi in quanto alcuni aspetti non si potevano conoscere a priori. \\
Per tutte queste ragioni, il documento è prodotto in maniera incrementale\glo, e i suoi contenuti iniziali sono da considerarsi incompleti: subiranno significative aggiunte e modifiche nel tempo.

\subsection{Scopo del documento}
Questo documento ha lo scopo di mostrare le strategie di verifica\glo e validazione\glo adottate al fine di garantire la qualità di prodotto e di processo\glo. Per raggiungere questo obiettivo viene applicato un sistema di verifica continua sui processi in corso e sulle attività\glo svolte. In questo modo è quindi possibile rilevare e correggere all'istante eventuali anomalie, riducendo al minimo lo spreco delle risorse.

\subsection{Scopo del prodotto}
Lo scopo del prodotto è quello di realizzare un plug-in\glo per il software Grafana\glo. Tale plug-in, utilizzando la Regressione Lineare\glo o la Support Vector Machine\glo addestrata dall'utente mediante un tool esterno, permetterà di monitorare e predire lo stato di un sistema in analisi. Grazie alle predizioni sarà possibile attivare degli allarmi così da poter gestire preventivamente eventuali situazioni di rischio. \\



\subsection{Glossario}
Per evitare ambiguità nei documenti formali, viene fornito il documento \textit{Glossario\_v1.0.0},
contenente tutti i termini considerati di difficile comprensione. Perciò nella documentazione fornita ogni vocabolo contenuto in Glossario è contrassegnato dalla lettera G a pedice.

\subsection{Riferimenti}
\subsubsection{Riferimenti normativi}
\begin{itemize}
	\item Norme di Progetto: \textit{Norme\_di\_Progetto\_v1.1.0};
	\item ISO/IEC 9126: \url{https://en.wikipedia.org/wiki/ISO/IEC_9126};
	\item ISO/IEC 15504: \url{https://en.wikipedia.org/wiki/ISO/IEC_15504}.
\end{itemize}
\subsubsection{Riferimenti informativi}
\begin{itemize}
	\item Capitolato d'appalto C4: \url{https://www.math.unipd.it/~tullio/IS-1/2019/Progetto/C4.pdf}.
	\item \textbf{Slide L12 del Corso Ingegneria del Software}:\\
	\url{https://www.math.unipd.it/~tullio/IS-1/2019/Dispense/L12.pdf};
	\item \textbf{Slide L13 del Corso Ingegneria del Software}:\\
	\url{https://www.math.unipd.it/~tullio/IS-1/2019/Dispense/L13.pdf};
	\item \textbf{Ingegneria del Software - Ian Sommerville - 10\textsuperscript{a} Edizione}.
\end{itemize}