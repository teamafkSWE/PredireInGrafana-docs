\section{Capitolato C3 - Natural API}

\subsection{Descrizione generale}
\textit{"What if I try to withdraw 20 Euros from the ATM could semi-automatically be converted to a function with a signature like withdraw (user, amount, source)?"} \\
Data l’ambiguità del linguaggio naturale che spesso comporta fraintendimenti tra stakeholders\glo, "Natural API" propone un collegamento univoco tra il linguaggio umano\glo e linguaggio di programmazione\glo, permettendo in futuro lo sviluppo di librerie\glo più consistenti, prevedibili e manutenibili\glo.


\subsection{Obiettivi}
L’obiettivo di questo progetto è fornire uno strumento Proof-of-Concept\glo che riduca il divario (gap) tra specifiche di progetto e API\glo, generando librerie complete e facilmente manutenibili e i relativi test di unità e di integrazione, ricoprendo i linguaggi di programmazione e i frameworks\glo più popolari.
NaturalAPI dovrebbe inoltre permettere agli sviluppatori di concentrarsi sull’implementazione dei metodi piuttosto che sulla modellazione dell’intero sistema, attraverso: \begin{enumerate}
\item ricerca di combinazioni tra nomi e verbi (Gherkin);
\item normalizzazione e conversione di predicati in funzioni e metodi;
\item ricerca di argomenti di funzioni ricorrenti al fine di generare oggetti e proprietà.
\end{enumerate}
Ogni Natural API deve essere accessibile in almeno due tra le seguenti modalità: \begin{itemize}
\item interfaccia web REST\glo;
\item un’interfaccia grafica minimale;
\item una command line interface\glo.
\end{itemize}

\subsection{Tecnologie utilizzate}
\begin{itemize}
\item Per la generazione di API e DLS: \begin{itemize}
\item Swagger: strumento per la generazione di codice basato in OpenAPI;
\item OWL v2: formato per la rappresentazione ontologica;
\item OpenAPI.
\end{itemize}
\item Behaviour Driven Development (BDD\glo): \begin{itemize}
\item Hiptest: è una piattaforma cloud di gestione dei test che supporta la continuous delivery\glo;
\item Gherkin: parser\glo di linguaggio di Cucumber;
\item Cucumber: è uno strumento software che supporta il BDD.
\end{itemize}
\end{itemize}

\subsection{Valutazione generale}
L’idea è sembrata molto interessante e all’avanguardia, utile ai team di sviluppo per essere più efficienti, riducendo comunicazioni superflue o possibili fraintendimenti.  \\
Nonostante ciò, il progetto è stato valutato come troppo ambizioso in quanto potrebbe comportare un probabile sforamento delle scadenze.  \\
È stato perciò deciso di non considerarlo come prima scelta.



