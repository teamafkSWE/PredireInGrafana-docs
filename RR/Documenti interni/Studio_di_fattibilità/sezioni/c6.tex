\section{Capitolato C6 - ThiReMa}

\subsection{Descrizione generale}
La continua necessità delle aziende di avere soluzioni sempre più versatili dovuta dalla gestione di grandi quantità di dati richiede un software che indirizzi nel modo giusto le informazioni agli stakeholders.   

\subsection{Obiettivi}
Il progetto richiede un software, che verrà realizzato attraverso una web-application, in grado di ricevere misurazioni da vari sensori (temperatura, tensione di corrente, etc) e di accumularli efficientemente e in maniera affidabile in un database centralizzato che contiene due macro-categorie: dati operativi e fattori influenzanti. \\
La web-application dovrà esser suddivisa in tre macro-sezioni:
\begin{itemize}
\item censimento dei sensori e dei relativi dati;
\item modulo di analisi di correlazione;
\item modulo di monitoraggio per ente.
\end{itemize}
Gli utenti avranno a disposizione un’interfaccia grafica che gli permetterà di seguire in tempo reale l’andamento dei sensori e visualizzare dati di interesse.

\subsection{Tecnologie utilizzate}
\begin{itemize}
\item Kafka: cluster\glo distribuito per lo streaming\glo di dati in tempo reale;
\item PostgreSQL, TimescaleDB, ClickHouse: database\glo suggeriti per la raccolta dei dati;
\item BootStrap: per il front-end\glo styling;
\item Docker: container per l’istanziazione di tutti i componenti;
\item Java: sviluppo della business logic\glo.
\end{itemize}

\subsection{Valutazione generale}
Mettendo a confronto questo progetto con il quarto capitolato proposto da \textit{Zucchetti SPA - "Predire in Grafana"}, è stata notata una notevole somiglianza sullo scopo del progetto, incentrato sulla raccolta e la gestione del flusso dei dati.  \\
Dopo averli esaminati e aver valutato aspetti positivi e negativi di entrambi, \textit{ThiReMa} è stato preso in considerazione come possibile candidato. \\
Purtroppo però tale scelta non è stata possibile data la disponibilità terminata del proponente.







